\section{User Guide}

This guide introduces the common types and categories provided by \libalgebra,
and presupposes some familiarity with programming in \aldor and \libaldor.
If you are unfamiliar with \aldor or \libaldor, we suggest that you first go
through the tutorial in {\tt aldorlib/tutorial/}, which
will familiarize you both with \aldor and \libaldor.

\libalgebra{} provides over 180 categories and domains, and over 400 different
exported functions, which can look daunting at first. Remember however
that a large part of those are low--level functionalities that make it
possible to programmers to write efficient applications. Most of the
high-level functionalities are found in a small subset of types and
those are the ones described in this short guide. As you become more
familiar with \libalgebra, you will find the reference guide to be more
useful when browsed on line.

As you learn programming with \libalgebra, make sure to check the
{\tt algebra/samples} directory for various programming samples.

\subsection{Basic algebraic categories}
The basic algebraic categories provided by \libalgebra{} are shown in
Figure~\ref{fig:cats}.
While it follows quite closely the usual algebraic structures, a few
points needs to be mentioned.
\begin{figure}[htb!]
\begin{center}
\includegraphics{algbcat}
\end{center}
\caption{The \libalgebra{} basic algebraic category hierarchy}
\label{fig:cats}
\end{figure}
\begin{itemize}
\item
The multiplication $\ast$ is in general not assumed to be commutative,
except in the subtree starting at \altype{CommutativeRing}, which appears
in blue in Figure~\ref{fig:cats}.
This means that code written for the other categories
should be careful and not assume commutativity of $\ast$. Addition is on
the other hand always assumed commutative.
\item
The \libalgebra{} category tree has a unique root
\altype{ExpressionType} to which
almost all the types provided by \libalgebra{} belong. The category
\altype{ExpressionType} inherits \altype{PrimitiveType} from \libaldor,
which means that types in \libalgebra{} must export an equality.
That equality however does not need to be complete, as described
in the \alfunc{PrimitiveType}{$=$} page of the \libaldor{} reference manual.
\altype{ExpressionType} plays also an important role in input/output as
explained below.
\item
The \libaldor{} categories \altype{AdditiveType},
\altype{ArithmeticType} and \altype{LinearCombinationType}
already export the basic arithmetic operations provided respectively
by abelian groups, rings and modules. In fact
\altype{AbelianGroup}, \altype{Ring} and \altype{Module}
do inherit their exports from them. There is however
a fundamental difference: while the \libaldor{} categories
export the usual operations $=$, $+$, $-$, $\ast$,\dots,
they make no assumption about their algebraic properties whatsoever.
On the other hand, the corresponding algebraic categories
in \libalgebra{} do assume that $=$ is a complete equality test, and that
the arithmetic operations satisfy the algebraic axioms of their
mathematical structure. So for example, while \altype{Integer} is
extended by \libalgebra{} to be a \altype{Ring} (among other things),
\altype{SingleFloat} remains an \altype{ArithmeticType} but is
not a \altype{Ring}.
There is a similar relationship between \altype{LinearArithmeticType}
and \altype{Algebra}, both provided by \libalgebra.
Mathematical types such as
\altype{DenseMatrix}$(R)$ or
\altype{DenseUnivariatePolynomial}$(R)$
allow their argument $R$ to be an \altype{ArithmeticType} rather
than a \altype{Ring}, but those among their functions that require
a complete equality on $R$ are not exported when $R$ is not a \altype{Ring}.
Similarly, while \altype{DenseUnivariatePolynomial}$(R)$ is always an
\altype{ArithmeticType}, it is a \altype{Ring} only when $R$ itself
is a \altype{Ring}. For example,\\
\vspace{-5mm}
\begin{center}
\altype{DenseMatrix}~\altype{DenseUnivariatePolynomial}~\altype{SingleFloat}
\end{center}
is a valid type in \libalgebra, but Gaussian elimination is not available for
such matrices, and the Euclidean algorithm is not available for their
polynomial entries. Basic arithmetic, as provided by \altype{ArithmeticType}
is however available.
\end{itemize}

\subsection{Arithmetic}
All of the arithmetic types provided by \libaldor (machine and software
integers, floats, as well as \altype{Complex}) remain available
in \libalgebra, and most of them are extended to various algebraic categories.
As in \libaldor, {\tt Integer} is actually a macro that defaults to
\altype{AldorInteger}. If integer efficiency is important for your application,
we strongly recommend that you link with the GMP version of
\libalgebra instead,
see the section on~\alalias{\this}{sec:gmp}{using GMP} for
more details. Regardless of the integer implementation that you choose,
we also recommended that you use~\alexttype{MachineInteger}{} whenever
appropriate, in particular for loop
or data structure indices. Conversions between
\alexttype{MachineInteger}{}
and \altype{Integer} are provided by \alfunc{IntegerType}{coerce}
and \alfunc{IntegerType}{machine}.

In addition, \libalgebra{}
provides two different implementations of the finite field $\ZZ / p\ZZ$
when $p$ is a machine prime: \altype{SmallPrimeField} provides a
standard implementation, while \altype{ZechPrimeField} provides
a significantly faster implementation based on a logarithmic
representation of its elements.
However, \altype{ZechPrimeField}
precomputes a table of size ${\cal O}(p)$ so it should only be used
for reasonably small values of $p$ and when a significant amount
of calculations in $\ZZ / p\ZZ$ are made following the creation
of the type (we have found the precomputation time to be under one second
for $p$ around $10^6$ on recent workstations).

The type \altype{Fraction}$(R)$ implements the fraction field of
the \altype{GcdDomain} $R$. Typical arguments are \altype{Integer}
(to get the rational numbers) or polynomial types. Note that fractions
are automatically normalized after each arithmetic operation.


\subsection{Data structures}
All the data structure of \libalgebra{} are provided by \libaldor. Some of them
are extended by mathematical operations and take on a new name:
\altype{Sequence} corresponds to \altype{Stream} from \libaldor{} and
\altype{Vector} corresponds to \altype{Array} from \libaldor,
in both cases with pointwise arithmetic operations added.
Note however that \altype{Array} is $0$--indexed while \altype{Vector}
is $1$--indexed.

\subsection{Input/Output}
\label{sec:inputoutput}
\libalgebra{} inherits the stream I/O model provided by \libaldor. In particular
\altype{ExpressionType} inherits \altype{OutputType} from \libaldor, which means
that elements of a \altype{ExpressionType} can be written in text format
to any \altype{TextWriter} (in particular I/O streams, strings or files).
Because it is desirable to output mathematical objects such as
matrices and polynomials in more than one format depending on the
context, \libalgebra{} types do not in general define their own
\alfunc{OutputType}{$<<$} function. Rather, they implement
the \alfunc{ExpressionType}{extree} function, that converts their elements
into elements of \altype{ExpressionTree}. Expression trees,
wich are similar to Lisp's S--expressions, are the unique layer
between all the types in \libalgebra{} and the outside world. When
writing your own types, once you provide or inherit an implementation
of \alfunc{ExpressionType}{extree}, your elements can then be written
to any \altype{TextWriter} in any of the many formats that
\altype{ExpressionTree} understands, for example using
\alfunc{ExpressionTree}{lisp} to produce Lisp output.
In that case, the default behavior of the \alfunc{OutputType}{$<<$}
is to convert your objects to expression trees and then to
use \alfunc{ExpressionTree}{tex} to produce \TeX output. You
can override this default behaviour by providing your own
implementation of \alfunc{OutputType}{$<<$} if you choose.
As with \libaldor, you can use {\tt \#include "aldorio"} {\em after}
{\tt \#include "algebra"} in order to import the types commonly used
for input and output.

\subsection{Linear algebra}
The basic types to use for doing linear algebra
are \altype{Vector} and \altype{DenseMatrix}.
The basic linear algebra computations are provided
by the type \altype{LinearAlgebra}. Although
\libalgebra{} provides many specialized types for
performing various sorts of triangulations and solving
linear systems, \altype{LinearAlgebra} knows about all of
them and contains procedures that decide which algorithm is
best suited for your particular input. So do not call
directly the more specialized packages unless you have
a particular reason to use a specific algorithm rather
than let \libalgebra{} decide for you.

The type of the entries of vectors and matrices does
not need to be a \altype{Ring}, it can be an
\altype{AdditiveType} or \altype{ArithmeticType} respectively.
This allows types that do not have a full equality such
as \altype{SingleFloat} or \altype{DenseUnivariateTaylorSeries}
to be entries of vectors and matrices. However,
\altype{LinearAlgebra} requires the entries to be from
a \altype{CommutativeRing} so its functionalities are
not available for matrices of series or floating point numbers.

\libalgebra{} contains many fraction--free algorithms that
allow you to perform
some computations, such as matrix inverses or solving systems,
over an \altype{IntegralDomain} rather than over a \altype{Field}
as is usually done.
Those fraction--free
algorithms are significantly faster over integral domains than
over their fraction fields, so consider using domains
such as $\ZZ$ or $R[x]$ as matrix entries rather than
fractions. See for example the description of the
\alfunc{LinearAlgebra}{inverse} function to see the
effect of working over domains rather than fields
on the signatures of the functions.

To write generic linear algebra code that does not depend
on the implementation of matrices, use a type parameter
of category \altype{MatrixCategory}. For example, a normal
form package could look like:
\begin{verbatim}
NormalForms(R:IntegralDomain, M:MatrixCategory R): with {
        frobenius: M -> M;     -- Returns the Frobenius form of its argument
        ...
} == add { ... }
\end{verbatim}

\subsection{Univariate polynomials and series}
The basic types to use for computing with polynomials and
series are \altype{DenseUnivariatePolynomial}
and \altype{DenseUnivariateTaylorSeries} respectively.
Both types are univariate but can be nested if needed to
produce dense multivariate polynomials and series with
a fixed number of variables. When the number of variables
is too large for a dense representation, you can also use
\altype{SparseUnivariatePolynomial} but be aware that for
univariate or bivariate use, its arithmetic is much less 
efficient than the one of \altype{DenseUnivariatePolynomial}.

As for matrices, the type of the coefficients of polynomials
or series does not need to be a \altype{Ring}, it can be an
\altype{ArithmeticType} instead.
This allows types that do not have a full equality such
as \altype{SingleFloat} to be used as coefficients, but
some polynomial functionalities are only available when
the coefficient type is a \altype{Ring} or something stronger.
The polynomial and series types take a \altype{Symbol} as
second argument. That symbol is used only for output when
converting the polynomial or series to an \altype{ExpressionTree},
so it is not necessary to give one when you use polynomials
or series inside a calculation. If you insist on naming the
variable, use \alfunc{Symbol}{-} with any string as argument
to create a symbol. For example,
{\tt DenseUnivariatePolynomial(Integer)} and
{\tt SparseUnivariatePolynomial(Fraction Integer, -"x")}
are both valid polynomial types.
Whether you give a name for a variable or not, you
can override that choice using \alfunc{MonogenicLinearArithmeticType}{apply}
with a \altype{Symbol} or \altype{ExpressionTree} as argument.
For example, if $p$ is a polynomial or series,
{\tt stdout(p, -"z")} writes $p$ to \alfunc{TextWriter}{stdout}
using ``z'' as variable name, and {\tt p(extree leaf(-"y"))}
returns the \altype{ExpressionTree} corresponding to $p$ with
``y'' as variable name.

To write generic code for manipulating polynomials or series
use a type parameter usually of category
\altype{UnivariatePolynomialCategory}
or \altype{UnivariateTaylorSeriesCategory} respectively.
Because polynomials, skew--polynomials and series share
many common operations, \libalgebra{} provides in fact a complete
category hierarchy for them, shown in Figure~\ref{fig:stdpolcat}.
\begin{figure}[htb!]
\begin{center}
\includegraphics{algpolcat}
\end{center}
\caption{The \libalgebra{} univariate polynomial category hierarchy}
\label{fig:stdpolcat}
\end{figure}
Those categories make it possible to write functions
that work for polynomials, skew--polynomials, series
or any combination of them.
\altype{UnivariatePolynomialCategory}$(R)$ is the category
for types whose elements are usual polynomials of
the form $\sum_n r_n x^n$ with finite support. While
$R$ is not assumed to be commutative, the generator $x$
is assumed to commute with coefficients in $R$, \ie~$rx = xr$.
Similarly, \altype{UnivariateTaylorSeriesCategory}$(R)$
is the category for types whose elements are series of
the form $\sum_n r_n x^n$. Here also, $R$ is not assumed
to be commutative, but the generator $x$ commutes with
coefficients in $R$. Those two categories are shown
in blue in Figure~\ref{fig:stdpolcat}, as they are the only ones assuming
that $x$ commutes with $R$.
The more general category \altype{UnivariatePolynomialAlgebra}$(R)$
is for types whose elements are sums of the form $\sum_n r_n x^n$ with
finite support, but where $x$ does not necessarily commute with $R$.
Polynomials and skew-polynomials are both in that category.
Even more general, \altype{MonogenicAlgebra}$(R)$
is for types whose elements are sums of the form $\sum_n r_n P_n$ with
finite support, but where the family $P_n$ is not necessarily the
power basis $x^n$. An example of such a type is
\altype{UnivariateFactorialPolynomial} for which
$P_n = x(x-1)\dots(x-n+1)$. Finally, \altype{MonogenicLinearArithmeticType}$(R)$
is for types whose elements are potentially infinite sums of the form
$\sum_n r_n P_n$. Code written at that level works for polynomials, series
and skew-polynomials as well.

The elements of \altype{DenseUnivariateTaylorSeries} are lazy series
represented by their coefficient sequences, themselves of type
\altype{Sequence}. Those coefficient sequences are in turn represented
as \altype{Stream} from \libaldor. Therefore, the usual way to write a
function producing a series is to produce first its coefficient stream $s$,
then call \alfunc{Sequence}{sequence} on $s$ to produce the
coefficient sequence, and finally call
\alfunc{UnivariateTaylorSeriesCategory}{series} on the coefficient sequence
to produce the series. Since streams are lazy, constructing a series
does not compute any of its coefficients until they are
specifically requested by another operation. There are several ways
to create streams, all documented in the \libaldor{} reference manual,
and you should become familiar with them before programming with series.
For example, the following function takes constants
$a_0$ and $c$ and produces the hypergeometric series
$\sum_{n\ge 0} a_n x^n$ where $a_{n+1}/a_n = c$.
\begin{verbatim}
SeriesSample(R:CommutativeRing, Rx:UnivariateTaylorSeriesType R): with {
         hypergeometricSeries: (R, R) -> Rx;
} == add {
         hypergeometricSeries(a0:R, c:R):Rx == {
                import from Sequence R;
                zero? a0 => 0;
                -- the following creates the stream [a0, c a0, c^2 a0, ... ]
                coeffs:Stream R := orbit(a0, (x:R):R +-> c * x);
                series sequence coeffs;
         }
}
\end{verbatim}
Finally, the type \altype{UnivariateTaylorSeriesCategory2Poly} provides
conversions between univariate polynomials and series.

\subsection{Multivariate polynomials}
The basic types for computing with multivariate polynomials
are \altype{SparseMultivariatePolynomial} and
\altype{DistributedMultivariatePolynomial1}.
Both types use a sparse representation.
Elements of the former one are regarded as univariate polynomials
with polynomial coefficients.
Elements of the latter one are regarded as lists of terms
where a term is the product of a coefficient by a power product
of variables.

To construct a multivariate polynomial type with 
\altype{SparseMultivariatePolynomial}
one needs a \altype{Ring} as  a coefficient type 
and a \altype{VariableType} as a variable type.
\begin{verbatim}
#include "algebra"
import from String, Symbol;
macro V == OrderedVariableTuple(-"x",-"y",-"z");
import from MachineInteger, V;
macro P == SparseMultivariatePolynomial(Integer,V);
x: P := variable(1)$V :: P;
y: P := variable(2)$V :: P;
z: P := variable(3)$V :: P;
p := (y + z)*x^2 + (y^3 + z)*x + z^4;
\end{verbatim}

To construct a multivariate polynomial type with 
\altype{DistributedMultivariatePolynomial1}
one needs a \altype{Ring} as  a coefficient type ,
a \altype{VariableType} as a variable type.
and \altype{ExponentCategory(V)} as an exponent type.
So we can continue the above sample program as follows
\begin{verbatim}
macro E1 == MachineIntegerDegreeLexicographicalExponent(V);
macro E2 == MachineIntegerLexicographicalExponent(V);
macro Q1 == DistributedMultivariatePolynomial1(Integer, V, E1);
macro Q2 == DistributedMultivariatePolynomial1(Integer, V, E2);
q1 := expand(E1)(Q1)(p);
q2 := expand(E2)(Q2)(p);
\end{verbatim}

The above polynomials appear respectively as follows
in an interpreter session.
\begin{verbatim}
%20 >> q1 := expand(E1)(Q1)(p)
x*y^3 + z^4 + x^2*y + x^2*z + x*z @ Q1

%21 >> q2 := expand(E2)(Q2)(p);
x^2*y + x^2*z + x*y^3 + x*z + z^4 @ Q2
\end{verbatim}
Do not forget to enter
\begin{verbatim}
#include "aldorinterp"
\end{verbatim}
after 
\begin{verbatim}
#include "algebra"
\end{verbatim}
in every interpreter session.

There are other multivariate polynomial types 
like \altype{RecursiveMultivariatePolynomial0}.
It is similar to \altype{SparseMultivariatePolynomial}
but takes a third argument which is a univariate polynomial
type constructor:
\begin{verbatim}
  UP: (R: Join(ArithmeticType, ExpressionType), avar: Symbol == new()
       ) -> UnivariatePolynomialCategory(R),
\end{verbatim}
This third argument is used for the internal representation.

Another useful type is \altype{IntegerPolynomial}
which provides multivariate polynomial with integer coefficient
and variables from \altype{OrderedSymbol}.
This multivariate polynomial type is easy to use as
shown by the following interpreter session.
\begin{verbatim}
%1 >> #include "algebra"
                                           Comp: 130 msec, Interp: 0 msec
%2 >> #include "aldorinterp"
                                           Comp: 70 msec, Interp: 0 msec
%3 >> macro P == IntegerPolynomial;
                                           Comp: 0 msec, Interp: 0 msec
%4 >> import from String, P, Integer;
                                           Comp: 210 msec, Interp: 0 msec
%5 >> x: P := "x" :: P
x @ IntegerPolynomial
                                           Comp: 0 msec, Interp: 520 msec
%6 >> y: P := "y" :: P        
y @ IntegerPolynomial
                                           Comp: 0 msec, Interp: 0 msec
%7 >> z: P := "z" ::P
z @ IntegerPolynomial
                                           Comp: 0 msec, Interp: 0 msec
%8 >> p := (y + z)*x^2 + (y^3 + z)*x + z^4 + 3::P;
z^4+(x^2+x)*z+x*y^3+x^2*y+3 @ IntegerPolynomial
\end{verbatim}
Observe the {\tt 3::P}. Indeed, it is necessary to convert
the integer {\tt 3} into a polynomial.
Observe also the ordering on symbols is reversed w.r.t. the dictionnary one.

\subsection{Compatibility with C types}
All the types of \libaldor{} that are compatible with their C counterparts
remain so inside \libalgebra. In addition, if you are using the GMP version
of the library, then \altype{Integer} and \altype{Float}
are compatible with {\tt mpz\_t} and {\tt mpf\_t} respectively.

\subsection{Using GMP}
\altarget{sec:gmp}
As in \libaldor, the type {\tt Integer} is actually a macro,
which defaults to \altype{AldorInteger}, the software integers provided
by the \aldor{} runtime. For efficiency or other reasons, you may prefer
to use the GMP library, which is supported by \libalgebra.
The easiest way
to use GMP is to compile all your source files with the option {\tt -dGMP}
and then to use the options
\begin{center}
{\tt -lalgebra-gmp -laldor -cruntime=foam-gmp,gmp -y\$ALGEBRAROOT/lib}
\end{center}
when linking your final executable. All you need is GMP 3.0 or later
installed in a file called {\tt libgmp.a} to produce executables.
Using GMP generally produces more efficient programs,
but programs calling GMP cannot be interpreted by the \aldor{}
compiler, nor can they run inside its interactive loop.

Using the {\tt -dGMP} option allows you to compile the same sources
either with or without GMP, which can be appreciable, but you must
ensure that you do not mix files compiled with and without {\tt -dGMP} since
the macro {\tt Integer} would then be expanded into two different types.

An additional advantage of using GMP, is that \altype{GMPInteger}
exports and uses internally several of the in--place or higher--level
operations of GMP, which are not available with \altype{AldorInteger}.
In addition, variables of type \altype{GMPInteger} are compatible
with the C type {\tt mpz\_t} from GMP, so you can directly call
C programs that use GMP from your \aldor{} code.

\subsection{Exceptions}
In addition to the exceptions thrown by \libaldor, \libalgebra throws a
\altype{ReducibleModulusException} when a divisor of zero
is discovered in a \altype{SimpleAlgebraicExtension}. This can be used
to implement algorithms based on lazy factorisation, since such an
exception contains a non--trivial factor of the polynomial defining
the extension.

\subsection{Profiling and debugging}
\altarget{sec:debug}
The macros TIME, TRACE and AGAT provided by \libaldor{} remain available
in \libalgebra. In addition,
\libalgebra{} also comes with a debug version, which makes many assertions
about the arguments of the functions called as well as their results.
While those assertions slow down the code considerably they tend to
be quite useful when chasing bugs since the release version of \libalgebra{}
does not make validity checks to most of its inputs.
The debug version of \libalgebra{} must be used jointly with the debug
version of \libaldor. To use them, just compile your application with the
\begin{center}
{\tt -lalgebrad -laldord -dDEBUG}
\end{center}
options rather than {\tt -lalgebra -laldor}.
It is also preferable when debugging to add the {\tt -q1} option
in order to prevent inlining.

