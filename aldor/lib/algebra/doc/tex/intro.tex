\section{Introduction}

\subsection*{What is \stdmath?}
\stdmath{} is a general-purpose computer algebra library designed to
provide reusable and efficient algorithms for manipulating the standard
objects of algebra, namely polynomials, series and matrices.
Built as an extension of the \salli library, it provides \aldor
programmers with an extensible computer algebra layer with a rich
data type hierarchy.

\subsection*{How do I get and install \stdmath?}
You can download \stdmath by anonymous ftp from the \cafe server at
{\tt ftp-sop.inria.fr} in {\tt cafe/software/algebra},
or from the URL:\\
\vspace{-5mm}
\begin{center}
\url{http://www.inria.fr/cafe/Manuel.Bronstein/algebra/}
\end{center}
After downloading the file {\tt algebra.tar.gz}, issue
``{\tt tar -xzvf algebra.tar.gz}'' in order to unpack it.
This will create the following directories:
\begin{itemize}
\item{\tt algebra/doc}: this user guide,
\item{\tt algebra/lib}: the library,
\item{\tt algebra/include}: the required include files,
\item{\tt algebra/test}: some test files,
\item{\tt algebra/samples}: \stdmath programming samples,
\end{itemize}
Once the above file is unpacked, do the following:
\begin{itemize}
\item add the option {\tt -csys=XXX} to your ALDORARGS environment variable,
where {\tt XXX}
depends on your hardware and operating system. Common values for {\tt XXX}
are {\tt axposf1v4} for OSF1 V4.0 on a DEC Alpha, {\tt linux-486} for linux on
a 486 or above PC, and {\tt sun4os55g-v8} for SunOS 5.5 on a SPARC v8 machine.
See the file {\tt \$ALDORROOT/include/aldor.conf} for other values;
\item go to {\tt algebra/lib} and execute {\tt sh makealgebra};
shell;
\item if you want to build the GMP version of the library,
execute {\tt sh makealgebra-gmp}.
See the subsection on~\alalias{\this}{sec:gmp}{using GMP}
for more information about using the GMP version of \libalgebra;
\item if you want to build the debug version of the library,
execute {\tt sh makealgebrad}.
See the subsection on~\alalias{\this}{sec:debug}{debugging}
for more information on using the debug library.
\end{itemize}

\subsection*{How do I use \stdmath in my programs?}
Once \stdmath is properly built, you need to set the following environment
variables before using it:
\begin{itemize}
\item the environment variable ALGEBRAROOT should be set to
the main {\tt algebra} directory;
\item {\tt \$ALGEBRAROOT/include}
should be appended to your INCPATH environment variable;
\item {\tt \$ALGEBRAROOT/lib}
should be appended to your LIBPATH environment variable;
\end{itemize}
In your \aldor programs, use {\tt \#include "algebra"} instead of
{\tt \#include "aldor"}.
When building your final executable, add the options
\begin{center}
{\tt -lalgebra -laldor -y\$ALGEBRAROOT/lib}
\end{center}
to your compiler command line, or
\begin{center}
{\tt -lalgebrad -laldord -dDEBUG -y\$ALGEBRAROOT/lib}
\end{center}
to link to the debug version of \libalgebra.
Check the subsection on~\alalias{\this}{sec:gmp}{using GMP}
for the options required if you want to use the GMP library and the GMP
version of \libalgebra.

If you are running \stdmath inside the compiler interactive loop, then
type the line
\begin{center}
{\tt \#include "aldorinterp"}
\end{center}
immediately after {\tt \#include "algebra"},
which will import various things for interactive use and make the interpreter
loop print values automatically.
As with any \aldor program, do not forget
the {\tt -q} option in order to optimize your programs, specially
if performance is an issue.

Before using \stdmath for the first time, please check your installation
by running {\tt make} in the {\tt algebra/test} directory, followed by
running {\tt testall}.

Please report any installation problem or bugs you encounter
to {\tt sumit@sophia.inria.fr}.
