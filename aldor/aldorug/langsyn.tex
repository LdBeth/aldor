%%%%%%%%%%%%%%%%%%%%%%%%%%%%%%%%%%%%%%%%%%%%%%%%%%%%%%%%%%%%%%%%%%%%%%%%%%%%%%
%%%%%%%%%%%%%%%%%%%%%%%%%%%%%%%%%%%%%%%%%%%%%%%%%%%%%%%%%%%%%%%%%%%%%%%%%%%%%%
%%%                                                                        %%%
%%% :: Basic Syntax
%%%                                                                        %%%
%%%%%%%%%%%%%%%%%%%%%%%%%%%%%%%%%%%%%%%%%%%%%%%%%%%%%%%%%%%%%%%%%%%%%%%%%%%%%%
%%%%%%%%%%%%%%%%%%%%%%%%%%%%%%%%%%%%%%%%%%%%%%%%%%%%%%%%%%%%%%%%%%%%%%%%%%%%%%

\head{chapter}{Basic syntax}{asugLangSyntax}

% \head{section}{Syntax Components}			% Done
% \head{section}{Escape Character}			% Done
% \head{section}{Keywords}				% Done
% \head{section}{Names:  Identifiers and Operators}	% Done
% \head{section}{Comments and Descriptions}		% Done
% \head{section}{Application syntax}			% Done
% \head{section}{Grouping}				% Done
% \head{section}{Piles}				% Done

%%%%%%%%%%%%%%%%%%%%%%%%%%%%%%%%%%%%%%%%%%%%%%%%%%%%%%%%%%%%%%%%%%%%%%%%%%%%%%
\head{section}{Syntax components}{asugLangElementsSyntax}
%%%%%%%%%%%%%%%%%%%%%%%%%%%%%%%%%%%%%%%%%%%%%%%%%%%%%%%%%%%%%%%%%%%%%%%%%%%%%%

An \asharp{} program consists of a series of lines of text.
These lines of text may be stored in a single file, or gathered
from several files, or typed in as interactive input. 

Some lines are not part of the \asharp{} program proper, but instead
control its composition and the environment in which it is handled.  
These lines are called {\em system commands}.  A system command%
\index{system command} is a line which has a hash character
\ttin{\#}\index{\verb+#+ (system command)} as its {\em first}
character.  (Note that no white space may precede the \ttin{\#} on the
line.)

The example programs in this chapter use the following system commands:

\begin{small}
\begin{verbatim}
    #include "filename.as"
    #pile
    #endpile
\end{verbatim}
\end{small}

The system command {\tt \#include "filename.as"} causes the lines of text
from \fname{filename.as} to be inserted into the \asharp{} program in place
of the {\tt \#include} command.

The system commands {\tt \#pile} and {\tt \#endpile} are used to enclose
lines of text in which indentation is used to determine the nesting of
sequences of \asharp{} expressions.  (See \secref{asugFLangPiles}.)

A complete list of system commands is given in \secref{asugFLangSource}.
System commands used in the interactive interpreter are described in
\secref{asugInteractSyscmds}.

\index{token}
%\pagebreak
When the series of lines comprising an \asharp{} program has been
collected together, these lines are interpreted as a series of words,
or {\em tokens}.
There are several classes of tokens, each of which has a different meaning:
\begin{description}
\item[Identifiers] such as \ttin{Fred} and \ttin{rgf32}
are used as names for variables and constants.
\item[Literals] such as \verb"42", \verb"1.414" and \verb+"Urania riphaeus"+,
represent explicit values.  Literals are described in
\spadref{asugLangExprLiterals}.
\item[Keywords] such as \ttin{if} and \ttin{==} each have a special meaning
in the language, and impose a special structure on neighbouring expressions.
\item[Operators] such as \ttin{by} and \ttin{+} have special
syntactic properties, but are otherwise the same as identifiers 
(\ie{}~they are used as names for variables and constants).
\item[Comments] are used to insert free-form text into a program.
A comment begins with the two characters \ttin{--} and continues until
the end of the current line of text.
\item[Descriptions] are used to provide user-level documentation for
functions, domains and categories defined in the program.
A description begins with the two characters \ttin{++} and continues
until the end of the current line of text.
\item[White space] consists of spaces, tabs, and newlines.  White space is
used to determine source position (line and column) information for message
reporting, and for piling (See \secref{asugFLangPiles}).
\end{description}

The exact rules for the syntax of each of these token classes is given
in \secref{asugFLangLexTokens}.

%%%%%%%%%%%%%%%%%%%%%%%%%%%%%%%%%%%%%%%%%%%%%%%%%%%%%%%%%%%%%%%%%%%%%%%%%%%%%%
\head{section}{Escape character}{asugLangElementsEscape}
%%%%%%%%%%%%%%%%%%%%%%%%%%%%%%%%%%%%%%%%%%%%%%%%%%%%%%%%%%%%%%%%%%%%%%%%%%%%%%

\index{escape character}
The underscore character \ttin{\_} is used as an {\em escape character\/}
in \asharp{} to modify the interpretation of the characters which follow.
For example, an escape character followed by any amount of white space
(spaces, tabs, and newlines) causes the white space to be ignored, allowing
the characters on either side of the white space to form a single token,
such as a name or a literal.

\Secref{asugLangExprNames} describes how the escape character can be used
inside an identifier, and \secref{asugLangExprLiterals} describes
how the escape character can be used inside a literal.

%%%%%%%%%%%%%%%%%%%%%%%%%%%%%%%%%%%%%%%%%%%%%%%%%%%%%%%%%%%%%%%%%%%%%%%%%%%%%%
\head{section}{Keywords}{asugLangElementsKeywords}
%%%%%%%%%%%%%%%%%%%%%%%%%%%%%%%%%%%%%%%%%%%%%%%%%%%%%%%%%%%%%%%%%%%%%%%%%%%%%%

The basic components of any \asharp{} program can be separated into
two broad categories:  those which are defined by the language, and
those which may be defined or redefined by the program.
For example, the meaning of the word \ttin{if} is defined by the
language, and all \ttin{if} statements behave according to
the same rules.
On the other hand, the meaning of a name such as \ttin{a} or \ttin{9}
or \ttin{+} is determined by the program in which it is used.

\index{keywords}
A {\em keyword\/} in \asharp{} is a word whose meaning is fixed by the
definition of the language.  The following words are keywords which
may not be redefined:

\begin{verbatim}
add      and       always    assert    break
but      catch     default   define    delay
do       else      except    export    extend 
fix      for       fluid     free      from     
generate goto      has       if        import
in       inline    is        isnt      iterate
let      local     macro     never     not 
of       or        pretend   ref       repeat
return   rule      select    then      throw
to       try       where     while     with
yield

.   ,    ;    :    ::   :*   $    @
|   =>   +->  :=   ==   ==>  '
[   ]    {    }    (    )    
\end{verbatim}

Generally, language-defined aspects of keywords offer protocols
which allow them
to work with new types as well as with language-defined types.  
So, for example, the language-defined \ttin{if}, provides a way for
the condition to be an expression which evaluates to any type, provided
that type has certain properties.

The following keywords are meaningless in the current language
definition, but are reserved for future language extensions.

\begin{verbatim}
delay    fix       is        isnt       let       rule      

(|  |)   [|  |]    {|  |}    `    &    ||
\end{verbatim}

%%%%%%%%%%%%%%%%%%%%%%%%%%%%%%%%%%%%%%%%%%%%%%%%%%%%%%%%%%%%%%%%%%%%%%%%%%%%%%
\head{section}{Names: identifiers and operators}{asugLangElementsNames}
%%%%%%%%%%%%%%%%%%%%%%%%%%%%%%%%%%%%%%%%%%%%%%%%%%%%%%%%%%%%%%%%%%%%%%%%%%%%%%

\index{name}
\index{identifier}
A {\em name\/} is an identifier used to denote a variable or a constant.
Most names begin with a letter or the character \ttin{\%} and are made up of
letters, digits and the characters \ttin{\%}, \ttin{?} and \ttin{!}.
The words \ttin{0} and \ttin{1} are also treated as names in \asharp{}
so that mathematical structures can export identity elements without
having to support integer literals.  (See \secref{asugLangExprLiterals}.)

Examples:

\begin{small}
\begin{verbatim}
    mylist       Integer       empty?      set!      %5
\end{verbatim}
\end{small}

\index{escape character}
Any character may be included in a name by preceding it with the
escape character (\ttin{\_}):

\begin{small}
\begin{verbatim}
    _*PACKAGE_*   _42skidoo    mod_+       _*_+
\end{verbatim}
\end{small}

When used in an identifier, the escape character is not included in the name
of the identifier.  To include a single underscore character in the name
of an identifier, the sequence \ttin{\_\_} must be used.  So the name of the
identifier denoted by \ttin{mod\_+} is \ttin{mod+}, and the name of the
identifier denoted by \ttin{My\_\_Integer} is \ttin{My\_Integer}.

A sequence of letters which would otherwise be considered a keyword
(such as \ttin{if}) can be treated as a name by escaping one of its
constituent letters (as in \ttin{\_if}).

\index{infix operator}
Certain names are treated as having special syntax properties by the language.
The following identifiers can be used as infix operators, prefix operators,
or both:

%\begin{small}
\begin{verbatim}
    by      case     mod      quo      rem
    #        +       -        +-       ~        ^
    *        **      ..       =        ~=       ^=
    /        /\      <        <=       <<       <-
    \        \/      >        >=       >>       ->
\end{verbatim}
%\end{small}

Aside from their syntactic properties,
these names behave just as other identifiers.
See \secref{asugLangTiesAppSyntax} for examples of using infix operators
in different contexts.

A few naming conventions are used in the standard libraries:

\begin{itemize}
\item 
  Names beginning with capital letters are used for types or
  type-producing functions.
\item
  Names ending with a question mark are used for Boolean values,
  or functions which return Boolean values.
\item
  Names ending with an exclamation mark are used for functions whose
  primary purpose is to perform a side-effecting (in particular, a
  so-called ``destructive'') operation.
%\item
%  Infixed names are used for functions.
% I don't understand that sentence.  MGR
\end{itemize}

Note that these are only notational conventions and are not considered
as part of the language.

%%%%%%%%%%%%%%%%%%%%%%%%%%%%%%%%%%%%%%%%%%%%%%%%%%%%%%%%%%%%%%%%%%%%%%%%%%%%%%
\head{section}{Comments and descriptions}{asugLangElementsComments}
%%%%%%%%%%%%%%%%%%%%%%%%%%%%%%%%%%%%%%%%%%%%%%%%%%%%%%%%%%%%%%%%%%%%%%%%%%%%%%

Comments and description strings
annotate a program to help other people and other programs understand it.  

\index{comment}
A {\em comment\/} begins with the two characters
\ttin{--}\index{\verb"--"} and continues until the end of the current
line of text. 
Comments can be used to describe how a program operates, including
an explanation of special assumptions made by the program, or a
step-by-step description of the implementation of the algorithms
used by the program.
Comments are not examined by the compiler, and do not affect the meaning
of a program.

A {\em description\/} begins with two or three plus characters
(\ttin{++}\index{\verb"++"} or \ttin{+++}\index{\verb"+++"})
and continues until the end of the current line of text.
A description should be used to describe the external characteristics
of a program, such as the parameters it will accept or the method used
to compute the result.

\index{description}
Description strings are saved in the compiler output in a form
accessible by other programs.
If a description begins with three plus characters (\ttin{+++}),
then the name it describes should appear immediately after the description.
If a description begins with only two plus characters (\ttin{++}),
then the name it describes should appear immediately before the description:

\begin{small}
\begin{verbatim}
    +++ An approximation to Euler's constant,
    +++ which is defined as the value of the limit
    +++
    +++   lim(n->infinity) (1 + 1/2 + 1/3 + ... + 1/n - ln n)
    +++
    gamma: DoubleFloat == 0.57721_56649_01532_86060_65121;

    +++ `pi' is the ratio of a circle's circumference to its diameter.
    pi: DoubleFloat == 3.14159_26535_89793_23846_26434;
        ++ This is not 22/7.

    Avogadro: DoubleFloat == 6.022e23;
        ++ The ratio between grams and molecular weights.
\end{verbatim}
\end{small}

Both \ttin{+++} and \ttin{++} are used so that after a semicolon we can
still associate a description with the previous declaration.
% I don't understand what this is meant to say.
% (Surely, only "++" associates a description with
% a PREVIOUS declaration; "+++" associates it with a forthcoming
% declaration.)  MGR

It is easy to remember the difference between comments and descriptions:
the \asharp{} compiler keeps positive remarks, and throws away negative ones.

Example:
\begin{small}
\begin{verbatim}
    -- This is a quick-and-dirty move generator, with two of
    -- the utility functions also made visible.

    ChessPiece: with {
            bestMoves: (Board, %) -> MoveTemplate;
                    ++ `bestMoves p' suggests the best moves in the
                    ++ given position.

            legalMoves: (Board, %) -> MoveTemplate;
                    ++ `legalMoves p' generates quasi-legal moves.
                    ++ It does not handle en passant or castling.

            value: (Board, %) -> DoubleFloat;
                    ++ This is a score which estimates the current
                    ++ value in the given position.
    }
    == add {
            ...
    }
\end{verbatim}
\end{small}

%%%%%%%%%%%%%%%%%%%%%%%%%%%%%%%%%%%%%%%%%%%%%%%%%%%%%%%%%%%%%%%%%%%%%%%%%%%%%%
\head{section}{Application syntax}{asugLangTiesAppSyntax}
%%%%%%%%%%%%%%%%%%%%%%%%%%%%%%%%%%%%%%%%%%%%%%%%%%%%%%%%%%%%%%%%%%%%%%%%%%%%%%

\index{function application}
Applications are typically used to denote function calls, array indexing,
or element accessors for compound data types.

A prefix application typically has the following form:

\begin{small}
\begin{verbatim}
    f(a1, ..., an)
\end{verbatim}
\end{small}

There are two additional forms for specifying a prefix 
application to one argument: juxtaposition\index{juxtaposition}
and an infix dot.

\begin{small}
\begin{verbatim}
    f a
    f.a
\end{verbatim}
\end{small}

The second of these forms is completely equivalent to {\tt f(a)}; the
first is equivalent in a free-standing occurrence but associates
differently --- to the right, rather than the left:

\begin{small}
\begin{verbatim}
    f a b c             -- is equivalent to (f (a (b c)))
    f.a.b.c             -- is equivalent to (((f.a).b).c)
    f(a)(b)(c)          -- is equivalent to ((f(a))(b))(c)
\end{verbatim}
\end{small}

Any application in which the argument is enclosed in parentheses
(\ttin{( )}) or square brackets (\ttin{[ ]}) is treated as being of
the ``typical'' form, and so associates to the left --- even if a
space follows the applied object.  Thus \ttin{first [1,2,3]} is treated
as identical with \ttin{first([1,2,3])}.

The interpretation of mixed forms is determined by precedence rules:
the precedence of juxtaposition is lower than that of the other forms,
which are all equivalent, so an expression such as
\ttin{f(a).2(b)(c).x y} is associated as
\ttin{((((((f(a)).2)(b))(c)).x) y)}.  (A complete table of \asharp{}
operator precedence appears in \secref{asugLangOpPrecTab}.)

\index{infix operator}
Infix operators are applied to a pair of arguments using infix notation
for function application:

\begin{small}
\begin{verbatim}
    a + b               -- infix notation for a call to `(+)(a, b)'
\end{verbatim}
\end{small}

Infix operators are {\em generic\/} in that they can be given definitions
in \asharp{} programs just as other identifiers.  The typical form for an
{\em infix function definition} is as follows:

\begin{small}
\begin{verbatim}
    (s1: S1) op (s2: S2) : T == E
\end{verbatim}
\end{small}

where \ttin{op} is one of the infix operators listed in \secref{asugLangElementsNames}.

An infix operator can be used in any context where other names can be used.
However, in some contexts the infix operator must be enclosed in parentheses
to suppress its normal syntactic properties:

\begin{small}
\begin{verbatim}
    -- Here is a declaration for `*'.
    *: (%, %) -> %

    -- Here `*' is used as a variable.
    * := myMultiplicationMethod

    -- Here is a typical use of `*' as an infix operator.
    a * b

    -- Here `*' is used as a prefix operator
    -- by enclosing it in parentheses.
    (*)(a, b)

    -- Here `*' is passed as an argument to a function.
    reduce(*, l)

    -- Here `*' is being used as an argument of another infix operator.
    f := (*) + g(*, 1)
\end{verbatim}
\end{small}

An infix operator must be enclosed in parentheses to be used as a prefix
operator.  Also, an infix operator cannot appear as an argument of another
infix operator unless it it enclosed in parentheses.

Alternatively, the same name may be given as an {\em identifier\/},
rather than an infixed operator, using the escape character
to include special characters, for example: \verb^_*(a, b)^.

%%%%%%%%%%%%%%%%%%%%%%%%%%%%%%%%%%%%%%%%%%%%%%%%%%%%%%%%%%%%%%%%%%%%%%%%%%%%%%
\head{section}{Grouping}{asugLangElementaryGrouping}
%%%%%%%%%%%%%%%%%%%%%%%%%%%%%%%%%%%%%%%%%%%%%%%%%%%%%%%%%%%%%%%%%%%%%%%%%%%%%%

Complex expressions in \asharp{} are formed according to the precedence
of the operators appearing in the expression.  When an expression is
formed, the operators with higher precedence form the subexpressions for
the operators with lower precedence.

Parentheses (\ttin{( )}) and braces (\ttin{\{ \}}) can be used to override
the natural precedence order defined by the language.
\index{parentheses}
\index{braces}
\index{grouping}

Because comma has a lower precedence than most other
syntactic forms, it is often necessary to enclose comma-separated
expressions in parentheses.  We write \ttin{f(1, 2)}, since
\ttin{f 1, 2} would be associated the same way as \ttin{(f 1), 2}.

Likewise, we write \ttin{(a, b) := (1, 2)}
(see \secref{asugLangExprMultis} for an
explanation of this notation),
since \ttin{a, b := 1, 2} would be associated as \ttin{a, (b := 1), 2}. 

Similarly, the expression \ttin{(1 + 2) * 3} evaluates to {\tt 9},
while the expression \ttin{1 + 2 * 3} evaluates to {\tt 7}, since
the \ttin{*} operator has a higher precedence than the \ttin{+} operator.

Braces are normally used to enclose sequences of expressions
(see \secref{asugLangExprSequences}):

\begin{small}
\begin{verbatim}
    foo(x: Integer, y: Integer): Integer == {
            a := x * y;
            b := a * a;
            b
    }
\end{verbatim}
\end{small}

The meaning of an expression is the same whether braces or parentheses
are used.  Braces are normally used to enclose a longer expression
(especially sequences) split over several lines.
Parentheses are normally used to enclose shorter expressions
(especially multiple values--see \secref{asugLangExprMultis})
as part of other expressions.


An implicit semicolon is assumed, if possible, after a closing brace.
This is determined by whether the following token may start a new
expression. For instance, in the construct

\begin{small}
\begin{verbatim}
f: with {...} == add ...
\end{verbatim}
\end{small}

introduced in \secref{asugLangNTypeCats}, the \ttin{==} may not start an
expression, so no semicolon is assumed.

%\pagebreak
To make the use of braces as natural as possible, an expression
in braces may not be used as an argument to an infixed operator,
\eg{}~\ttin{+}, \ttin{-}, \ttin{..}.
This is because many infixed operators may also be used in prefixed
position.  (Some, incidentally, may also be postfixed.)  With
infixed operators, parentheses may be used to achieve the
desired effect --- for example:

\begin{small}
\begin{verbatim}
 {a; b; c}   + d     -- not ok
({a; b; c})  + d     -- ok
( a; b; c )  + d     -- ok
\end{verbatim}
\end{small}

%%%%%%%%%%%%%%%%%%%%%%%%%%%%%%%%%%%%%%%%%%%%%%%%%%%%%%%%%%%%%%%%%%%%%%%%%%%%%%
\head{subsection}{Precedence}{asugLangOpPrecTab}
%%%%%%%%%%%%%%%%%%%%%%%%%%%%%%%%%%%%%%%%%%%%%%%%%%%%%%%%%%%%%%%%%%%%%%%%%%%%%%

\Figref{asugLangPrecTable} provides a table of keywords and
operators, % More of former.
given in order of syntactic precedence.
Expressions are represented
by ``$e\/$'' and keywords or operators by ``$\circ$''.%

Each of the numbered entries in the table lists syntactic elements
with the same precedence.
The entries at the top of the table
group most loosely, and those at the bottom most strongly.
So, for example, since \ttin{+} is above \ttin{*}, the expression

\begin{small}
\begin{verbatim}
a * b + c * d
\end{verbatim}
\end{small}

groups as \ttin{(a*b) + (c*d)}.
Entries with the same level number have the same precedence.
For instance, ``\verb"and"'' and ``\verb"/\"'' have the same grouping strength.

Some operators have both unary and binary forms.
Some of these operators have meanings defined in the standard base
libraries (\eg{}~infixed \ttin{+} and \ttin{*}), while others do not
(\eg{}~prefixed \ttin{=} and \ttin{+-}).
A programmer may provide new meanings for operators, but not for keywords.
%Operators are indicated in the table with a ``\AssDef'' symbol.
Entries for operators are flagged with ``\dag''; keyword entries are unflagged.

This table can serve as a convenient reference for determining relative
% changed partly to get rid of an overfull box.
strengths of keywords and operators. 
The full details of the language syntax are given in \chapref{asugFLang}.

\index{associativity}\index{unary operators}

\input{precedence.tbl}

\clearpage
%%%%%%%%%%%%%%%%%%%%%%%%%%%%%%%%%%%%%%%%%%%%%%%%%%%%%%%%%%%%%%%%%%%%%%%%%%%%%%%
\head{section}{Piles}{asugFLangPiles}
%%%%%%%%%%%%%%%%%%%%%%%%%%%%%%%%%%%%%%%%%%%%%%%%%%%%%%%%%%%%%%%%%%%%%%%%%%%%%%%

\index{piling}
Programmers often use indentation to make the visual structure of a program
conform to its syntactic structure, so that programs are easier to read.
In \asharp{}, white space is usually ignored except to delimit tokens
and to compute source position information.  However, the compiler can
be instructed to use indentation as part of the syntax of the language
using a scheme known as {\em piling}.

Two system commands are used to instruct the compiler to enable and
disable piling syntax, as desired, at various points in an \asharp{}
program.  The system command \ttin{\#pile} instructs
the compiler to use piling syntax for the source lines which follow, and
the system command \ttin{\#endpile} instructs the compiler to ignore
initial white space on the source lines which follow.
% If it ignored white space, "f x" would be read as the name "fx".  MGR

Although the system commands \ttin{\#pile} and \ttin{\#endpile} are
usually found in pairs, the \ttin{\#endpile} system command can be
omitted at the end of a file.

When piling syntax is being used, indentation is treated roughly as
follows (see \secref{asugFLangLayout} for full details);
a further example of an \asharp{} program which uses piling syntax can be found
at page \pageref{factorialSample}.

Expressions which are indented by the same amount are grouped together
as a sequence (see \secref{asugLangExprSequences}) as though
%\secref{asugLangExprSequences} does not use the term "sequence
% expression", only "sequence".  MGR
they were enclosed in braces and separated by semicolons.  A sequence
of expressions indented by the same amount is called a {\em pile}.

\goodbreak
An expression which is too large to fit on one line at the current
indentation level can be continued on another line by indenting the
continuation line more than the initial line.

The indentation rules are applied first to the most indented lines,
working outward to the lines which are indented the least.

The following example shows the piling rules being applied to a program
which uses piling syntax, to convert it to an equivalent program
which does not use piling syntax:

\begin{small}
\begin{verbatim}
    #pile
    GetUp()
    if Saturday then
        CookBreakfast()
        Eat << Toast << Tomato << Bacon
            << Eggs
    HaveShower()
    DrinkCoffee()
\end{verbatim}
\end{small}

Because the line \ttin{<< Eggs} is indented with respect to the
previous line, the two are joined.

\begin{small}
\begin{verbatim}
    #pile
    GetUp()
    if Saturday then
        CookBreakfast()
        Eat << Toast << Tomato << Bacon << Eggs
    HaveShower()
    DrinkCoffee()
\end{verbatim}
\end{small}

The \ttin{CookBreakfast} and \ttin{Eat...} lines form a pile,
which can be rewritten as a semicolon separated sequence:

\begin{small}
\begin{verbatim}
    #pile
    GetUp()
    if Saturday then {
        CookBreakfast();
        Eat << Toast << Tomato << Bacon << Eggs
    }
    HaveShower()
    DrinkCoffee()
\end{verbatim}
\end{small}

And finally the entire program is treated as a pile:

\begin{small}
\begin{verbatim}
{
    GetUp();
    if Saturday then {
        CookBreakfast();
        Eat << Toast << Tomato << Bacon << Eggs
    }
    HaveShower();
    DrinkCoffee()
}
\end{verbatim}
\end{small}

Readers wishing to experiment interactively with our examples
(by using ``\asharpcmd{ -gloop}'') should note that piling is
the default in interactive use.  The examples generally should
still run correctly if the illustrated layouts are used --- a
few may require the addition of braces (\ttin{\{ \}}).  See
\chapref{asugUsingInteractive} for further details.
