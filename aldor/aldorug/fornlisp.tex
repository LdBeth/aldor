% *********************************************************************
\head{section}{Running generated Lisp code}{asugForeignLisp}
% *********************************************************************

This section describes how to load \asharp{} code into a Common Lisp
system.  After the general treatment (which should work for any
Common Lisp), there are some special instructions which apply only
to loading the code into an {\sc Axiom} system.

This section documents an area which is undergoing active development
and we expect to provide a set of lisp utilities to make the interface
smoother\footnote{We don't really now recommend it for the faint of
heart!}.

To use the compiler to generate Lisp for the file ``{\tt mandel.as}'',
do the following:
\index{compiler options!Fl@\protect{\tt -Fl} (generate Lisp code)}
\begin{verbatim}
    cd test
    aldor -F l mandel.as
\end{verbatim}
This generates the file ``{\tt mandel.l}''.

To load the Lisp code, first issue {\tt lisp} to start the
Lisp system.  
The generated code references various macros, defined in the file
``{\tt \$ALDORROOT/include/foam\_l.l}''.
So it is necessary to load this file before using the file ``{\tt mandel.l}''.

These files may either be loaded for interpretation,
\begin{verbatim}
    (load "/usr/aldor/base/rs3.2/include/foam_l.l")
    (load "mandel.l")
\end{verbatim}
or they may be compiled beforehand, \eg{}:
\begin{verbatim}
    (load "/usr/aldor/base/rs3.2/include/foam_l.l")
    (compile-file "mandel.l")
    (load "mandel.o")
\end{verbatim}

Once the ``{\tt foam\_l.l}'' file has been loaded, it is no longer necessary to
leave lisp to compile \asharp{} files. 
The above steps for compiling and loading ``{\tt mandel.as}'' could have all
been accomplished with the one command
\begin{verbatim}
    (compile-asxln-file "mandel.as")
\end{verbatim}

Next you must compile and load the libraries used by your file
(for {\tt mandel.as,} the libraries are {\tt basic.as} and
{\tt complex.as}).
\begin{verbatim}
    (compile-asxln-file "../aldor/basic.as")
    (compile-asxln-file "../aldor/complex.as")
    (compile-asxln-file "mandel.as")
\end{verbatim}

The {\tt compile-asxln-file} function compiles {\tt .as} files to
Lisp code, then compiles and loads the Lisp code.

You next need to instantiate the file you want to run.
The name of the instantiation program is obtained by concatenating
\verb+G_+ to the name of the file.
For example, the instantiation program for {\tt mandel.as}
is called \verb+G_mandel+.
\begin{verbatim}
    (setq c (|CCall| |G_mandel|))
\end{verbatim}
This creates a Lisp variable {\tt c} holding the closure for the file
getter program.

To call a function from the package, you have two choices: directly call
the function using its generated Lisp name, or call the getter
function to get a closure and call the closure.
To call {\tt drawMand} the first way:
\begin{verbatim}
    (|Cmandel-1_mandel_drawMand|
        -2.0 2.0 25 -2.0 2.0 25 (|ClosEnv| c))
\end{verbatim}

The name of the progam can be determined by looking at the
generated Lisp file.
The last argument is the environment for the file.

The other method is to get the closure by calling the file getter with the
apropriate hash codes:
\begin{verbatim}
    (setq mand (|CCall| c 259747446 526180818))
    (|CCall| mand -2.0 2.0 25 -2.0 2.0 25)
\end{verbatim}

The hash codes can be determined by looking at the code for the
file getter (the last function in the Lisp file), and seeing what
codes it tests to return the function you wish to call.

These two methods only work for calling top-level functions in a file.

If you are in an {\sc Axiom} workspace, then there are some additional 
utilities to make using the file getter more palatable.

First you must translate and load the file `{\tt hashax.boot}'' from the
compiler source directory.  This file provides functions which generate the
numbers which are arguments to the getter (such as 259747446 526180818 above).

The fist number is a hash code for the name of the export. 
In this example it can be obtained by
\begin{verbatim}
    (setq scode (|hashString| "drawMand"))
\end{verbatim}

The second number is a hash code for the type of the export.
If there is only one export with the given name, then the number {\tt -1}
may be used to match any code.  Otherwise the hash code can be obtained
from an S-expression from of the type:
\begin{verbatim}
   (setq tcode (|hashType| '(|Mapping| |MachineInteger|
                                  |DoubleFloat| |DoubleFloat| |MachineInteger|
                                  |DoubleFloat| |DoubleFloat| |MachineInteger|)))
\end{verbatim}

In a forthcoming release of the \asharp{} compiler, we expect to
provide a much better interface that will allow compiling, loading
and running code directly from the AXIOM interpreter, and all the
low level manipulations above will be unnecessary.


