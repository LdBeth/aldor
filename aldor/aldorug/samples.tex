% *********************************************************************
\head{chapter}{Sample programs}{asugSample}
% *********************************************************************
\index{samples of \asharp{} code}
In this chapter we show several examples of \asharp{} programs.
The first few give a brief introduction to the language, followed by some
examples using more advanced features of the language. The final
few examples show how \asharp{} can variously emulate or interact with
other languages.

This chapter supplements the material in \chapref{asugLangSimple} and
in part II, the language description.
For reference, the examples are as follows:

\begin{tabular}{@{}llr@{}}
{\bf Hello} & A few simple constructs
            & p. \pageref{helloWorldSample} \\
{\bf Fact}  & Function definition and calling, and simple 
            & p. \pageref{factorialSample} \\
            & iteration & \\
{\bf Greet} & More function definitions, user input and 
            & p. \pageref{greetSample} \\
            & importing & \\
{\bf Cycle} & Functions and multi-valued returns
            & p. \pageref{cycleSample} \\
{\bf Gener} & Generators and iteration
            & p. \pageref{generSample} \\
{\bf Symbol} & Defining domains
            & p. \pageref{symbolSample} \\
{\bf Stack} & Defining parameterised domains
            & p. \pageref{stackSample} \\
{\bf Tree} & Defining recursive data types
           & p. \pageref{recurseSample} \\
{\bf Swap} & Dependent types and higher order functions
           & p. \pageref{swapSample} \\
{\bf Object} & Object-oriented programming
             & p. \pageref{objectSample} \\
%{\bf Table} & Hash-tables from the \asharp{} library
%            & p. \pageref{libTableSample} \\
%{\bf Float} & Arbitrary precision floating point, formatted output
%            & p. \pageref{floatSample} \\
{\bf Mandel} & Machine floating point arithmetic
             & p. \pageref{mandelSample} \\
%{\bf Gr\"obner} & Use of the Gr\"obner basis package
%            & p. \pageref{gbSample} \\
{\bf Imod} & Integers modulo some number
           & p. \pageref{imodnSample} \\
{\bf Extend} & Extensions
             & p. \pageref{extendSample} \\
{\bf TextIo} & Input and output
             & p. \pageref{ioSample} \\
{\bf Quanc8} & Fortran-style programming 
             & p. \pageref{quanc8Sample} \\
%{\bf Xmandel} & \x11{} interoperability
%            & p. \pageref{x11Sample} \\
%{\bf Hilbert} & \axiom{} interoperability
%            & p. \pageref{AxiomSample} \\
\end{tabular}
\newpage
% *********************************************************************
\head{section}{Hello}{helloWorldSample}
% *********************************************************************

This program prints out a familiar greeting, and then exits. Line 1
allows the use of the base \asharp{} library in this
program. Line 3 prints the greeting. The program shows how a simple
program is written, and the syntax for printing objects.

\program{hello}

\begin{description}
\lineNote{1} Include the file \ttin{aldor.as}. This allows all the domains
and categories in the standard library to be used.
%(see chapters \ref{asugBaselibCats} and \ref{asugBaselibDoms}).
\lineNote{2} Include the file \ttin{aldorio.as}. This allows to import
automatically the necessary types to do some input/output.
\lineNote{4} Print the message. \ttin{stdout} (available because \fname{aldorio.as}
has been included) is an identifier referring
to the console output stream. ``\verb+<<+'' is an infix operator that
prints an object to a given stream, and returns the stream as a
value. This allows a cascade of ``\verb+<<+'' calls.
\end{description}
\index{stdout}

\newpage
% *********************************************************************
\head{section}{Factorial}{factorialSample}
% *********************************************************************

The next example\label{factpage} shows how to define and call functions in \asharp,
and a simple form of iteration.

\program{myfact}

This program defines two functions for calculating the factorial of an
integer. The first shows a simple recursive version, the second is
an iterative version. 
\index{recursion}
\begin{description}
\lineNote{3} Activates the ``piling'' syntax mode (see \secref{asugFLangPiles} for details).
\lineNote{5} Defines a function called {\tt rfact}, which takes an {\tt
Integer} and returns the result as an {\tt Integer}
\lineNote{7} Defines a function {\tt ifact} with the same signature as {\tt
rfact}. 
\lineNote{8} Assigns the \verb"Integer" \verb"1"
to {\tt i}. Note that no declaration is
needed if the compiler can infer the type of {\tt i}.
In this example, \verb"1"
must have been exported from the domain {\tt Integer}. If, for example,
{\tt MachineInteger} was also imported, a declaration would have been
necessary.
\lineNote{9} The {\tt repeat} keyword is used to indicate a loop, and the
{\tt while} keyword gives the test for the loop, in this case the loop
will exit once {\tt n} is less than or equal to one.
\lineNote{10,11} The loop body --- multiply {\tt i} and decrement {\tt n} by
one. Note that it is legal to assign to a function's parameters.
\lineNote{12} The last statement in the function body gives the value to
be returned.
\end{description}
\keywordIndex{repeat}
\index{loop}
\keywordIndex{while}

\newpage
% *********************************************************************
\head{section}{Greetings}{greetSample}
% *********************************************************************

Most operations in \asharp{} are defined in and exported by {\em
domains}. In order to import from a domain,
the {\tt import} statement is used. Imports implicitly happen for the
types of parameters of a function, and its return value. The example
program below shows the use of the {\tt import} statement, and also
how to read user input.
\index{domain}
\keywordIndex{import}

\program{greet}

This program declares functions to read a name from the console, and
then print a personalised note for that name. 
\begin{description}
\lineNote{1} include the standard library
\lineNote{4 to 8} import operations from the given domains, allowing us
to use operations. Note that the {\tt import from} statement is only
needed when operations have not been implicitly imported. In this example,
\fname{aldorio.as} was not included and so imports for input/output are
necessary.
\lineNote{11} Defines a function, {\tt readName}. Note that the {\tt ()}
indicates that the function takes no parameters.
\lineNote{20} Defines a function, {\tt greet}. The return type of the
function, {\tt ()} indicates that the function does not return a
value.
\lineNote{25} Call the function.
\end{description}

The program produces the following output (user input is in {\it italic}):
\begin{flushleft}\small\leftskip=\baseLeftSkip
\tt
What is your name?	\\
{\it Aldor}	\\
Hello Aldor, and goodbye...
\end{flushleft}

% *********************************************************************
\newpage
\head{section}{Cycle}{cycleSample}
% *********************************************************************

This example demonstrates the manipulation of functions as first-class
values, creating new closures over the course of the computation and
multiple valued returns.
\index{multiple values!returning}

\program{cycle}

\begin{description}
\lineNote{9--11} Define some macros as shorthand for some common types
used in the program. \ttin{MapIII} is a map type for functions that
take three integers and return three integers.
\lineNote{13} The identity function for {\tt MapIII}. Note that
\ttin{+->} produces a closure (otherwise known as a ``lambda expression'',
``anonymous function'' or just ``function'').
\lineNote{16} Define a function \ttin{*} (which is an infix operator)
to be a function that given two functions (of type {\tt MapIII})
returns a third function (in this case the composition of the two
functions).
\lineNote{19--24} Define a function that returns 
$\underbrace{f f \cdots f}_{n \; {\rm times}} (a,b,c)$. \\
The new function is computed by repeated squaring.
\lineNote{27--38} A test routine.
\lineNote{28, 30} One can define constants inside functions. These have
scope local to the function.
\end{description}
\index{closure}
\index{function expression}

The program produces the following output:
\verbatiminput{samples/cycle.out}

% *********************************************************************
\newpage
\head{section}{Generators}{generSample}
% *********************************************************************

Iteration in \asharp{} is mainly achieved through the use of
generators. These are objects representing the state of an iteration,
and may be passed around as first-class values. There are two ways of
creating generators in \asharp{}: The {\tt generate} keyword, and the
collect form, created by iterators.
\index{generators}

\program{gens}

\begin{description}
\lineNote{5} Define a helper function which computes an approximated
exponential for a \verb+DoubleFloat+ value.
\lineNote{17} Define a function which creates a generator of 
\verb"DoubleFloat"s.
\lineNote{17} The \verb+generate+ keyword introduces the generator body. This is
evaluated when a value from the generator is needed, up to the first
yield statement, the value of which is returned as the next value for
the generator. 
\lineNote{18--23} The body of the generator. In this case we yield the
value of $e^{-x^2}$ for increasing values of $x$.
\lineNote{25} Define a function to calculate the running mean of a
sequence. 
\lineNote{28--34} The generator body. The value is calculated by
maintaining a sum of values so far, and dividing by the number of
values. 
\lineNote{37} Define a helper function which returns a generator of
\verb+step+s between $a$ and $b$ using $n$ as the size of the interval.
\lineNote{49} create a generator and iterate over it. In this case the
generator is the running average of 0, 0.1, \ldots, 1.0.
\lineNote{52} where more than one iterator  (formed by {\tt for}
and {\tt while}) precedes the repeat, iteration is in parallel and
terminates when one of the iterators reaches its end condition. In
this case we want the first few values of a generator, so the parallel
iteration ensures that the loop will complete at or before 10 iterations.
\end{description}
\keywordIndex{yield}
There is a further example on the use of generators on page \pageref{recurseSample}.


% *********************************************************************
\newpage
\head{section}{Symbol}{symbolSample}
% *********************************************************************

Most programming in \asharp{} is done by defining domains and
packages. Here we give a small example of a domain. A package is simply
a domain which does not export any operations involving values of
type \%. 

Typically, writing a domain is done in four stages:
\begin{enumerate}
\item decide what operations a domain will provide, and from which
\linebreak categories it should inherit,
\item decide how to represent the domain,
\item implement the operations,
\item test the domain.
\end{enumerate}
In this example we show how to define a domain. The example is a
symbol datatype which ensures that only a single instance of a given
symbol is created.

\program{symbol}

\begin{description}
\lineNote{4} Define the category \verb+BasicType+ which is the union of
PrimitiveType and OutputType.
\lineNote{6} Define symbol to be a constant with the given type --- in
this case \verb"Symbol" is a \verb"Domain"
which implements {\tt BasicType} and
provides 3 additional operations.
\lineNote{6--9} Signatures for operations on the domain. {\tt \%} in the
type refers to ``this \verb"Domain"''.
\lineNote{10} The {\tt add} expression creates a domain from a sequence of
definitions. 
\lineNote{11} Define how \verb"Symbols",
our new type are represented. In this case,
we just use a string. If the signature required that additional
information was stored on the symbol we might want to use a different
representation.
\lineNote{13} Allow operations from the representation domain, \ie{}
{\tt String} to be used.
\lineNote{15} {\tt \%} can be used inside the add body to refer to the
current domain.
\lineNote{15} define a local variable for storing the symbol table.
\lineNote{17--34} Define functions exported by the domain.
\lineNote{19 and 26} The operation {\tt rep} allows a value of type
\% to be treated as if it were of type {\tt Rep}. {\tt per} is the
inverse operation, \ie{} it converts an object of type {\tt Rep} into 
type \%. These two operations are currently macros\footnote{ {\tt per}
can be thought of as `rep backwards', or as `{\em per}cent'.}. See
\secref{asugLangSimpleIV}.
\end{description}
%\index{"%@"{\tt \%}}
\keywordIndex{add}
\index{macros}
\keywordIndex{pretend}

% *********************************************************************
\newpage
\head{section}{Stack}{stackSample}
% *********************************************************************

It is possible to define a function whose return type is a domain. In
this case, the result is called a {\em parameterised domain}.  The
example is a simple stack with a few operations for creating new
stacks, as well as pushing and popping values from an existing stack.
\index{parameterised type}

\program{stack}

The chosen representation type is a list over the same domain as the
stack. This allows us to implement the operations with minimum
complications. A better representation might be a linked list of
arrays, but this would clutter the example more than necessary.

\begin{description}
\lineNote{8} Define {\tt Stack} to be a function that takes a
BasicType (\ie{} most of the domains in \asharp{}),
% --- see \chapref{asugBaselibCats})
and returns an object which satisfies
BasicType, and additionally provides some stack operations. The
interface is specified by the {\tt with} construct.
\lineNote{9} Declare an operation `empty?' which takes a value from the
current domain as an argument, and returns a Boolean value.
The line starting `++' is a description comment, and is saved along
with the declaration of {\tt empty?} in the intermediate file.
\lineNote{16} The `export from' statement indicates that all operations
exported by {\tt S} should be imported when Stream S is imported. 
\lineNote{23} Define the representation of the Stack. This form of a
define statement (without a declaration) implies that the type of Rep
is exactly that of the type of the right hand side of the expression,
\ie{} {\tt Record(contents: List S)}.
\lineNote{24} Allow operations from Rep to be used. We do not need to say
{\tt import from List S}, as Record exports its arguments.
\lineNote{27} Define a function which returns the internal list of values
maintained by the stack. 
\lineNote{30--43} Define the operations required explicitly by the category.
\lineNote{46--47} Define the operations needed to satisfy inherited operations.
\end{description}
\keywordIndex{with}

\newpage
Once the domain is defined, it may be tested. \asharp{}'s interactive
loop, \ttin{aldor -G loop} is useful here.
\index{compiler options!\protect{\tt -Gloop}}
\begin{flushleft}\small\leftskip=\baseLeftSkip
\verb+%+ {\em aldor -G loop}			\\
\verb+%1 >>+ {\em \#include "stack.as"}	\\
\verb+stack is:<stack>+\\
\verb+Next is: 6+\\
\verb+Next is: 5+\\
\verb+Next is: 4+\\
\verb+Next is: 3+\\
\verb+Next is: 2+\\
\verb+Next is: 1+\\
\verb+%2 >>+ {\em import from Stack MachineInteger}	\\
\verb+%3 >>+ {\em s: Stack MachineInteger := empty()}	\\
\verb+<stack> @ Stack(MachineInteger)+			\\
\verb+%4 >>+ {\em push!(12, s)}		\\
\verb+<stack> @ Stack(MachineInteger)+	\\
\verb+%5 >>+ {\em top s}	\\
12 @ MachineInteger		\\
\verb+%5 >>+ {\em pop! s}	\\
12 @ MachineInteger
\end{flushleft}

We should probably test it a little further (\eg{}~boundary conditions),
but this gives the general idea.

% *********************************************************************
\newpage
\head{section}{Recursive structures}{recurseSample}
% *********************************************************************

This program shows a recursively defined data type:
\index{recursion}
the type of binary trees parameterised with respect to the type
of data placed on the interior nodes.  This tree type provides
several generators which allow the trees to be traversed in different ways.

\program{tree}

When compiled and run, this program gives the following output:
\begin{small}
\verbatiminput{samples/tree.out}
\end{small}

% *********************************************************************
%\newpage
%\head{section}{swap}{swapSample}
% *********************************************************************

%The following is an example of a package which is parameterised over a
%domain.

%\program{mysort}

%The next example demonstrates how parameterised type constructors may
%themselves be passed as arguments to other functions.

% *********************************************************************
\newpage
\head{section}{Swap}{swapSample}
% *********************************************************************

Higher order functions which construct types are first-class values.  
This example shows how to swap structure layers in a data type
by using higher order functions as parameters to a generic program.

\program{swap}

\begin{description}
\lineNote{6} Define a macro \ttin{Ag} as a shorthand for the type
which takes a Type as an argument, and returns a second type which
is a {\tt BoundedFiniteLinearStructureType} over it.
\lineNote{11--13} Define the \ttin{swap} function.  This curried definition
exchanges the \ttin{X} and \ttin{Y} layers in a structure.
This function is written generically to use
\begin{itemize}
\item \ttin{generator} from \verb"X Y S" for the outer iteration,
\item \ttin{generator} from \verb"Y S" for the inner iteration,
\item \ttin{bracket} from \verb"X S" as the inner constructor,
\item \ttin{bracket} from \verb"Y X S" as the outer constructor.
\end{itemize}

\lineNote{25} Call \ttin{swap} to exchange the \verb"Array"
and \verb"List" layers.
\end{description}

When executed via ``{\tt \asharpcmd{} -G run swap.as},'' the following
output is produced.
{\small
\verbatiminput{samples/swap.out}
}

% *********************************************************************
\newpage
\head{section}{Objects}{objectSample}
% *********************************************************************

In \asharp{}, values are {\sl not\/} self-identifying --- there is no way
of retrieving a given value's type from the value itself.  
\index{self-identifying values}

We can implement this functionality in the \ttin{Object}
datatype, which holds both a value and its type.

The following example shows the implementation of Object and a use of it.

\program{objectb}

\begin{description}
\lineNote{9}
   Define the \verb+Object+ datatype. As we can see there is a constructor
   called \verb+object+ to bundle a value and its type and a ``dismantler''
   called \verb+avail+ to get the value and the type from a previously
   constructed object. This domain is parameterized by a category called
   \verb+C+. All types that we can bundle in an \verb+Object(C)+ will
   satisfy the category \verb+C+.
\lineNote{23}
   Define a function, \ttin{bobfun}, to take object arguments.
   The arguments are objects whose underlying type satisfies the
   category \verb"OutputType".
\lineNote{24}
   Use the \ttin{avail} operation to split an object into its type
   and value components, then call the local function \ttin{f}
   on the dependent type/value pair.
\lineNote{26}
   Print the object value.  The \ttin{<<} operation is available,
   because each object's type satisfies \verb"OutputType".
\lineNote{30}
   Form a list of \verb"OutputType" objects.  Each is formed with the
   \ttin{object} function from \verb"Object(OutputType)".
\lineNote{35}
   Call \ttin{bobfun} on each of the objects in the list.
\end{description}

When run with ``{\tt \asharpcmd{} -G run objectb.as}'' this program
produces the following output:
{\small
\verbatiminput{samples/objectb.out}
}

\vspace{1cm}
The richer the category argument to \verb"Object", the more interesting
operations may be performed on the object values.  A second example
of using \verb"Object" is shown below.  In this case each object
value belongs to some ring, and this fact is used in the arithmetic
calculation.
%%FIXME: Need to make this work!
\program{objectr}
\begin{description}
\lineNote{24}
   Define a function, \ttin{robfun,} to take object arguments.
   This time the arguments are objects whose type slots satisfy the
   category \verb"IntegerType".
   Again \ttin{avail} is used to split an object into its component parts
   (type and value).
\lineNote{28--30}
   Perform various arithmetic operations on the value.
   All of \ttin{+} \ttin{-} \ttin{*} ``\verb"^"'' and \ttin{1} are provided
   by the particular object.
\lineNote{33--36}
   Print the results of the arithmetic.  This is possible because each
   \verb"IntegerType" provides a \ttin{<<} operation.
\lineNote{39}
   The {\tt has} operator tests whether a given domain
   satisfies a particular category. This test is made at run-time.
\lineNote{40--44} 
   Inside an {\tt if} statement, if it can be deduced that
   an imported domain satisfies an additional category
   (using the information in the evaluation of the {\tt if} expression),
   then the additional operations are made available within the \ttin{then}
   branch of the {\tt if} statement. 
   In this case, \ttin{<} and \ttin{>} are available because \verb"T" 
   is seen to also satisfy \verb"TotallyOrderedType".
\end{description}

The output produced when running this program with the command
``{\tt \asharpcmd{} -G run objectb.as}'' is shown below.
{\small
\verbatiminput{samples/objectr.out}
}
%% % *********************************************************************
%% \newpage
%% \head{section}{\asharp{} libraries}{libNotesSample}
%% % *********************************************************************
%% 
%% Here we give some examples of code which uses a number of domains from
%% the libraries provided with \asharp{}. Four libraries are provided in
%% the \asharp{} distribution: 
%% \begin{itemize}
%% \item The basic library (\ttin{libaldor})
%% \item The \axiom{} interoperability library (\ttin{libaxiom})
%% \item A library of useful demonstration code (\ttin{libaldordem})
%% \item The \x11{} interface library (\ttin{libaldorX11})
%% \end{itemize}
%% The first two libraries are documented elsewhere (see
%% \chapref{asugBaselibCats} and \chapref{asugBaselibDoms} for
%% \ttin{libaldor} and \chapref{asugUsingAxiom} for \ttin{libaxiom}).
%% 
%% Source code for \ttin{libaldordem} is provided in the distribution.
%% It includes the following types:
%% \io{Dir\-ect\-Prod\-uct},
%% \io{Groeb\-ner\-Pack\-age},
%% \io{Index\-ed\-Bits},
%% \io{Multi\-Dict\-ionary},
%% \io{List\-Multi\-Dict\-ionary},
%% \io{Matrix\-Op\-Dom},
%% \io{Matrix},
%% \io{Non\-Neg\-ative\-Int\-eger},
%% \io{Poly\-nomial\-Cat\-egory},
%% \io{Poly\-nomial},
%% \io{Int\-eger\-Primes\-Package},
%% \io{Rand\-om\-Number\-Source},
%% \io{Small\-Prime\-Field} and
%% \io{Vect\-or}. To use this library, use \verb+#include"aldordem"+.
%% 
%% The \ttin{libaldorX11} library contains declarations for
%% the data structures and functions found in the Xlib library. 
%% This library provides an interface to the
%% low level \x11{} functions, on top of which can be built various
%% toolkits.
%% 
\newpage
% *********************************************************************
%\head{section}{Tables}{libTableSample}
% *********************************************************************

%The next example shows a possible implementation of a \verb"Table" datatype
%using a hash table.

%\program{table}

% *********************************************************************
%%\newpage
%%\head{section}{Floating point}{floatSample}
%%% *********************************************************************
%%
%%The next example demonstrates the use of arbitrary precision floating
%%point numbers
%%
%%\program{float1}
%%
%%When executed via ``{\tt \asharpcmd{} -G run float1.as},'' the following
%%output is produced.
%%{\small
%%\verbatiminput{samples/float1.out}
%%}

% *********************************************************************
\newpage
\head{section}{Mandel}{mandelSample}
% *********************************************************************

The next example shows the use of machine-level floating point in \asharp.  
This program would be a bit simpler if we first implemented a Complex
domain.

\program{mandel}

Machine level operations are done inline when the optimiser is used
while compiling (use the options ``{\tt -Q3 -Qinline-limit=10}'').
\index{optimisation}
This has the result that the generated code speed is
comparable with that of the equivalent code in languages such as C.

%%% *********************************************************************
%%\newpage
%%\head{section}{Gr\"obner bases }{gbSample}
%%% *********************************************************************
%%
%%Some symbolic algebra packages are supplied in the \asharp{}
%%demonstration library. For example, \io{GroebnerPackage} provides a
%%mechanism for computing the Gr\"obner basis of a list of polynomials.
%%(These may be used to solve polynomial systems of equations,
%%among other things.)
%%Compile this file with the \verb"-laldordem" option to link with the appropriate
%%library.
%%
%%\program{gbtest1}
%%\newpage
%%When executed via ``{\tt \asharpcmd{} -G run gbtest1.as -L aldordem}'' the program
%%produces the following output: 
%%{\small
%%\verbatiminput{samples/gbtest1.out}
%%}
%%
% *********************************************************************
\newpage
\head{section}{Integers mod n}{imodnSample}
% *********************************************************************

This example shows how add-inheritance can be used in the
implementation of integers modulo a particular number.  We first
define a category and a generic member of the category, called
{\tt ModularIntegerNumberRep}.  We then have two specific instances of the
category which inherit most of their operations from the generic
domain.  Note that in the definition of {\tt MachineIntegerMod} we
over-ride the generic multiplication with something more efficient.
Also note that the definition of {\tt Rep} is required in both the specific
implementations so that the {\tt rep} and {\tt per} macros will work,
however it is essential that it is compatible with the {\tt Rep} in
the generic case.

\program{imod}

% *********************************************************************
\newpage
\head{section}{Extensions}{extendSample}
% *********************************************************************


\asharp{} allows the library types to be extended with new
operations. For example, one may wish to add a {\tt DifferentialRing}
category to the language. The extension mechanism allows existing
domains, such as {\tt Integer} and {Polynomial} to include these new
categories. The extension mechanism operates via the \ttin{extend} 
keyword.

The following example allows us to sort lists of symbols.  The {\tt List(S)}
domain exports a {\tt sort} operator if {\tt S} belongs to the category
{\tt TotallyOrderedType}.  Although {\tt Symbol} does not belong to this category
{\tt String} does, and we can use this fact to implement the necessary
exports in a fairly straightforward manner.
\keywordIndex{extend}

\program{extend}

\begin{description}
\lineNote{6--24} Extend Symbol. In order to satisfy {\tt TotallyOrderedType} we need
to provide six operations which in this case we implement in terms of
strings.
\lineNote{26--32} Test the extended domain constructor.
\end{description}

% *********************************************************************
\newpage
\head{section}{Text input}{ioSample}
% *********************************************************************

%The next example uses the \asharp{} datatypes {\tt TextReader} and
%{\tt TextWriter} to provide a number of useful text processing
%operations. 
%% gives a bad linebreak.
In the next example, the \asharp{} {\tt TextReader} and
{\tt TextWriter} {\hyphenation{datatypes data-types} datatypes}
are used to provide a number of useful text processing operations. 

\program{textin}

\begin{description}
\lineNote{7} Define a predicate for testing whether a given character
is a vowel
\lineNote{15} Define a function to print the non-vowel characters in a
file. 
\lineNote{15} {\tt TextReader} provides a generator on a reader which
returns the sequence of characters in the reader. {\tt TextWriter}
provides primitives to write on a stream.
\lineNote{19} {\tt write!} puts a single character onto a stream.
The stream is passed as a parameter with the name {\tt tw}. A
possible value for {\tt tw} would be {\tt stdout} which is a value
of type {\tt TextWriter} and thus an output stream. It is attached
to the default output device of the process.
\lineNote{25} Define a function to print a text message.
\lineNote{33} Define a conversion function from a list of {\tt Character}s
to a String.
\lineNote{44} {\tt lines} creates a generator which returns the
contents of an input stream (passed as a {\tt TextReader}) as a
sequence of lines terminated by {\tt newline}.
\lineNote{61} Open a file for input. The value {\tt fileRead} is exported
by {\tt File} and is the mode chosen to open the file.
\lineNote{62} {\tt fileWrite} is also exported by {\tt File}.
\lineNote{63} The file {\tt f1} is \verb+coerce+d to an input stream of
type {\tt TextReader}.
\lineNote{64} The file {\tt f2} is \verb+coerce+d to an output stream of
type {\tt TextWriter}.
\end{description}


% *********************************************************************
\newpage
\head{section}{Quanc8}{quanc8Sample}
% *********************************************************************

The next example gives a Fortran-style program for numeric
integration. The program demonstrates how an algorithm described in
the pre-structured programming era may be transcribed without
introducing errors by reworking its logic. The program was transcribed
from the textbook described in the first comment, and produced correct
values on its first run. Of course, if you have access to a callable
library containing the routines it should be possible to import the
operations directly into \asharp{}.
\index{Fortran}

The \ttin{goto} construct in \asharp{} takes the name of a label, and
transfers control to that label. Labels are introduced by the
\ttin{@} symbol.
\keywordIndex{goto}
\keywordIndex{label}

\program{quanc8}


% *********************************************************************
%\newpage
%\head{section}{\x11{}}{x11Sample}
% *********************************************************************
%
%\asharp{} provides an interface to \x11{}. The following example
%draws a Mandelbrot fractal set in a window and allows it to be resized,
%recoloured, etc.
%
%\program{xmandel}
%
%%% *********************************************************************
%%\newpage
%%\head{section}{\axiom{} library }{AxiomSample}
%%% *********************************************************************
%%
%%\asharp{} programs can run within the \axiom{} environment, and can use all the
%%categories, domains and packages supplied by \axiom{}. The following
%%example computes the Hilbert polynomial for a set of monomials.
%%
%%\program{hilbert0}
%%
%%\pagebreak
%%Note that \axiom{} domains and categories are used freely within the
%%program, for example, {\tt OrderedSet}, {\tt Monomial} and {\tt List}. 
%%
%%As well as providing all the \axiom{} domains, {\tt libaxiom} also
%%extends a few so that features new in \asharp{} are (apparently)
%%provided in the old \axiom{} domain. As an example, in the function {\tt
%%variables}, the \verb"List" \ttin{M}{} is iterated using a generator, the
%%definition of which came from the \asharp{} extension of \verb"List".
%%
%Data structures
%	table.as
%Mathematical
%	Numeric
%		-- arb. precision floating point
%		-- dfloat arithmetic
%	Algebraic
%		groebner basis
%		imodn
%			
%	Note: Testing using the interpreter
%Tuning
%
%Fortran code
%	quanc8.as
%
%C Code
% Array DFlo, and sort it
%Not covered:
%	Implicit operations
%	categories
%	Enumerations 
%	Unions
%
%
