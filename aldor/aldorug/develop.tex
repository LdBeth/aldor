%
% *********************************************************************
\head{chapter}{For Internal Compiler Developers}{asugDeveloper}
% *********************************************************************

This chapter provides information for those who wish to maintain
or develop the \asharp{} compiler.

% *********************************************************************
\head{section}{The Internal Developer Environment}{asugDeveloperEnv}
% *********************************************************************

In addition to the environment described in \spadref{asugInstallDir},
internal developers need to set the {\tt MACHINE} environment variable.
See the table in \spadref{asugInstallPlatform} for allowable values for this
variable.
Any other value for {\tt MACHINE} will cause a compiler called
{\tt cc} to be used for compiling the compiler source code.
Normally, {\tt MACHINE} and the last component of the
{\tt \$ALDORROOT} directory name should be the same for a developer.
A developer should have a local copy of the entire main directory and
\index{MACHINE@\protect{\tt MACHINE} environment variable}
set the {\tt \$ALDORROOT} variable appropriately.
\index{environment variables!MACHINE@\protect{\tt MACHINE}}

The main directory of the compiler has a subdirectory called
{\tt version.}
\index{directories!version@\protect{\tt version}}
In most cases, this will have only one subdirectory.
If there is more than one, use the newest one in what follows.
For example, the subdirectory {\tt \asharpver{}} corresponds to an
early $\beta$-test version of the compiler.
The subdirectories of that directory have names like {\tt src,
tools, test, doc} and so on.
We will often refer to those directories simply by their
relative names.
Thus, unless otherwise specified, the directory {\tt src/} will
\index{directories!src@\protect{\tt src}}
mean, for example, {\tt \$ALDORROOT/../version/\asharpver{}/src} and {\tt
tools/} will mean {\tt \$ALDORROOT/../version/\asharpver{}/tools.}

% *********************************************************************
\head{section}{Making a New Compiler}{asugDeveloperMake}
% *********************************************************************

The following general instructions for making the compiler are
for Unix systems.
\index{compiler!making the}
Some specific tools for making the compiler in other environments
can be found in subdirectories of the {\tt tools/} (see below) directory.
See, for example, the {\tt cms} and {\tt os2} subdirectories.
\index{directories!tests@\protect{\tt tools/cms}}
\index{directories!tests@\protect{\tt tools/os2}}

To make the compiler for version {\tt \asharpver{}} on an IBM RS/6000
running AIX 3.2, issue the following:
\begin{enumerate}
\item Issue {\tt echo \$ALDORROOT} and {\tt echo \$PATH} to ensure
that they are set correctly.
\item Make sure you have the proper file permissions to update
the subdirectories of {\tt \$ALDORROOT}: use {\tt su} if
necessary.
\item Issue {\tt cd \$ALDORROOT/../version/\asharpver{}}
\item Issue {\tt sh makeon rs3.2 all}
\end{enumerate}

The {\tt rs3.2} is the value of the {\tt MACHINE} environment variable.

The above process compiles the source for the tools, compiles the
source for the compiler, makes the \asharp{} libraries and runs
the regression tests. Inspect the console output or the log {\tt
\$ALDORROOT/makeon.log} to make sure the build and tests ran
successfully.
\index{files!makeon.log@\protect{\tt makeon.log}}

% *********************************************************************
\head{section}{Adding Tests}{asugDeveloperTests}
% *********************************************************************

The regression tests are contained the {\tt tests/} subdirectory.
\index{compiler!tests}
We do not recommend you modify this directory from the
distribution.
\index{tests}
You can, however, create a local test directory and place tests
and their output there.
\index{directories!tests@\protect{\tt tests}}
We now assume you have created a directory called {\tt test/}
(relative, perhaps, to your home directory).
Within {\tt test/} you must place a subdirectory called {\tt
test.out/.}
That is, do the following:
\begin{enumerate}
\item Use {\tt cd} to move to what will be the parent directory of
{\tt test/}.
\item Issue {\tt mkdir test}
\item Issue {\tt mkdir test/test.out}
\end{enumerate}

To add a new test (we will call it {\tt newtest.as}), do the
following:
\begin{enumerate}
\item Change to the {\tt tests/} subdirectory and
place a copy of {\tt newtest.as} there.
\item Issue {\tt testas -install newtest.as}
\item Issue {\tt cd test.out}
\item Issue {\tt ls newtest.*} and inspect the files
{\tt newtest.E,} {\tt newtest.R,}
and {\tt newtest.out,} if they exist, to ensure that
the test is doing what you want.
\end{enumerate}

Depending on what test options you have given via ``{\tt -->}''
in {\tt newtest.as} other test output files may exist.
Some may not be readable (for example, a binary {\tt .ao} file).
See \chapref{asugOptions} for a listing of the type of files the
compiler can produce.

A test file has special ``{\tt -->}'' comments at its top.
These comments cause {\tt testas -install} to create various new standard
result files.
They also cause {\tt testas} to compare the newly generated files
with the existing standard ones.

\def\testopt#1#2{\par\hangafter=1\hangindent=.25in{\tt --$>$
#1}\index{compiler!tests!#1@\protect{\tt\  #1} option}\newline{#2}}

\testopt{testerrs}{This adds the option {\tt -DTestErrorsToo} to
\index{compiler options!D@\protect{\tt -D} (add global assertion)}
the compilation and captures error output in a file with
extension {\tt .E}.}

\testopt{testrun}{This adds the option {\tt -DTestRun} to
the compilation and captures output in a file with
extension {\tt .out}.}

\testopt{testcomp}{This compiles and creates a {\tt .ao} file
containing byte-codes.}

\testopt{testphase PHASE}{This compiles and uses the {\tt -W T}
option to create a {\tt .R} file
containing the compiler information for phase PHASE.
See the output for {\tt \asharpcmd{} -W h} for a list of the phases.
%% WE SHOULD FIX THE FOLLOWING
The output of only one phase can be collected.}

\testopt{testgen KIND}{This compiles and produces a file of the
given KIND.  For example, use {\tt c} for KIND to produce a C
file and use {\tt l} to produce a Lisp file.}

Issue {\tt testas -help} for additional information.
Once you have created test files and their output, we suggest you
create a {\tt Makefile} in each of {\tt test/} and {\tt
test/test.out/} to automate your testing.
See the distribution subdirectories by those names for models.

