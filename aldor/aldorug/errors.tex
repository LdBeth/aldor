\newcommand{\mysubhead}[1]{\vskip -2mm {$\bullet$ \bf #1}}
\newcommand{\mysubcont}[1]{\vskip -2mm \hspace{1.9mm} {\bf #1}}
\newcommand{\Dos}[0]{DOS} % I believe that's the standard form.  MGR
%
% *********************************************************************
\head{chapter}{Understanding messages}{asugMsgs}
% *********************************************************************
%

The \asharp{} compiler may display messages of various sorts, including
errors, warnings and remarks about the program being compiled.
This section is a guide to understanding the format and controlling the 
generation of \asharp{} messages.

A complete list of the messages produced by the compiler
may be found in \chapref{asugMessages}.

% *********************************************************************
\head{section}{\asharp{} error messages}{asugErrors}
% *********************************************************************

Even a particularly skilled programmer will have to deal with compiler
error messages at some point.  Upon initial encounter, these messages
often seem confusing.  However, with practice their meaning will
become clear.  The \asharp{} compiler tries to give messages which are
reasonably brief, but still informative.  Although more detailed
messages may be useful to novice programmers, most beginners quickly
move past the initial desire for more detailed messages.  The
following paragraphs describe the components of the messages produced
by the \asharp{} compiler.

% *********************************************************************
\head{subsubsection}{Format of \asharp{} error messages}{asugErrorsFormat}
% *********************************************************************

Each message produced by the \asharp{} compiler is assigned a line and
a column position based on the place in the source text of the program
associated with the message.
Then, for each source line for which messages have been generated, the
compiler reports the source line, a line containing pointers to the
column assigned to each message, and a description of each message
which appears for the source line.

Each source line is reported by showing the source file name in which the
line appears (enclosed in quotation marks), the line number within the
file, followed by a colon (\verb":"), and then the text of the line itself.

After the source line is reported, the next output line contains a
circumflex (\verb+^+), used as a caret below each column at which an
error was detected.
Dots (\verb"...") are used to fill in the spaces between the carets.

Each error that appears on a line is then precisely located and shown
in a format which can be used to quickly locate and fix the error.
For each error, the line and column number are printed first,
enclosed within square brackets (\verb"[ ]"), then the severity of the
error is \index{error messages!severity of}
indicated by one of the keywords \ttin{Fatal}, \ttin{Error},
\ttin{Warning}, or \ttin{Remark} appearing in parentheses.
% I'd rather omit the quotes here - but this seems to fit closer to a
% "form" than a "value" in SMW's terminology.  MGR
Following the severity of the error is the text of the
message, which may include references to parts of the source text,
or references to types or other data structures used to compile the program.

% *********************************************************************
\head{section}{Example showing \asharp{} messages}{asugErrorsExample}
% *********************************************************************

The message format described in the previous section will be illustrated
by examining the error messages produced when we attempt to compile the file
\fname{error0.as} shown in \figref{FigureError0In}.  The command
``\asharpcmd{ error0.as}'' produces the messages shown in
\figref{FigureError0Out}.

\thefile{FigureError0In}{\fname{error0.as} --- A program containing mistakes.}{examples/error0.as}

\thefile{FigureError0Out}{Error messages for \fname{error0.as}.}{examples/error0.out}

The first line of this message indicates that the message was generated
from line~4 of the file \fname{error0.as}.  The next lines specify that
errors were detected at columns 12 and 15 on this line.  The carets
indicate that the first error reported on this line is due
to the operator {\tt =>} and that the second error reported on this line
resulted from the symbol {\tt 1}.
In addition, these are the second and third messages, respectively,
which were encountered during the compilation of \fname{error0.as}.

The next message shown is actually the first detected by the compiler.
It is a ``Warning'' which complains of a missing ``\verb+;+''.
There is also a hint which suggests that piling has been used incorrectly.
However, since the declaration ``\verb+#pile+'' was not included in the file,
piling syntax (see \secref{asugFLangLayout}) is not being used.

\thefile{FigureError1In}{\fname{error1.as} -- A program containing fewer mistakes.}{examples/error1.as}
\thefile{FigureError1Out}{Error messages for \fname{error1.as}.}{examples/error1.out}
\thefile{FigureError2In}{\fname{error2.as} -- A program which compiles.}{examples/error2.as}

Let's add the declaration ``\verb+#pile+'' to \fname{error0.as} and call
the updated file \fname{error1.as}, shown in
\figref{FigureError1In}.  Compiling this file produces the messages
shown in \figref{FigureError1Out}.
Now there is a single error message at line~6, which complains that
no meaning can be found for the operator ``\verb+f+''.  There is certainly
no declaration for ``\verb+f+'' in \fname{error1.as}.  This operator
should in fact be ``\verb+factorial+'', and not ``\verb+f+''.
After making this change (shown in \figref{FigureError2In}),
the file will compile cleanly.

This example was compiled using the default message options as described
in \figref{asugErrorsDefault}.  The compiler's message options
determine what kinds of messages the compiler will generate.

% *********************************************************************
\head{section}{Some common error messages}{asugComErrMsg}
% *********************************************************************

\newcommand{\Situation}[1]{{\bf \underline \sf {#1}}}

This section explains some error messages likely to be seen by new users.

Compiler error messages do not always explain the real nature of
the problem. This statement is true in general, and not only for the
\asharp{} compiler. Compilers analyse programs on the basis of more or
less rigid syntactic rules, and they can become enormously confused by
a missing character, or misspelled word.

What follows is a collection of `mysterious' messages (or, at least,
messages which are puzzling to new \asharp{} programmers), generated by
\asharp{}.

%%%%%%%%%%%%%%%%%%%%%%%%%%%%%%%%%%%%%%%%%%%%%%%%%%%%%%%%%%%%%%%%%%%%%%%%%%%
\mysubsect{`aldor.as' cannot be opened}

{\small
\begin{verbatim}
"myfile.as", line 1: #include "aldor.as"
                     ^
[L1 C1] #1 (Error) Could not open file `aldor.as'.
\end{verbatim}
}

{\small
\begin{verbatim}
"myfile.as", line 1: import from Integer;
                     ............^
[L1 C13] #1 (Error) No meaning for identifier `Integer'.
\end{verbatim}
}

This is probably a message that will occur only once --- it indicates
that \asharp{} has not been set up properly.
\begin{itemize}
\item Check the {\tt ALDORROOT} environment variable. For Unix
users, for \linebreak
example, the file \fname{aldor.as} must be located in the directory
\linebreak \fname{\$ALDORROOT/include}; for \Dos{} users it must be
located in the directory \fname{\%ALDORROOT\%\back include}.  (Type
\ttin{set} on \Dos{} to see the value of this environment variable.
Check the installation instructions to ensure that you have completed
all necessary steps.)

\item (For \Dos{} users only) check ``{\tt FILES=}...'' in your 
\verb"config.sys".  A minimum value of 40 is recommended.
\end{itemize}
See also \secref{asugOptionsEnvironment}.

%%%%%%%%%%%%%%%%%%%%%%%%%%%%%%%%%%%%%%%%%%%%%%%%%%%%%%%%%%%%%%%%%%%%%%%%%%%
\mysubsect{No meaning for integer-style literal `{\it xxx}'}
\mysubhead{No meaning for string-style literal `{\it xxx}'}
\mysubhead{No meaning for float-style literal `{\it xxx.xx}'}

Unlike traditional programming languages, in \asharp{} there is no
built-in knowledge about the meaning of numeric or string constants.
Every domain can export an interpretation for these constants, with the
advantage that you can define your own methods for interpreting them.
The standard \asharp{} library contains some standard domains
exporting interpretations for these constants, but you must explicitly import
these domains in the scope of your program. So in your program you need to
say:

{\small \verb"import from MachineInteger;    "}%
{\em or}{\small \verb"    import from Integer;"}

to specify if you want to use the machine representation or the
infinite
precision representation for integer-style literals such as 2, 3, 4, ...
%
% IS 1 an "integer-style literal"?  It was included (with "i.e.") in the
% original list, here; if so, why not 0 as well?  Or was the "i.e."
% meant to be "e.g."?
%
For string-style literals, such as \ttin{dog} and \ttin{cat}, you can
import from the \ttin{String} domain, and for floating point-style
literals you can import from \ttin{SingleFloat} (single precision
hardware floating point) or \ttin{DoubleFloat} (double precision hardware
floating point).  See also \secref{asugLangExprLiterals}.

%%%%%%%%%%%%%%%%%%%%%%%%%%%%%%%%%%%%%%%%%%%%%%%%%%%%%%%%%%%%%%%%%%%%%%%%%%%
\mysubsect{No one possible return type satisfies the context type}

\asharp{} is a strongly typed language. This means that the type of
each expression forming the program is determined during the
compilation process. The following example shows a typical situation
in which this message occurs:

{\small
\begin{verbatim}
#include "aldor"

foo():Integer == 1;

b: MachineInteger == foo();
\end{verbatim}
}

Compiling this example with the option \option{-M2} (full error
messages), gives:

\begin{small}
\begin{verbatim}
"myfile.as", line 5: b: MachineInteger == foo();
                     .....................^
[L5 C22] #1 (Error) No one possible return type satisfies the context type.
  These possible return types were rejected:
          -- AldorInteger
  The context requires an expression of type MachineInteger.
\end{verbatim}
\end{small}

\ttin{b} is a constant whose type is \ttin{MachineInteger}, so on the right-hand
side of the definition an expression of the same type is required. When
the compiler searches for all the possible return types of the
expression \ttin{foo()}, it
finds that the unique possible type, \ttin{AldorInteger} does not satisfy
\ttin{MachineInteger}\footnote{The error message is in terms of {\tt AldorInteger},
whereas the source code uses {\tt Integer}.  This is because the {\tt aldor.as}
file maps the readable name {\tt Integer} to the unique identifier
{\tt AldorInteger}, as explained in \secref{asugLangSimpleI}.}.
Note that a function could be overloaded, which is why
the message says: ``{\tt No one possible return type}...''. To fix this
problem, \ttin{foo} can be overloaded with a new definition, or the result of
foo can be {\em coerced} to a {\tt MachineInteger} as in:
\begin{verbatim}
b: MachineInteger == foo()::MachineInteger;
\end{verbatim}
% Overgeneral, but I'm not sure what to point at!
See also part II.

\goodbreak
%%%%%%%%%%%%%%%%%%%%%%%%%%%%%%%%%%%%%%%%%%%%%%%%%%%%%%%%%%%%%%%%%%%%%%%%%%%
\mysubsect{No meaning for identifier `{\it domain}'}

\begin{itemize}

\item If you are defining {\it domain} in the same file, are you sure that the
spelling for {\it domain} is correct? (Remember that \asharp{} is 
case-sensitive: \ttin{Integer} and \ttin{integer} are two different
identifiers!)

\item If {\it domain} is from an external library, for example 
\ttin{mylib}, did you
include the \ttin{\#library} statement in your file? For instance:

{\small
\begin{verbatim}
#library MyLib "mylib"
import from MyLib;
\end{verbatim}
}

In this case, if you get the error message in the import statement and the
symbol that you are importing has been defined in a \ttin{\#library}
statement, then check the following rules:
\begin{enumerate}
\item If the library is a library created with \ttin{ar} (with suffix .al), than
you must {\em not} write the suffix. This is the case,  for  \ttin{mylib}
--- it corresponds to the file \fname{libmylib.al}
\item If the library is a \ttin{.ao} file, than you {\bf must} use the
\verb".ao" suffix in the \verb+#library+ statement.
\end{enumerate}
\end{itemize}
See also \secref{asugCmdlineLib}.

%%%%%%%%%%%%%%%%%%%%%%%%%%%%%%%%%%%%%%%%%%%%%%%%%%%%%%%%%%%%%%%%%%%%%%%%%%%
\mysubsect{Argument 1 of `test' did not match any possible parameter}
\mysubcont{type}
% "With" deleted from message database at NAG.  MGR

If the test is for equality, in an \ttin{if} statement, are you sure that you
are using \ttin{=} and not \ttin{==}? This can be a frequent mistake for C/C++
programmers.  The message says ``{\tt Argument 1 of `test'}...'' even if
there is no explicit call to the \ttin{test} operator, because in some
%contexts, such as an \ttin{if} statement, there is an implicit
%application of the \ttin{test} operator.
%% Avoiding bad page break
contexts, such as an \ttin{if} statement, the \ttin{test} operator
is implicitly applied.

%%%%%%%%%%%%%%%%%%%%%%%%%%%%%%%%%%%%%%%%%%%%%%%%%%%%%%%%%%%%%%%%%%%%%%%%%%%
\mysubsect{`Syntax error' at the beginning of a line}

If you are sure about the correctness of the displayed line, look at the
previous line. A macro definition using \ttin{==>} and not ending
with \ttin{;} can often cause this error. For example, compiling the following
statements from the file \fname{myfile.as}:

{\small
\begin{verbatim}
#include "aldor"
-- Not using `#pile'.

MI==>MachineInteger

import from MI;
\end{verbatim}
}

gives the error message:
{\small
\begin{verbatim}
"myfile.as", line 6: import from MI;
                     ^
[L6 C1] #1 (Error) Syntax error.
\end{verbatim}
}
See also \chapref{asugFLang}.

% *********************************************************************
\head{section}{Common pitfalls}{asugComPitf}
% *********************************************************************

This is a collection of mistakes frequently made by novice \asharp{}
programmers. A subclass of this consists of common mistakes for
programmers from other languages, such as C, C++ and Fortran.  It is
important to be able to recognise situations where such mistakes occur,
so that you can understand the compiler's error messages.

%%%%%%%%%%%%%%%%%%%%%%%%%%%%%%%%%%%%%%%%%%%%%%%%%%%%%%%%%%%%%%%%%%%%%%%%%%%
\mysubsect{{\tt `==>'} and {\tt `=>'}}

These symbols have completely different uses in \asharp{}. The
\ttin{==>} (long arrow) symbol is used to define a macro (you can
% a "double arrow" would be =>=>
also use the form ``{\tt macro} ...''), see \chapref{asugLangMacros},
whereas the \ttin{=>} (short arrow) symbol is used as an early exit in a
sequence, see \secref{asugLangExprExit}.

%%%%%%%%%%%%%%%%%%%%%%%%%%%%%%%%%%%%%%%%%%%%%%%%%%%%%%%%%%%%%%%%%%%%%%%%%%%
\mysubsect{`=' and `==', `=' and `:='}

The \ttin{=} symbol is usually used as an equality operator (but can
be overloaded by user programs). The \ttin{==} (double-equal) is used
to define constants --- categories, domains, functions and values are
all \asharp{} constants. The double-equal symbol is sometimes referred
to as the ``very-equal'' symbol for obvious reasons. The \ttin{:=}
operator is used for assignments.

%%%%%%%%%%%%%%%%%%%%%%%%%%%%%%%%%%%%%%%%%%%%%%%%%%%%%%%%%%%%%%%%%%%%%%%%%%%
\mysubsect{Using `{\tt \_}' as a separator in identifiers}

The \ttin{\_} (underscore) symbol is used as an escape character in
\asharp{} --- thus the identifiers \ttin{list\_of\_integers} and
\ttin{listofintegers} are exactly the same.
 For this reason, to prevent ambiguities, we suggest using capital
letters, instead of \ttin{\_}, to separate words in identifiers, as in
\ttin{listOfIntegers}.

%%%%%%%%%%%%%%%%%%%%%%%%%%%%%%%%%%%%%%%%%%%%%%%%%%%%%%%%%%%%%%%%%%%%%%%%%%%
\mysubsect{Missing `;' at the end of a macro definition}

Unlike other languages, such as C, the \asharp{} preprocessor requires
that macro definitions end with a \ttin{;} (or a newline, if you are
using \ttin{\#pile}.  Unfortunately the compiler has no way of knowing
that the statement defining the macro has ended and so will usually complain
about a syntax error at the start of the next line.

%%%%%%%%%%%%%%%%%%%%%%%%%%%%%%%%%%%%%%%%%%%%%%%%%%%%%%%%%%%%%%%%%%%%%%%%%%%
\mysubsect{Undefined symbol \_INIT\_\_xxx referenced from text segment}

This is a message generated by the C compiler (cc, gcc, xlc, \etc{}) that
\asharp{} is using on your platform to generate stand-alone executable
files. 
Everything in the compilation process of the \asharp{} program succeeded, but
something went wrong in the final linking step.

One common cause of this kind of problem is compilation using another
library, as well as/instead of the standard \asharp{} library
\libaldor{}, without letting the compiler know about the other
library.
%
% Using "you" in the above sentence sounds like blame, best avoided.
%
When you include a \ttin{.al} library, you should add
\verb"-l"{\em libname} to the command line\footnote{Note that \asharp{}
options, such as {\tt -l}, are case-insensitive.}.

See \chapref{asugSeparate} and, in particular, 
\secref{asugCmdlineLib} to understand how to use other libraries with
\asharp{}.  A particular point to note when {\em compiling} is:
\begin{itemize}
\item If you want to use a domain from \ttin{mylib} (a library containing
definitions of some domains), you should include the following lines
in your \asharp{} program:

{\small
\begin{verbatim}
#library MyLib "mylib"
import from MyLib;
\end{verbatim}
}

(Note that any identifier may be used in place of \ttin{MyLib}.)
When calling the compiler, you need to include the \option{-lmylib}
option among the parameters; for example:

\osprompt \asharpcmd{\ -lmylib myprog.as}
\end{itemize}


%%%%%%%%%%%%%%%%%%%%%%%%%%%%%%%%%%%%%%%%%%%%%%%%%%%%%%%%%%%%%%%%%%%%%%%%%%%
\mysubsect{User program with the same name as a library file}

If you create a \ttin{.as} file with the same name of a library file,
it won't compile. The compiler will issue a warning message if there
is such a name conflict.  For example, there is a file in \libaldor{}
with the name \ttin{sal\_int.as}; suppose that your file is also named
\ttin{sal\_int.as}, when your file is compiled you will get a
warning, followed by one or more error messages
relating to the symbols being used in your file:
{\small
\begin{verbatim}
#1 (Warning) Current file over-rides existing library in
 `/usr/local/aldor/lib/libaldor.al'.
"pointer.as", line 4: foo():Integer == 1;
                      .................^
[L4 C18] #2 (Error) No meaning for identifier `1'.
\end{verbatim}
}
The easiest way to view the names of the files in a library is usually
to use a command provided for that purpose by your operating system ---
for example, the \ttin{ar} command in Unix (see \secref{asugCmdlineLib}
for an alternative):
{\small
\begin{verbatim}
cd $ALDORROOT/lib
ar vt libaldor.al
ar vt libfoam.al
\end{verbatim}
}

The compiler will issue a warning message if the name of your file is
used by any library needed for compilation of the file.
If you are recompiling a library, then the warning can safely be
ignored.


%\newpage
% *********************************************************************
\head{section}{Controlling compiler messages}{asugErrorsMessages}
% *********************************************************************

The types of messages returned by the \asharp{} compiler are controlled
using the {\tt -M} option, which takes an argument identifying the
kind of message information to display.
Preceding any message option with ``\verb+no-+'' negates that option, 
so {\tt -M no-details} will suppress the details associated with a compiler 
message.  A complete listing of the message options can be found in 
\secref{asugOptionsMessage}.
\index{compiler options!M@\protect{\tt -M}}
%%\index{compiler options!M@\protect{\tt M}!M@\protect{-M no-opt}}

% *********************************************************************
\head{subsubsection}{The default compiler message options}{asugErrorsDefault}
% *********************************************************************

The default message option used by the \asharp{} compiler is {\tt -M 2},
which is equivalent to the following collection of options:  {\tt -M number},
{\tt -M sort}, {\tt -M warnings}, {\tt -M source}, {\tt -M details},
{\tt -M notes}, {\tt -M mactext}, {\tt -M abbrev} and {\tt -M human}.
%\index{compiler options!M@\protect{\tt M}!M@\protect{-M 2}}
%\index{compiler options!M@\protect{\tt M}!M@\protect{-M details}}

The option {\tt -M human} indicates that the compiler 
messages are to be read and interpreted by a human user, as opposed to a 
computer program.
%\index{compiler options!M@\protect{\tt M}!M@\protect{-M human}}

The {\tt -M number} option enumerates the order in which the
compiler messages were encountered.  An ordinal {\tt \#}{\it n} appears
after the bracketed line and column values.  The default option
{\tt -M sort} shows the compiler messages sorted by the order in which
they appear within the file.  Using {\tt -M no-sort} instead causes the
messages to appear in the order in which they are encountered by the
compiler, which may be different from their order in in the program.
%\index{compiler options!M@\protect{\tt M}!M@\protect{-M number}}
%\index{compiler options!M@\protect{\tt M}!M@\protect{-M sort}}

The {\tt -M warnings} option will display the compile time
warnings which do not prevent a successful compilation.  When the option
is changed to {\tt -M no-warnings} then those messages, which are only
advisory in nature, are suppressed.
%\index{compiler options!M@\protect{\tt M}!M@\protect{-M warnings}}

The compiler normally displays the source text which caused each
message.  However, using {\tt -M no-source} displays only the compiler message
with the line and character location without showing the source text.
%\index{compiler options!M@\protect{\tt M}!M@\protect{-M source}}

Some compiler messages may have notes associated with them which cross
reference other compiler messages having a similar or identical meaning.
They are produced by the option {\tt -M notes}.
%\index{compiler options!M@\protect{\tt M}!M@\protect{-M notes}}

Compiler messages abbreviate the data types which are contained within
the text of the message, when the option {\tt -M abbrev} is used.
To expand the type names, use the option {\tt -M no-abbrev}.
%\index{compiler options!M@\protect{\tt M}!M@\protect{-M abbrev}}

%\newpage
% *********************************************************************
\head{section}{Interactive error investigation}{asugErrorsBloop}
% *********************************************************************

In most cases, the error messages will give enough information for
the programmer to determine what is wrong with a program.
In some cases, however, extra detail is required.
Rather than print many lines of debugging information for each error,
the \asharp{} compiler gives messages of moderate length and allows
the programmer to get more information interactively.

The {\tt -M~inspect} option (not to be confused with {\tt -G~loop})
must be given if interactive error investigation is desired.  Then,
whenever the compiler detects an error, it stops and enters
a debugger or ``break loop''.
The break loop gives the programmer access to information such as the
types of variables and the names in scope.
An example using the interactive loop is shown in \figref{FigureBloop}.
\thegeneralfile{h}{FigureBloop}{Interactive Error Investigation}{commands/bloop.rec}
\clearpage
The interactive error inspector allows the following commands:

\begin{tabular}{ll}
\\{\bf help} &    
list the commands, with descriptions
\\{\bf show} &
show the current node
\\{\bf means} &
show the possible meanings of the current node
\\{\bf use} &
show how the current node is used
\\{\bf seman} &
show the extra semantic information for the current \\ & node
\\{\bf scope} &
show information about the current scope
\\
\\{\bf up} &
use the parent as the current node
\\{\bf down} &
use the $0$th child as the current node
\\{\bf next} &
use the next sibling as the current node
\\{\bf prev} &
use the previous sibling as the current node
\\{\bf home} &
return to the original node
\\{\bf where} &
return the source position of the current node 
\\
\\{\bf getcomsg} &
get information on the current message
\\{\bf notes} &   
show the notes associated with the current message
\\{\bf mselect} {\em i} &
select message {\em i} to be the current message
\\{\bf mnext} &   
select the next message in the list
\\{\bf mprev} &   
select the previous message in the list
\\{\bf msg} &     
display the error message again
\\{\bf nice} &    
show with pretty printed form
\\{\bf ugly} &    
show with more detailed, internal form
\\
\\{\bf quit} &
exit the compiler, showing all messages so far
\\ & \\
\end{tabular}

%When the options
%{\tt -M inspect} and {\tt -M no-human} are both given, the interactive
%break loop operates under the same mode as the \xasharp{} debugger.  Refer
%to \chapref{asugXasharp} for more detail on \xasharp{}.%
%\index{compiler options!M@\protect{\tt M}!M@\protect{-M inspect}}
%\index{\xasharpcmd{}@\xasharp{}} This doesn't work
%				  - macro before @ not expanded
%\index{xaxiomxl@\xasharp{}}

\clearpage
% *********************************************************************
\head{section}{Selecting error messages}{asugErrorsSelect}
% *********************************************************************

The \asharp{} compiler is organised in such a way that messages may
be replaced, enabled, and/or disabled from the command line.
\index{error messages!selecting display of}
Messages are identified by a name tag, \eg{} ``AXL\_W\_CantUseLibrary''.
The name tags of all messages used by
the compiler may be found in the catalogue of error messages in
\chapref{asugMessages}.

Once the name of a particular message is identified, the command

{\small
{\tt \asharpcmd{} -M {\it msgname} input.as}
}

can be used to enable
the message {\it msgname} if it is otherwise disabled.  Conversely, the
command 

{\small
{\tt \asharpcmd{} -M no-{\it msgname} input.as}
}

will disable the message {\it msgname} if it is otherwise enabled.
Normally, the messages classified as warnings are enabled,
while those classified as remarks are disabled.
%\index{compiler options!M@\protect{\tt M}!M@\protect{-M msgname} (enable message)}

The special message name ``\verb+warnings+'' can be used to enable or
disable all warning messages, and the special name ``\verb+remarks+'' can
be used to enable or disable all remarks.
``\verb"Error"'' messages cannot be disabled.

In addition to selecting error messages by name, it is possible to use
the command

{\small
{\tt \asharpcmd{} -M emax={\it n}}
}

to specify the number
of error messages which will be reported before \asharpcmd{} gives up and
stops processing the input file.  The default is ten.
%\index{compiler options!M@\protect{\tt M}!M@\protect{-M emax}}

% *********************************************************************
\head{section}{Error messages and macros}{asugErrorsMactext}
% *********************************************************************

When an error occurs in the processing of program text which involves
a macro, it is sometimes useful for the error message to point to the
macro invocation, and sometimes for it to
point into the body of the macro to the location where the error was
detected.  The command 

{\small
{\tt \asharpcmd{} -M mactext}
}

instructs
\asharpcmd{} to point to the text of the macro, while the command

{\small
{\tt \asharpcmd{} -M no-mactext}
}

instructs \asharpcmd{} to point to the macro invocation.
%\index{compiler options!M@\protect{\tt M}!M@\protect{-M mactext}}
\index{macros}

There is no mechanism for instructing \asharpcmd{} to point to the
invocation of a macro which appears in the body of another macro.

% *********************************************************************
\head{section}{Error messages and GNU Emacs}{asugErrorsEmacs}
% *********************************************************************

GNU Emacs offers several commands for compiling and debugging
\index{Emacs}
programs.
\index{error messages!and GNU Emacs}
Among these are ``{\tt M-x compile}'', which can be used to compile
\asharp{} programs.
By default, {\tt M-x compile} issues the command ``{\tt make}'', in a
separate process, but the command may be changed to any other Unix
command, including any command line which invokes the \asharp{}
compiler.
Output from the command goes to the buffer {\tt *compilation*}.
% Quote *compilation* here?
The format of the error messages produced by the \asharp{} compiler
is then understood by the command ``{\tt C-x `}'', which finds the
next error message in the {\tt *compilation*} buffer and displays
the source line which produced the error.
%\index{compiler options!M@\protect{\tt M}!M@\protect{-M compile}}

% *********************************************************************
\head{section}{Using an alternative message database}{asugErrorsDB}
% *********************************************************************

It is possible to have the \asharp{} compiler use an alternative
message database.
\index{error messages!database of}
This is done via the {\tt -Mdb} command line option.
Thus, a command script can cause the compiler to use a database of
messages translated to some other language, or which give a
different level of detail.
%\index{compiler options!M@\protect{\tt M}!M@\protect{-Mdb} (set message database)}

The messages which have been built into the compiler were derived
from the database {\tt samples/comsgdb.msg}.
\index{files!comsgdb.msg@\protect{\tt comsgdb.msg}}
A replacement message database must provide messages for all the
message tags in that file, and should 
be in the X/Open format\footnote{Note that
the {\tt comsgdb.msg} file is {\em not} in that format}.
\index{X/Open message database format}

Once a new message database file has been created, it can be
placed in any directory which is normally searched for executable
programs.
If the database is called {\tt newcomsgs.cat,} for example, then
the compiler argument ``{\tt -Mdb=newcomsgs}'' will find it.
