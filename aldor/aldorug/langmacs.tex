%%%%%%%%%%%%%%%%%%%%%%%%%%%%%%%%%%%%%%%%%%%%%%%%%%%%%%%%%%%%%%%%%%%%%%%%%%%%%
%%%%%%%%%%%%%%%%%%%%%%%%%%%%%%%%%%%%%%%%%%%%%%%%%%%%%%%%%%%%%%%%%%%%%%%%%%%%%
%%%                                                                       %%%
%%% :: Source Macro Expansion
%%%                                                                       %%%
%%%%%%%%%%%%%%%%%%%%%%%%%%%%%%%%%%%%%%%%%%%%%%%%%%%%%%%%%%%%%%%%%%%%%%%%%%%%%
%%%%%%%%%%%%%%%%%%%%%%%%%%%%%%%%%%%%%%%%%%%%%%%%%%%%%%%%%%%%%%%%%%%%%%%%%%%%%
\newcommand{\stdindent}{{\normalsize {\tt \ \ \ \ }}}

\head{chapter}{Source macros}{asugLangMacros}
\index{macros}

\head{section}{Macro definition}{asugLangMacrosDefining}
\asharp{} provides a way to define abbreviations, or {\em macros},
to make programs easier to read.

A common use of macros is to abbreviate names or type expressions,
as in the example below.
The lines containing \ttin{==>} are macro definitions:
\keywordIndex{==>}

\begin{small}
\begin{verbatim}
    -- Abbreviations for frequently used types:
    MI  ==> MachineInteger;
    L T ==> List T;

    split(lmi: L MI): Record(first: MI, second: MI, rest: L MI) ==
            [first lmi, first rest lmi, rest rest lmi];

    -- Abbreviations for long package names:
    MStudy ==> LongitudinalStudiesOfMitralValveReplacementPatients;

    d0 := intakes()$MStudy;
    d1 := examinations()$MStudy;
\end{verbatim}
\end{small}

The left hand side of a macro definition may be either an
identifier or an application.
A definition of the form

\stdindent {\it Op} {\it Parms} {\tt ==>} {\it Body}

is equivalent to

\stdindent {\it Op} {\tt ==> macro} {\it Parms} {\tt +->} {\it Body}
\keywordIndex{macro}

and defines a parameterised macro.
Any application syntax is permitted on the original left hand side:
prefix, infix, or bracketed.
The resulting right hand side is called a \Term{macro function}.

This rule is applied until the left hand side is an identifier.
Macros \linebreak may consequently be defined to receive their arguments
in groups, in a ``curried'' fashion.
For example, when $P_1$ and $P_2$ are parameter sequences, the definition

\stdindent {\tt {\it Op} ($P_1$)($P_2$) ==> {\it Body}}

is equivalent to

\stdindent {\tt {\it Op} ==> macro ($P_1$) +-> macro ($P_2$) +-> {\it Body}}

Macro functions may appear directly in the source program,
and may be written with their parameters in groups:

\stdindent {\tt macro ($P_1$)($P_2$)...($P_n$) +-> {\it Body}}

is equivalent to

\stdindent {\tt macro ($P_1$) +-> macro ($P_2$) +-> ... macro ($P_n$) +-> {\it Body}}

\head{section}{Macro expansion}{asugLangMacrosExpansion}
The process or replacing the abbreviations with what they stand for
is called {\em macro expansion}.
Macro expansion works by making substitutions
in the parsed form of programs.
Because the substitution is on trees, it is not
necessary to include extra parentheses in macro definitions.

Macro expansion transforms the source syntax trees with the following steps:
\begin{itemize}
\item
   Record and remove any new macro definitions encountered.
   A macro definition is scoped and persists for the remainder of the
  ``\verb"+->"'', ``\verb"where"'', ``\verb"add"'' or ``\verb"with"''
  in which it occurs.
\item
   Replace identifiers, if they have macro definitions,
   with the right-hand side of their macro definitions.
\item
   Reduce expressions of the form

\stdindent {\tt (macro {\it Parms} +-> {\it Body}) ({\it Exprs})}

by introducing new macro definitions for the formal parameters
and replacing the application with the macro expanded body.
\end{itemize}
Once macro expansion is complete, the entire program should be
free of macro definitions and macro functions.
The process of macro expansion removes all macro definitions;
any remaining unreduced macro functions are reported as errors.


\head{section}{Examples}{asugLangMacrosExamples}

Given the following definitions,

\begin{small}
\begin{verbatim}
    a ==> A1 - A2;
    b ==> B;
    c ==> C;
    d(e,f)(g,h) ==> (e+f)*(g+h);
    x + y    ==> c(x,y);
    f(g,x,y) ==> g(x,y) / g(y,x);
\end{verbatim}
\end{small}

the following expressions expand as shown:

\begin{small}
\begin{verbatim}
    a;                                            --> A1 - A2
    b;                                            --> B
    a + b;                                        --> C(A1 - A2, B)
    d(1,2)(3,4);                                  --> C(1,2) * C(3,4)
    f(+,u,v);                                     --> C(u,v) / C(v,u)
    f((macro (a,b) +-> a), u, v);                 --> u / v
    (macro (a,b)(c)(d) +-> [a,b,c,d])(3,4)(5)(6); --> [3, 4, 5, 6]
\end{verbatim}
\end{small}


\head{section}{Points of style}{asugLangMacrosStyle}
It is often convenient to use macros which are local to a function
or expression.  Below, we show a macro \ttin{l1?} which is
local to the function \ttin{f}.

\begin{small}
\begin{verbatim}
    f(li: List Integer, lp: List Pointer): () == {

            l1? x0 ==> (not empty? x and empty? rest x) where x := x0;

            if l1? li then ...
            if l1? lp then ...
            ...
    }
\end{verbatim}
\end{small}
The purpose of \ttin{l1?} is to provide an inexpensive test
whether a list has length one.
%(If a list \ttin{li} might be long,
%then asking the question \ttin{\#li = 1} can be expensive.)
This example illustrates two additional points:
\begin{itemize}
\item
   Since macros are strictly syntactic substitutions the \ttin{l1?}
   macro may be applied to different values of different types.
\item
   Since the macro uses the parameter in two places, the best practice
   is to introduce a new temporary variable so the argument is evaluated
   only once.
\end{itemize}

If the body of the macro is large, having a \ttin{where} at the end of
a long expression can make a program harder to read.
A convenient way to introduce local variables in this case is to make
the left-hand side of the \ttin{where} a local macro.

% Better not use "BigMac" - they might object to the possibility of
% "Bad Food"!
\begin{small}
\begin{verbatim}
    BigBurger(patty1, patty2) ==> BB where {
            p1 := patty1;
            p2 := patty2;

            BB ==> {
                    if not allBeef? p1 or not allBeef? p2 then
                            error "Bad food";

                    b := burger();
                    b << bottomBun;
                    b << p1;
                    b << cheese;
                    b << lettuce;
                    b << onions;
                    b << sauce;
                    b << middleBun;
                    b << p2;
                    b << cheese;
                    b << lettuce;
                    b << onions;
                    b << sauce;
                    b << topBun;
                    b
            }
    }
\end{verbatim}
\end{small}

Just as a macro function

\stdindent {\tt macro {\it Parms} +-> {\it Body}}

is analogous to a function \ttin{{\it Parms} +-> {\it Body\/}},
a macro definition

\stdindent {\tt {\em Lhs} ==> {\em Expr}}.

is analogous to a definition \ttin{{\it Lhs} == {\it Expr\/}}.
For this reason, a macro definition may be written
in the equivalent form

\stdindent {\tt macro {\em Lhs} == {\em Expr}}.

