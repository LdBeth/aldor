% *********************************************************************
\head{chapter}{Using \asharp{} with C}{asugUsingWithC}
% *********************************************************************

\thegeneraltwofile{p}{AtoCFigure}%
{\asharp{} code using a C function.}{examples/arigato.as}{examples/nputs.c}
\thegeneraltwofile{p}{CtoAFigure}%
{C code using an \asharp{} function.}{examples/cside.c}{examples/aside.as}
\index{C, interface with!{\em see \chapref{asugUsingWithC}}}

Functions and data structures may be shared between programs written
in \asharp{} and other languages.  Here we give simple examples
of sharing functions in a mixed \asharp{} and C programming environment.

\asharp{} has types corresponding to the primitive C types.
These will be described in \secref{asugUsingWithCTypes}.

\section{Using C code from \asharp{}}

For the first example, we show how to call a C function from \asharp{}.
The \asharp{} file \fname{arigato.as} refers to the function \ttin{nputs,}
supplied by the C file \fname{nputs.c}.
Figure \ref{AtoCFigure} shows these files.

The commands to compile these two files and link them together,
say on Unix, are:
\shio{arigato}

The first command produces the object file \fname{nputs.o}.
The second command compiles the file \fname{arigato.as}
and links it with our other file to form an executable program.
Finally, the third command runs the resulting program.

%\pagebreak
The \asharp{} compiler can make use of C-generated object files,
whether they are kept loose or packaged with others in a library archive.
The \asharp{} file using the code or data from C must declare it with
the statement 
\ttin{import ... from Foreign C}: this is the purpose of this
statement in \fname{arigato.as}.
When a function is imported from C, a declaration is placed at the
head of the generated C file. This declaration is constructed using
the data correspondence below. 

To call a C function or macro (such as {\tt fputc}) defined in a header file
(such as \fname{<stdio.h>} or \fname{myfile.h}) an import of the following form should be used:
\begin{verbatim}
import { fputc: (MachineInteger, OutFile) -> MachineInteger }
       from Foreign C "<stdio.h>";

import { myfun: MachineInteger -> () }
       from Foreign C "myfile.h";
\end{verbatim}
\keywordIndex{import}
The filename indicates the file to include when generating C. No
declaration for {\tt fputc} or {\tt myfun} is produced in the generated C
--- it is assumed that all the imports are declared
(or, in the case of macros, defined). 

A ``\verb+#include+'' line is produced in the generated C
for every foreign header file mentioned in
the source code, even when no imported function is used.
One use of this would be to allow
some of the definitions in \fname{foam\_c.h} to be over-ridden. For
example, one could replace the memory management primitives with
operations specifically optimised for the current
application\footnote{This is only useful in very unusual
circumstances.}.

%\pagebreak
\section{Using \asharp{} code from C}


For the second example, we show how to call an \asharp{} function from C.

C code which uses \asharp{} functions should include the file
\fname{foam\_c.h}.  This file contains the C type definitions
which correspond to the various \asharp{} primitive types.
For example, \ttin{FiSInt} is a {\tt typedef} for the C type corresponding
to \ttin{MachineInteger}.
On Unix, the full path name for this file is
\fname{\$ALDORROOT/include/foam\_c.h}.
\index{files!foam\_c.h}

For this example, the C file \fname{cside.c} refers to the function \ttin{lcm,}
supplied by the \asharp{} file \fname{aside.as}.
These files are shown in figure \ref{CtoAFigure}.

The commands to compile, link, and run these files are:
\shio{cside}

The first command compiles the \asharp{} code in the normal way
to produce an object file.
On Unix, this produces the object file \fname{aside.o}.

Compiling the C code which uses \fname{aside.o}
requires the use of a \option{-I} option to tell the compiler where
\index{compiler options!I@\protect{-I}}
to find \fname{foam\_c.h}.

Additional options are needed to link an executable program:
a \option{-L} option tells the C compiler where to look for libraries,
and the \option{-l} options list the libraries which provide \asharp{}
support functions.
%\index{compiler options!l@\protect{-L}}

The \option{-laldor} option provides a library with basic \asharp{} types
such as floating point numbers, lists, file I/O, and so on.
\index{compiler options!l@\protect{\tt -laldor}}
\index{compiler options!l@\protect{\tt -lfoam}}

The \option{-lfoam} option provides a library with run-time support for
such things as memory management and  big integer arithmetic.
Applications can supply their own run time support library instead, if desired
%%The steps to do this will be described in Section \ref{asugReplacingFoam}
%%but,
(this involves providing alternative macro definitions to those
in \fname{foam\_c.h} and a C file with whatever code is needed by the macros).

The \option{-lm} option makes the standard C math library available.
\index{compiler options!l@\protect{\tt -lm}} Because of
the way \asharp{} compiles domains, this generally needs to be
included even if no operations from the math library are used.

In \fname{aside.as}, the line beginning \ttin{export to} tells the
\keywordIndex{export}
compiler that a wrapper function called {\tt lcm} should be generated
for the \asharp{} function with the same name. This wrapper will
convert the C calling convention into that used by
\asharp{} using the rules in the next section. Currently it is
possible to export only functions in this way (an \asharp{} constant
can be wrapped in a function, and types have no particular use in C).

\head{section}{Data correspondence}{asugUsingWithCTypes}
%% FIXME: This section needs to understand packed arrays etc.

This section describes the correspondence between the way data values are
represented in \asharp{} and the way they are represented in C.
It should be possible from this to understand which \asharp{}
declaration will correspond to a declaration in C, and {\em vice versa\/}.

\asharp{}'s abstract machine defines a number of types which
correspond to types on the target machine (in this case C on top of
some operating system). 
The ``{\tt Machine}'' package, described in \spadref{asugLangTypesMachine},
exports the types provided by the abstract machine.
All \asharp{} values are represented internally as
elements of one of these types. 
The complete listing and definition of
the types is given in the FOAM reference guide.  
%% PI: are we giving this guide with A#? If not, we must give a reference.

Because many \asharp{} domains can be parameterized over different
types, \asharp{} uses a pointer-sized object when passing
domains. Thus, double precision floating point numbers (which are
typically bigger than pointers)  are ``boxed'',
and a pointer to the box is passed, rather that the number
itself. Types which are the same size or smaller than pointer-size are
cast to the pointer type when used in a generic context and cast back
as appropriate.

In order to make the underlying types available, \asharp{} provides
the \\ \ttin{Machine} package, which exports these types and operations
on them. For example, the underlying representation type of {\tt
DoubleFloat} is \\ {\tt DFlo\$Machine}\footnote{or equivalently {\tt
BDFlo}, which is a macro defined in \fname{aldor.as}}. This type
should be used when calling foreign functions, and the result coerced
back to appropriate generic type at the \asharp{} level.

Records in \asharp{} are represented by an aggregate type of some kind
in the hosting language. For example, in Scheme a vector is used (and
all objects are the same size anyway). In C, structures are used. When
calling C-defined functions that use records it is important to ensure
that the elements of the \asharp{} record correspond to elements in the
C structure. This implies that records intended for use in Foreign
functions should use the underlying types, rather than the user-level
types\footnote{A brief perusal of the file
\fname{\$ALDORROOT/samples/lib/libaldorX11} provides a rather extended example
of this.}.

The \asharp{} types correspond to C types which are
given as {\tt typedef}s
in the file \ttin{\$ALDORROOT/include/foam\_c.h}.  
The following table shows the correspondance between the types
exported from the \asharp{} package \ttin{Machine} and C:

{\tt
\begin{tabular}{lll}
\asharp{} type &  C typedef &  Usual C type \\
\hline \\
 \verb+Nil$Machine+          &  \verb+FiNil+      &  \verb+Ptr+       \\
 \verb+Word$Machine+         &  \verb+FiWord+     &  \verb+int+       \\
 \verb+Arb$Machine+          &  \verb+FiArb+      &  \verb+long int+  \\
 \verb+Ptr$Machine+          &  \verb+FiPtr+      &  \verb+Ptr+       \\
 \verb+Bool$Machine+         &  \verb+FiBool+     &  \verb+char+      \\
 \verb+Byte$Machine+         &  \verb+FiByte+     &  \verb+char+      \\
 \verb+HInt$Machine+         &  \verb+FiHInt+     &  \verb+short+     \\
 \verb+SInt$Machine+         &  \verb+FiSInt+     &  \verb+long+      \\
 \verb+Char$Machine+         &  \verb+FiChar+     &  \verb+char+      \\
 \verb+Arr$Machine+          &  \verb+FiArr+      &  \verb+Ptr+       \\
 \verb+Rec$Machine+          &  \verb+FiRec+      &  \verb+Ptr+       \\
 \verb+BInt$Machine+         &  \verb+FiBInt+     &  \verb+Ptr+       \\
 \verb+SFlo$Machine+         &  \verb+FiSFlo+     &  \verb+float+     \\
 \verb+DFlo$Machine+         &  \verb+FiDFlo+     &  \verb+double+    \\
 {\em A\/} \verb"->" {\em B} &  \verb+FiClos+     &  \verb+struct _FiClos *+ \\
% ---                     &  \verb+FiEnv+      &  \verb+struct _FiEnv *+  \\
% ---                     &  \verb+FiProg+     &  \verb+struct _FiProg *+ \\
% ---                     &  \verb+FiFun+      &  \verb+Ptr (*X)()+        \\
\end{tabular}
}

Here \ttin{Ptr} is defined as the type \ttin{char *} for compatibility
with old C dialects, but could equally well be defined as \ttin{void *}.

All other \asharp{} types defined by the system (\eg{}~in the
\libaldor{} library) or by a user correspond to the C 
typedef
\ttin{FiWord}. This includes {\tt Integer}, {\tt MachineInteger} and so
on. The data correspondence on most 32 bit machines allows one to
treat {\tt MachineInteger} and \verb+SInt$Machine+ as the same type
(which is the reason that \ttin{arigato}, above, works). However, on
other machines, for example 16 or 64 bit machines, the two types are
not equivalent.

Values belonging to \ttin"Record" types, are pointers to C structs of
the corresponding members.  For example, the C declaration

\begin{small}%
\begin{verbatim}
struct {
        int    x;
        short  y;
        double z;
} *r;
\end{verbatim}
\end{small}
%
corresponds to the the \asharp{} declaration
%
\begin{small}%
\begin{verbatim}
local r: Record(
        x: SInt$Machine,    -- or  x: MachineInteger
        y: HInt$Machine, 
        z: DFlo$Machine
);
\end{verbatim}
\end{small}

Functions which return no value, or more than one value,  are
declared to be of type void, and additional return results are
returned through pointers passed in as additional arguments.

Thus, the expression:
\begin{small}%
\begin{verbatim}
import {
        foo: (Integer, Integer) -> (SInt$Machine, BInt$Machine)
} from Foreign C;
\end{verbatim}
\end{small}
implies that foo should be declared (in ANSI C) as:
{\small
\begin{verbatim}
static void foo(FiWord, FiWord, FiSInt *, FiBInt *);
\end{verbatim}
}

%$ALDORROOT/samples/lib/.. whereever mentioned.

A number of examples of exporting and importing C-defined functions
can be found in \fname{\$ALDORROOT/samples/test}.

%Examples using Foreign C "<stdio.h>"
%Foreign C "myheader.h"

%\section{\asharp{} Types}
%
% This section describes the representation of \asharp{} types as FOAM
% objects. 
%
% Unfortunately, there is almost no point doing this as we haven't
% exported any useful operations for combining or producing
% hashcodes, plus it's all a bit complicated really...

