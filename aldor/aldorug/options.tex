% *********************************************************************
\head{chapter}{Command line options}{asugOptions}
%
%*********************************************************************

\index{compiler options!{\em see \chapref{asugOptions}}}

The \asharpcmd{} command has the following general form:

\asharpcmd{ \[\meta{options}\] \meta{file1} \meta{file2} ...}

This compiles the files one after the other,
each in a fresh environment.  Depending on the particular command line
options given by the user, the files
resulting from the compilations may be combined or run.

The options are case-insensitive so \ttin{-G INTERP} is treated the same
way as \ttin{-g interp}.

If the environment variable \ttin{ALDORARGS} is defined, its
contents are handled as options before the ones on the command line.

% *********************************************************************
\head{section}{File types}{asugOptionsFiles}
% *********************************************************************

Files with the following type extensions are accepted on the command line:
\index{file types}

\dtdd{.as}{%
  \meta{\asharp{} source.}
  If a file name has no type extension,
  then it is treated as if it were a \fname{.as} file.}
%\index{files!.as}
\index{as@.as}
\dtdd{.ai}{%
   \meta{Included \asharp{} source.}
   This is a file with all ``{\tt \#include}''
   and ``{\tt \#if}'' statements processed.
   This sort of file is produced by running the compiler with a
   \option{-Fai} option.}
%\index{files!.ai}
\index{ai@.ai}
%\index{compiler options!F@\protect{\tt F}!F@\protect{-Fai}}
\dtdd{.ap}{%
   \meta{Parsed \asharp{} source.}
   This is a file with the program in an S-expression syntax.
   This is the easiest way to have the \asharp{} compiler process
   programs generated by a Lisp program.
   This sort of file is produced by running the compiler with a
   \option{-Fap} option.}
%\index{files!.ap}
\index{ax@.ap}
%\index{compiler options!F@\protect{\tt F}!F@\protect{-Fax}}
\dtdd{.fm}{%
   \meta{Foam source.}
   ``Foam'' is an acronym for ``First Order Abstract Machine''.
   The \asharp{} compiler produces Foam as its intermediate code.
   An \fname{fm} file contains Foam code in S-expression syntax.
   This sort of file is produced by running the compiler with a
   \option{-Ffm} option.}
\index{Foam}
%\index{files!.fm}
\index{fm@.fm}
%\index{compiler options!F@\protect{\tt F}!F@\protect{-Ffm}}
\dtdd{.ao}{%
   \meta{Machine-independent object file.}
   This is the result of compiling an \asharp{} source file.
   It contains type information, documentation, Foam code, symbol table
   information, and so on.
   Characters are internally represented in ASCII form, and floating
   point numbers are represented in a transportable format guaranteed
   not to loose significance or exponent range on any platform.}
%\index{files!.ao}
\index{ao@.ao}
\dtdd{.al}{%
   \meta{Archive of machine-independent object files.}
   This is treated as a homogeneous aggregated library by the compiler.
   The file is laid out as a Unix-style archive on all platforms.
   Thus both \fname{.ao} and \fname{.al} files may be moved from machine
   to machine.}
%\index{files!.al}
\index{al@.al}
\dtdd{\meta{obj}}{%
   \meta{Object file. }
   This is a platform-specific object file.
   These are named in the usual way for the platform.
   For example on Unix these files have extension \fname{.o},
   while on DOS they have extension \fname{.obj} and on CMS they have extension
   \fname{TEXT}.}
%\index{files!.obj}
%\index{files!.o}
\dtdd{\meta{arch}}{%
   \meta{Archive of object files. }
   The name, contents, and format of these archives is determined by
   the platform.  For example on Unix this would be a \fname{.a} file
   containing \fname{.o} files.}
%\index{files!.a}
\index{archive files}


% *********************************************************************
\head{section}{General options}{asugOptionsGenOpts}
% *********************************************************************

\dtdd{-V}{Run verbosely, giving compilation information.}
%
\dtdd{-D \meta{id}}{Add global assertion as \ttin{\#assert \meta{id}}.}
%
\dtdd{-U \meta{id}}{Remove global assertion as \ttin{\#unassert \meta{id}}.}
%
\dtdd{-A \meta{fn}}{Read command line options from response file \meta{fn}.}
%
\dtdd{-K \meta{n}}{Compile only the first \meta{n} files (0-9).}
%
\dtdd{--}{Treat remaining arguments as input files.}
%
\dtdd{-H ...}{Help.}
%
\dtdd{-B \meta{dir}}{Use \meta{dir} as the base directory for
\asharp{} system files.}
%
\dtdd{-I \meta{dir}}{Search \meta{dir} for additional include
files.}
%
\dtdd{-Y \meta{dir}}{Search \meta{dir} for additional libraries.}
%
\dtdd{-R \meta{dir}}{Put the resulting files in directory \meta{dir}.}
%
\dtdd{-L ...}{Use the given library.}
%
\dtdd{-F ...}{Indicate which output files are to be generated.}
%
\dtdd{-E \meta{fn}}{Specify the main entry point.}
%
\dtdd{-G}{Run the program.}
%
\dtdd{-O}{Standard optimisations.}
%
\dtdd{-Q ...}{Select code optimisations.}
%
\dtdd{-Z ...}{Debugging and profiling options.}
%
\dtdd{-C ...}{Control C generation.}
\dtdd{-S ...}{Control Lisp generation.}
\dtdd{-M}{Control compiler messages.}
\dtdd{-W}{Developer options.}
%\index{compiler options!V@\protect{\tt V}}
%\index{compiler options!V@\protect{\tt V}!F@\protect{-V}}
%\index{compiler options!D@\protect{\tt D}}
%\index{compiler options!D@\protect{\tt D}!F@\protect{-D}}
%\index{compiler options!U@\protect{\tt U}}
%\index{compiler options!U@\protect{\tt U}!F@\protect{-U}}
%\index{compiler options!A@\protect{\tt A}}
%\index{compiler options!A@\protect{\tt A}!F@\protect{-A}}
%\index{compiler options!K@\protect{\tt K}}
%\index{compiler options!K@\protect{\tt K}!F@\protect{-K}}
%\index{compiler options!-@\protect{\tt --}}
%\index{compiler options!H@\protect{\tt H}}
%\index{compiler options!H@\protect{\tt H}!F@\protect{-H}}
%\index{compiler options!B@\protect{\tt B}}
%\index{compiler options!B@\protect{\tt B}!F@\protect{-B}}
%\index{compiler options!I@\protect{\tt I}}
%\index{compiler options!I@\protect{\tt I}!F@\protect{-I}}
%\index{compiler options!Y@\protect{\tt Y}}
%\index{compiler options!Y@\protect{\tt Y}!F@\protect{-Y}}
%\index{compiler options!R@\protect{\tt R}}
%\index{compiler options!R@\protect{\tt R}!F@\protect{-R}}
%\index{compiler options!L@\protect{\tt L}}
%\index{compiler options!L@\protect{\tt L}!F@\protect{-L}}
%\index{compiler options!F@\protect{\tt F}}
%\index{compiler options!F@\protect{\tt F}!F@\protect{-F}}
%\index{compiler options!E@\protect{\tt E}}
%\index{compiler options!E@\protect{\tt E}!F@\protect{-E}}
%\index{compiler options!G@\protect{\tt G}}
%\index{compiler options!G@\protect{\tt G}!F@\protect{-G}}
%\index{compiler options!O@\protect{\tt O}}
%\index{compiler options!O@\protect{\tt O}!F@\protect{-O}}
%\index{compiler options!Q@\protect{\tt Q}}
%\index{compiler options!Q@\protect{\tt Q}!F@\protect{-Q}}
%\index{compiler options!Z@\protect{\tt Z}}
%\index{compiler options!Z@\protect{\tt Z}!F@\protect{-Z}}
%\index{compiler options!C@\protect{\tt C}}
%\index{compiler options!C@\protect{\tt C}!F@\protect{-C}}
%\index{compiler options!S@\protect{\tt S}}
%\index{compiler options!S@\protect{\tt S}!F@\protect{-S}}
%\index{compiler options!M@\protect{\tt M}}
%\index{compiler options!M@\protect{\tt M}!F@\protect{-M}}
%\index{compiler options!W@\protect{\tt W}}
%\index{compiler options!W@\protect{\tt W}!F@\protect{-W}}

% *********************************************************************
\head{section}{Help options}{asugOptionsHelp}
% *********************************************************************
\index{help}

\widedtdd{-H elp}{%
  Brief, general help.}
%\index{compiler options!H@\protect{\tt H}!H@\protect{-H elp}}
\widedtdd{-H all}{%
  Help about {\em all\/} options.}
%\index{compiler options!H@\protect{\tt H}!H@\protect{-H all}}
\widedtdd{-H files}{%
  Help about input file types.}
%\index{compiler options!H@\protect{\tt H}!H@\protect{-H files}}
\widedtdd{-H options}{%
  Help about summary of options.}
%\index{compiler options!H@\protect{\tt H}!H@\protect{-H options}}
\widedtdd{-H [A|args]}{%
  Help about argument gathering options.}
%\index{compiler options!H@\protect{\tt H}!H@\protect{-H args}}
\widedtdd{-H [H|help]}{%
  Help about help options.}
%\index{compiler options!H@\protect{\tt H}!H@\protect{-H help}}
\widedtdd{-H dir}{%
  Help about directory and library options.}
%\index{compiler options!H@\protect{\tt H}!H@\protect{-H dir}}
\widedtdd{-H [F|fout]}{%
  Help about output file options.}
%\index{compiler options!H@\protect{\tt H}!H@\protect{-H fout}}
\widedtdd{-H [G|go]}{%
  Help about execution options.}
%\index{compiler options!H@\protect{\tt H}!H@\protect{-H go}}
\widedtdd{-H [O|Q|optimise]}{%
  Help about optimisation options.}
%\index{compiler options!H@\protect{\tt H}!H@\protect{-H optimize}}
\widedtdd{-H [Z|debug]}{%
  Help about debugging options.}
%\index{compiler options!H@\protect{\tt H}!H@\protect{-H debug}}
\widedtdd{-H C}{%
  Help about C code generation options.}
%\index{compiler options!H@\protect{\tt H}!H@\protect{-H C}}
\widedtdd{-H [S|lisp]}{%
  Help about Lisp code generation options.}
%\index{compiler options!H@\protect{\tt H}!H@\protect{-H lisp}}
\widedtdd{-H [M|message]}{%
  Help about message options.}
%\index{compiler options!H@\protect{\tt H}!H@\protect{-H message}}
\widedtdd{-H [W|dev]}{%
  Help about developer options.}
%\index{compiler options!H@\protect{\tt H}!H@\protect{-H dev}}

% *********************************************************************
\head{section}{Argument gathering options}{asugOptionsArgGath}
% *********************************************************************
%\index{command line arguments}

\dtdd{-A \meta{fn}}{%
  Read command line options from response file \meta{fn}.}
%\index{compiler options!A@\protect{\tt A}!A@\protect{-A}}
\dtdd{-K \meta{n}}{%
  Use only the first \meta{n} files (0-9).
  The remaining arguments are given to the \asharp{} program,
  if run with \ttin{-G}.}
%\index{compiler options!K@\protect{\tt K}!H@\protect{-K}}
\dtdd{--}{%
  Treat remaining arguments as input files, even if they begin with \ttin{-}.
  If the environment variable \ttin{ALDORARGS} is defined, its contents are
  handled as options before the command line.}
%\index{compiler options!-@\protect{\tt --}}
\index{environment variables}
\index{ALDORARGS}

% *********************************************************************
\head{section}{Directories and libraries options}{asugOptionsDirLib}
% *********************************************************************

\index{directories}
\dtdd{-I \meta{dir}}{%
  Search \meta{dir} for include files.
  The environment variable \ttin{INCPATH} provides default locations.}
\index{include@\protect{\tt \#include}}
\index{environment variables}
\index{INCPATH}
%\index{compiler options!I@\protect{\tt I}!I@\protect{-I}}
\dtdd{-Y \meta{dir}}{%
  Search \meta{dir} for additional libraries.
  The environment variable \ttin{LIBPATH} provides default locations.}
\index{libraries}
\index{environment variables}
\index{LIBPATH}
%\index{compiler options!Y@\protect{\tt Y}!Y@\protect{-Y}}
\dtdd{-R \meta{dir}}{%
  Put the resulting files in directory \meta{dir}.
  The default is the current directory.}
%\index{compiler options!R@\protect{\tt R}!R@\protect{-R}}
\dtdd{-B \meta{dir}}{%
  Use \meta{dir} as the base directory for \asharp{} system files.
  If the \ttin{-B} option is omitted, then the base directory must be given 
  by the environment variable \ttin{ALDORROOT}.}
\index{ALDORROOT}
\index{environment variables}
%\index{compiler options!B@\protect{\tt B}!B@\protect{-B}}
\index{libraries}
\dtdd{-L \meta{fn}}{%
  Use the library given by file name \meta{fn}.
  An alphanumeric \meta{fn} is a short form for \ttin{lib\meta{fn}.a}.}
%\index{compiler options!L@\protect{\tt L}!L@\protect{-L}}
\dtdd{-L \meta{id}=\meta{fn}}{%
  Same as \ttin{-L \meta{fn}}, but the source name \meta{id} is
  associated with the library.}

% *********************************************************************
\head{section}{Generated file options}{asugOptionsFile}
% *********************************************************************

The \option{-F} option indicates which output files are to be
generated.
Options marked ``{(*)}'' cause one file of the given type to be generated for 
each input file, whilst those marked ``{(1)}'' cause one file to be generated 
for the entire compilation.

\index{file types}

\dtdd{-F ai}{%
{(*)} Source after all {\tt \#include}
statements have been processed (from \fname{.as})}
%\index{compiler options!F@\protect{\tt F}!F@\protect{-Fai}}
\dtdd{-F ax}{%
{(*)} Macro-expanded parse tree  (from \fname{.as})}
%\index{compiler options!F@\protect{\tt F}!F@\protect{-Fax}}
\dtdd{-F asy}{%
{(*)} Symbol information}
%\index{compiler options!F@\protect{\tt F}!F@\protect{-Fasy}}
\dtdd{-F ao}{%
{(*)} Machine-independent object file}
%\index{compiler options!F@\protect{\tt F}!F@\protect{-Fao}}
\dtdd{-F fm}{%
{(*)} Foam code}
%\index{compiler options!F@\protect{\tt F}!F@\protect{-Ffm}}
\dtdd{-F lsp}{%
{(*)} Lisp code}
%\index{compiler options!F@\protect{\tt F}!F@\protect{-Flsp}}
\dtdd{-F c}{%
{(*)} C code}
%\index{compiler options!F@\protect{\tt F}!F@\protect{-Fc}}
\dtdd{-F o}{%
{(*)} Object file}
%\index{compiler options!F@\protect{\tt F}!F@\protect{-Fo}}
\dtdd{-F x}{%
{(1)} Executable file}
%\index{compiler options!F@\protect{\tt F}!F@\protect{-Fx}}
\dtdd{-F aldormain}{%
{(1)} Generate \ttin{aldormain.c} containing function \ttin{main}.}
%\index{compiler options!F@\protect{\tt F}!F@\protect{-Faldormain}}

  If no \ttin{-F} or \ttin{-G} option is given, then \ttin{-Fao} is assumed.

  If the parameter \meta{fn} is given, it is used as the
  name of the corresponding output file.

  The compiler will not overwrite a C or Lisp file
  it did not generate.

% *********************************************************************
\head{section}{Execution options}{asugOptionsExec}
% *********************************************************************

\dtdd{-G run}{%
  Compile the program to machine code, and then run it.
  The executable file is removed afterwards unless the \ttin{-Fx} option
  is present.}
\index{running programs}
%\index{compiler options!G@\protect{\tt G}!G@\protect{-Grun}}
\dtdd{-G interp}{%
  Translate the program to Foam code, and then run it in interpreted mode.
  The \fname{.ao} file is removed afterwards unless the \ttin{-Fao} option
  is present.}
\index{running programs}
\index{Foam}
%\index{compiler options!G@\protect{\tt G}!G@\protect{-Ginterp}}
\dtdd{-G loop}{%
  Run interactively, interpreting each expression typed by the user.
  With this option, the file \fname{aldorinit.as} is used for initialisation.}
%\index{compiler options!G@\protect{-Gloop}}
\dtdd{-E \meta{fn}}{%
  Use the input file \fname{\meta{fn}.*} as the main entry point.
  The default is the first file.  \ttin{-E} is useful only with \ttin{-Fx},
  \ttin{-Grun} or \ttin{-Ginterp}.}
\index{main entry point}
%\index{compiler options!F@\protect{\tt F}!F@\protect{-Fx}}
%\index{compiler options!E@\protect{\tt E}!E@\protect{-E}}

% *********************************************************************
\head{widesection}{Optimisation options}{asugOptionsOptimize}
% *********************************************************************

Combinations can be used, \eg{} \ttin{-OQno-cc} or \ttin{-Qno-all -Qcc}.
The default is \ttin{-Qcfold -Qdeadvar -Qpeep}.
The option \ttin{-Qcc} may exclude \ttin{-[gp]} on some platforms.
\index{optimisation}

\widedtdd{-O}{%
  Optimise.  This is equivalent to \ttin{-Q2}.}
%\index{compiler options!O@\protect{\tt O}!O@\protect{-O}}
\widedtdd{-Q \meta{n}}{%
  Select a level of code optimisation. (default is \ttin{-Q1}).}
%\index{compiler options!Q@\protect{\tt Q}}
%\index{compiler options!Q@\protect{\tt Q}!Q@\protect{-Q}}
\widedtdd{-Q \meta{opt}}{%
  Turn on an optimisation \meta{opt} from the list below.}
\widedtdd{-Q no-\meta{opt}}{%
  Turn off an optimisation \meta{opt} from the list below.}

\tabledtdd{}{}{Q0}{Q1}{Q2}{Q3\\ \hline}
\tabledtdd{-Q all}{All supported optimisations.}{}{}{}{X}
\tabledtdd{-Q inline}{Allow open coding of functions.}{}{}{X}{X}
\tabledtdd{-Q inline-all}{Allow open coding of any functions.}{}{}{}{X}
\tabledtdd{-Q inline-limit=\meta{n}}{Set maximum acceptable increase in code size from inlining to \meta{n}.}{1.0}{1.0}{5.0}{6.0}
\tabledtdd{-Q cfold       }{  Evaluate non floating point constants at compile time.    }{}{X}{X}{X}
\tabledtdd{-Q ffold       }{  Evaluate floating point constants at compile time.     }{}{ }{X}{X}
\tabledtdd{-Q hfold       }{  Determine, where possible, the run time hash code
which domains will have, reducing the number of domain instantiations
and speeding up cross-file look-ups.       }{}{X}{X}{X}
\tabledtdd{-Q peep        }{  Local ``peep-hole'' optimisations.     }{}{X}{X}{X}
\tabledtdd{-Q deadvar     }{  Eliminate unused variables and values. }{}{X}{X}{X}
\tabledtdd{-Q emerge      }{  Combine lexical levels and records.    }{}{ }{X}{X}
\tabledtdd{-Q cprop       }{  Copy propagation.                      }{}{ }{X}{X}
\tabledtdd{-Q cse         }{  Common sub-expression elimination.     }{}{ }{X}{X}

%\pagebreak
\tabledtdd{}{}{Q0}{Q1}{Q2}{Q3 \\ \hline}
\tabledtdd{-Q flow        }{  Simplify computed tests and jumps.     }{}{ }{X}{X}
\tabledtdd{-Q cast        }{  Reduce the number of casts.            }{}{ }{X}{X}
\tabledtdd{-Q cc          }{  Use the C compiler's optimiser.        }{}{ }{X}{X}
\index{inlining}
\index{procedural integration}
\index{constant folding}
\index{dead variable elimination}
\index{common subexpression elimination}
\index{peep hole optimisation}
\index{flow optimisation}
%\index{compiler options!Q@\protect{\tt Q}!Q@\protect{-Qall}}
%\index{compiler options!Q@\protect{\tt Q}!Q@\protect{-Qinline}}
%\index{compiler options!Q@\protect{\tt Q}!Q@\protect{-Qinline-all}}
%\index{compiler options!Q@\protect{\tt Q}!Q@\protect{-Qinline-limit}}
%\index{compiler options!Q@\protect{\tt Q}!Q@\protect{-Qcfold}}
%\index{compiler options!Q@\protect{\tt Q}!Q@\protect{-Qffold}}
%\index{compiler options!Q@\protect{\tt Q}!Q@\protect{-Qhfold}}
%\index{compiler options!Q@\protect{\tt Q}!Q@\protect{-Qpeep}}
%\index{compiler options!Q@\protect{\tt Q}!Q@\protect{-Qdeadvar}}
%\index{compiler options!Q@\protect{\tt Q}!Q@\protect{-Qemerge}}
%\index{compiler options!Q@\protect{\tt Q}!Q@\protect{-Qcprop}}
%\index{compiler options!Q@\protect{\tt Q}!Q@\protect{-Qcse}}
%\index{compiler options!Q@\protect{\tt Q}!Q@\protect{-Qflow}}
%\index{compiler options!Q@\protect{\tt Q}!Q@\protect{-Qcc}}

  Combinations can be used, \eg{} \ttin{-Q3 -Qno-ffold}.
  The option \ttin{-Qcc} may exclude \ttin{-Z} on some platforms.

% *********************************************************************
\head{section}{Debug options}{asugOptionsDebug}
% *********************************************************************

\widedtdd{-Z db}{%
  Generate debugging information in object files.}
\index{debugging}
%\index{compiler options!Z@\protect{\tt Z}!Z@\protect{-Zdb}}
\widedtdd{-Z prof}{%
  Generate profiling code in object files.}
\index{profiling}
%\index{compiler options!Z@\protect{\tt Z}!Z@\protect{-Zprof}}

% *********************************************************************
\head{section}{C code generation options}{asugOptionsC}
% *********************************************************************

Control the behaviour of \ttin{-F c}, the option for generating C code.
\index{C code generation}
\index{unicl}

%		-Cstandard

\widedtdd{-C standard}{%
  Generate ANSI/ISO standard C (the default). }
%\index{compiler options!C@\protect{\tt C}!C@\protect{-Cstandard}}

%		-Cold

\widedtdd{-C old}{%
  Generate old (Kernighan \& Ritchie) C.
  \par
  The options \ttin{-Cstandard} and \ttin{-Cold} are mutually exclusive.}
%\index{compiler options!C@\protect{\tt C}!C@\protect{-Cold}}

%		-Ccomp=<name>

\widedtdd{-C comp=\meta{name}}{%
  Use \meta{name} instead of the default C compiler driver \uniclcmd{}.
  Use \ttin{-v} to see how the driver is invoked.}
%\index{compiler options!C@\protect{\tt C}!C@\protect{-Ccomp}}

%		-Clink=<name>

\widedtdd{-C link=\meta{name}}{%
  Use \meta{name} instead of the default linker driver \uniclcmd{}.
  Use \ttin{-v} to see how the driver is invoked.}
%\index{compiler options!C@\protect{\tt C}!C@\protect{-Clink}}

%		-Ccc=<name>

\widedtdd{-C cc=\meta{name}}{%
  Use \meta{name} instead of the default C compiler and linker driver \uniclcmd{}.
  Without this option, the environment variable \ttin{CC} is tried.
  Use \ttin{-v} to see how the driver is invoked.}
\index{environment variables}
\index{CC}
%\index{compiler options!C@\protect{\tt C}!C@\protect{-Ccc}}

%		-Cgo=<name>

\widedtdd{-C go=\meta{name}}{%
  Use \meta{name} to run the output of the linker.
  Without \ttin{-Cgo=}, the environment variable \ttin{CGO} is tried.
  Use \ttin{-v} to see how the driver is invoked.
  Most implementations don't need this.}
\index{environment variables}
\index{CGO}
%\index{compiler options!C@\protect{\tt C}!C@\protect{-Cgo}}

%		-Cargs=<string>

\widedtdd{-C args=\meta{opts}}{%
  Pass \meta{opts} as options to the C compiler and linker.
  Use \ttin{-v} to see how the driver is invoked.}
%\index{compiler options!C@\protect{\tt C}!C@\protect{-Cargs}}
%%%%%%%%
% -Cfname does not work FIXME!
%%%%%%%%
%\widedtdd{-C fname=\meta{name}}{%
%  Use \meta{name} as the name of the generated file.
%  This is also used as the prefix for any external ids.}
%\index{compiler options!C@\protect{\tt C}!C@\protect{-Cfname}}

%		-Csmax=<n>

\widedtdd{-C smax=\meta{n}}{%
  Attempt to put no more than \meta{n} statements in each file,
  if necessary by splitting the generated file into: 
  \fname{name.h}, \fname{name.c}, \fname{name001.c}, \fname{name002.c}, etc.
  Using \ttin{-C smax=0} turns off file splitting.
  (default: \ttin{smax=2000})}
%\index{compiler options!C@\protect{\tt C}!C@\protect{-Csmax}}

%		-Cidlen=<n>

\widedtdd{-C idlen=\meta{n}}{%
  Set the maximum length of C identifiers to be \meta{n}.
  Using \ttin{-C idlen=0} turns off identifier truncation.
  (default: \ttin{idlen=32})}
%\index{compiler options!C@\protect{\tt C}!C@\protect{-Cidlen}}

%		-C[no-]idhash

\widedtdd{-C [no-]idhash}{%
  (Do not) use hash codes in global C identifiers.
  (default: \ttin{idhash})}
%\index{compiler options!C@\protect{\tt C}!C@\protect{-Cidhash}}
\index{name mangling}

%		-C[no-]lines

\widedtdd{-C [no-]lines}{%
  Preserve Aldor source line numbers in generated C.
  (default: no-lines)
}

%		-Csys=<id>

\widedtdd{-C sys=\meta{name}}{%
  Pass the option \ttin{-Wsys=\meta{name}} to the C compiler and linker.
  This option is interpreted by \uniclcmd{} to select a group
  of options defined in its configuration file \ttin{aldor.conf}.
  It also  prepends to the list of library directories a modified list
  in case there is a special implementation for the particular name.
  This is of great help when making libraries for specific flavours
  of CPU.
  
}

%               -Cruntime=<name>,<name>,..,<name>

\widedtdd{-C runtime=\meta{id1},..}{%
  Select a list of libraries \ttin{lib\meta{id1}.a}, \ttin{lib\meta{id2}.a} etc. 
  that implement the runtime system.
  Default is \ttin{foam}.
}

%               -Cfortran

\widedtdd{-C fortran}{%
  Pass \ttin{-Wfortran} to link driver \uniclcmd{} which enables fortran options
  such as fortran runtime libraries.
}

%               -Clib=<name>

\widedtdd{-C lib=\meta{name}}{%
  Link with the library \ttin{lib\meta{name}.a}
}

\vspace{5mm}
% *********************************************************************
\head{section}{Lisp code generation options}{asugOptionsLisp}
% *********************************************************************

\index{Lisp!code generation}
Control the behaviour of \ttin{-F lsp}.
\widedtdd{-S common}{%
  Produce Common Lisp code   (the default).}
\widedtdd{-S standard}{%
  Produce Standard Lisp code.}
\widedtdd{-S scheme}{%
  Produce Scheme code.
  \par
  The options \ttin{-Scommon}, \ttin{-Sstandard} and \ttin{-Sscheme}
  are mutually exclusive.}
%\index{compiler options!S@\protect{\tt S}!S@\protect{-Scommon}}
\index{Lisp}
\widedtdd{-S ftype=\meta{ft}}{%
  Use \meta{ft} as the file extension for the generated lisp file
  (default: \ttin{-Sftype=lsp}).}
%\index{compiler options!S@\protect{\tt S}!S@\protect{-Sftype}}

\vspace{5mm}
% *********************************************************************
\head{section}{Message options}{asugOptionsMessage}
% *********************************************************************

%Message options:  Use \ttin{-M no-\meta{word}} to negate \ttin{-M \meta{word}}.
\index{messages}

\widedtdd{-M emax=\meta{n}}{%
  Stop after \meta{n} error messages.        (default: 10)}
\index{message limit}
%\index{compiler options!M@\protect{\tt M}!M@\protect{-M emax}}
\widedtdd{-M db=\meta{fn}}{%
  Use \meta{fn} as the message database.     (default: no-db)}
%\index{compiler options!M@\protect{\tt M}!M@\protect{-Mdb} (set message database)}
\widedtdd{-M \meta{msgname}}{%
  Turn on warning or remark named \meta{msgname}.
  Use \ttin{-Mname} (described below) to find out the names of messages.}
%\index{compiler options!M@\protect{\tt M}!M@\protect{-M msgname}}
\widedtdd{-M no-\meta{msgname}}{%
  Turn off warning or remark named \meta{msgname}.}
\widedtdd{-M \meta{n}}{%
Control how much detail is given.     (default: \ttin{-M2})}
\widedtdd{-M no-\meta{opt}}{%
Turn of \ttin{-M \meta{opt}}.}

\tabledtdd{}{}{M0}{M1}{M2}{M3 \\ \hline}
\tabledtdd{-M warnings }{ Display warnings.                      }{N}{Y}{Y}{Y}
\tabledtdd{-M source   }{ Display the program lines for messages.}{N}{N}{Y}{Y}
\tabledtdd{-M details  }{ Display details.                       }{N}{N}{Y}{Y}
\tabledtdd{-M notes    }{ Display cross reference notes.         }{N}{N}{Y}{Y}
\tabledtdd{-M remarks  }{ Display remarks.                       }{N}{N}{N}{Y}
\tabledtdd{-M sort     }{ Sort messages by source position.      }{Y}{Y}{Y}{Y}
\tabledtdd{-M mactext  }{ Point to macro text, rather than use.  }{Y}{Y}{Y}{Y}
\tabledtdd{-M abbrev   }{ Abbreviate types in messages.          }{Y}{Y}{Y}{Y}
\tabledtdd{-M human    }{ Human-oriented format.                 }{Y}{Y}{Y}{Y}
\tabledtdd{-M name     }{ Show the name of each message as well as the message itself.}{N}{N}{N}{N}
\tabledtdd{-M antiques }{ Display warnings for old-style code.   }{N}{N}{N}{N}
\tabledtdd{-M preview  }{ Display messages as they occur.        }{N}{N}{N}{N}
\tabledtdd{-M inspect  }{ Use interactive inspector for errors.  }{N}{N}{N}{N}
%\index{compiler options!M@\protect{\tt M}!M@\protect{-M warnings}}
%\index{compiler options!M@\protect{\tt M}!M@\protect{-M source}}
%\index{compiler options!M@\protect{\tt M}!M@\protect{-M details}}
%\index{compiler options!M@\protect{\tt M}!M@\protect{-M notes}}
%\index{compiler options!M@\protect{\tt M}!M@\protect{-M remarks}}
%\index{compiler options!M@\protect{\tt M}!M@\protect{-M sort}}
%\index{compiler options!M@\protect{\tt M}!M@\protect{-M mactext}}
%\index{compiler options!M@\protect{\tt M}!M@\protect{-M abbrev}}
%\index{compiler options!M@\protect{\tt M}!M@\protect{-M human}}
%\index{compiler options!M@\protect{\tt M}!M@\protect{-M name}}
%\index{compiler options!M@\protect{\tt M}!M@\protect{-M antiques}}
%\index{compiler options!M@\protect{\tt M}!M@\protect{-M preview}}
%\index{compiler options!M@\protect{\tt M}!M@\protect{-M inspect}}

% *********************************************************************
\head{section}{Developer options}{asugOptionsDeveloper}
% *********************************************************************

Developer options (subject to change):

\widedtdd{-W check}{%
  Turn on internal safety checks.}
%\index{compiler options!W@\protect{\tt W}!W@\protect{-W check}}
\widedtdd{-W runtime}{%
  Produce code suitable for the runtime system.}
%\index{compiler options!W@\protect{\tt W}!W@\protect{-W runtime}}
\widedtdd{-W nhash}{%
  Assume all exported types have constant hashcodes.}
%\index{compiler options!W@\protect{\tt W}!W@\protect{-W nhash}}
\widedtdd{-W loops}{%
  Always inline generators when possible.}
%\index{compiler options!W@\protect{\tt W}!W@\protect{-W loops}}
\widedtdd{-W missing-ok}{%
  Do not stop the compiler if an export is missing in a domain.}
\index{missing exports}
%\index{compiler options!W@\protect{\tt W}!W@\protect{-W missing-ok}}
\widedtdd{-W audit}{%
  Set maximum Foam auditing level.}
%\index{compiler options!W@\protect{\tt W}!W@\protect{-W audit}}
\widedtdd{-W trap}{%
  Trap failure exits (for debugging \asharp{}).}
\index{debugging}
%\index{compiler options!W@\protect{\tt W}!W@\protect{-W trap}}

\widedtdd{-W gc}{%
  Garbage collect as needed (if gc is available).}
\index{garbage collection}
%\index{compiler options!W@\protect{\tt W}!W@\protect{-W gc}}
\widedtdd{-W gcfile}{%
  Garbage collect after each file (if gc is available).}
%\index{compiler options!W@\protect{\tt W}!W@\protect{-W gcfile}}

\widedtdd{-W sexpr}{%
  Run a read-print loop.}
%\index{compiler options!W@\protect{\tt W}!W@\protect{-W sexpr}}
\widedtdd{-W seval}{%
  Run a read-eval-print loop.}
%\index{compiler options!W@\protect{\tt W}!W@\protect{-W seval}}

\widedtdd{-W test=\meta{name}}{%
  Run compiler self-test \meta{name}.
  Use the option \ttin{-W test+show} to see a list of self-tests.}
%\index{compiler options!W@\protect{\tt W}!W@\protect{-W test}}
\widedtdd{-W D+\meta{name}}{%
  Turn on debug hook \meta{name}.
  Use \ttin{-W D+show} to see a list of debug hooks.}
%\index{compiler options!W@\protect{\tt W}!W@\protect{-W D+}}
\widedtdd{-W D-\meta{name}}{%
  Turn off debug hook \meta{name}.
  Use \ttin{-W D+show} to see a list of debug hooks.}
%\index{compiler options!W@\protect{\tt W}!W@\protect{-W D-}}

\widedtdd{-W T{apdrgst0}+\meta{ph}}{%
  Phase tracing:
  Several phases can be given as \ttin{ph1+ph2+ph3} or \ttin{all}.
  Several options can be given at once, \eg{} \ttin{-WTags+all}.
  Compile a file with \ttin{-v} to see the phase abbreviations.
  The remaining options give more details about the trace code letters.}
%\index{compiler options!W@\protect{\tt W}!W@\protect{-W T+}}

\widedtdd{-WTa+\meta{ph}} {Announce entry to \meta{ph}.}
\widedtdd{-WTp+\meta{ph}} {Pretty print result of \meta{ph}.}
\widedtdd{-WTd+\meta{ph}} {Print debug information for \meta{ph}.}
\widedtdd{-WTr+\meta{ph}} {Show result of \meta{ph}.}
\widedtdd{-WTg+\meta{ph}} {Garbage collect after \meta{ph}.}
\widedtdd{-WTs+\meta{ph}} {Storage audit after \meta{ph}.}
\widedtdd{-WTt+\meta{ph}} {Terminate after \meta{ph}.}
\widedtdd{-WT0+\meta{ph}} {Ignore earlier \ttin{-WT} option for \meta{ph}.}

% *********************************************************************
\head{widesection}{Environment variables}{asugOptionsEnvironment}
% *********************************************************************

\index{environment variables}
Sometimes there are certain aspects of the compiler's behaviour
which you may wish to change for most of your compilations.

Most operating systems have some notion of environment variables
and the \asharp{} compiler checks a number of these to guide its actions.
None of these needs to be defined, but if they are the  \asharp{} compiler
will use them.

\widedtdd{\tt ALDORROOT}{%
  This gives a directory under which the compiler will find its own
  include files, libraries, {\em etc.}}
\index{environment variables}
\index{ALDORROOT}
\widedtdd{\tt ALDORARGS}{%
  This provides options to the compiler which are treated before those
  appearing on the command line.}
\index{environment variables}
\index{ALDORARGS}

\widedtdd{\tt INCPATH}{%
  This gives the compiler additional places to search for include files.
  The syntax is according to the operating system's conventions.
  For example, on Unix a suitable initialisation could be:
\index{environment variables}
\index{INCPATH}
\par\noindent
{\small\tt INCPATH=/home/jane/include:.:/usr/local/include}
\par\noindent
This has the same effect as using the command line options
\par\noindent
{\small\tt  -I/home/jane/include -I. -I/usr/local/include}}

\widedtdd{\tt LIBPATH}{%
  This gives the compiler additional places to search for libraries.}
\index{environment variables}
\index{LIBPATH}
\widedtdd{\tt CC}{%
  This gives the \asharp{} compiler the name of a C compiler which
  it should call to generate and link object code.
  The default is \uniclcmd{} -- a program provided in the \asharp{}
  distribution to convert C compile command lines to the native system syntax.}
\index{environment variables}
\index{CC}
\widedtdd{\tt CGO}{%
  This gives the \asharp{} compiler the name of a loader which it should
  call to run an executable program it has generated.
  On most platforms no explicit loader is needed.
\index{environment variables}
\index{CGO}

  Sometimes you will wish to specify \ttin{CC} and \ttin{CGO} together.
  For example on DOS using the \ttin{djgpp} port of GCC,
  a useful combination is
  \par\noindent
  {\tt set CC=gcc}
  \par\noindent
  {\tt set CGO=go32}}
\index{gcc}
\index{djgpp}
