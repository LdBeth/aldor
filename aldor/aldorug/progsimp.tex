%%%%%%%%%%%%%%%%%%%%%%%%%%%%%%%%%%%%%%%%%%%%%%%%%%%%%%%%%%%%%%%%%%%%%%%%%%%%%%
%%%%%%%%%%%%%%%%%%%%%%%%%%%%%%%%%%%%%%%%%%%%%%%%%%%%%%%%%%%%%%%%%%%%%%%%%%%%%%
%%%                                                                        %%%
%%% :: Introductory Programs
%%%                                                                        %%%
%%%%%%%%%%%%%%%%%%%%%%%%%%%%%%%%%%%%%%%%%%%%%%%%%%%%%%%%%%%%%%%%%%%%%%%%%%%%%%
%%%%%%%%%%%%%%%%%%%%%%%%%%%%%%%%%%%%%%%%%%%%%%%%%%%%%%%%%%%%%%%%%%%%%%%%%%%%%%
\head{chapter}{Some simple programs}{asugLangSimple}

% Perhaps the easiest way to get a feeling for a programming language is
% to read a few programs.  This % section
% chapter presents simple programs
% and uses them as a departure point to discuss some of the basic ideas
% of \asharp{}.

Perhaps the easiest way to get a feeling for a programming language is
to read a few programs.  This chapter presents simple programs
and uses them as a departure point to discuss some of the basic ideas
of \asharp{}.

Two main options are available to readers after completing this
chapter.  Those who prefer a structured approach may choose to
progress through the development in part II.  Those preferring to
learn by example may want to skip ahead to \chapref{asugSample},
where they will find extended examples of more advanced
programming techniques in the form of further annotated programs,
and refer to part II only as necessary.

%%%%%%%%%%%%%%%%%%%%%%%%%%%%%%%%%%%%%%%%%%%%%%%%%%%%%%%%%%%%%%%%%%%%%%%%%%%%%%
%\newpage
\head{section}{Doubling integers}{asugLangSimpleI}
\thegeneralfile{h}{SimpleProgI}
{Simple program 1}{examples/simple1.as}

The first program is one which doubles integers.
%shown in Figure~\ref{SimpleProgI}.
This program illustrates a number of things:
\begin{enumerate}
\item
The \asharp{} language is itself almost empty. 
This allows libraries to define their own environments all the
way down to such basic questions such as what an integer ought to be.
Therefore, almost all programs begin with an \ttin{\#include} line
to establish a basic context.
The \ttin{\#include} line in this example provides a context in
which \ttin{Integer} has a specific meaning, provided by the stand-alone
\libaldor{} library. Actually the identifier \ttin{Integer} is a convenient
\ttin{macro} (see \chapref{asugLangMacros}) for \ttin{AldorInteger}.
Please refer to the \libaldor{} library documentation for more details
about \ttin{Integer} and \ttin{AldorInteger}.
%To allow the standard libraries to be used with a
%user's own libraries without name-clashes, the types in \libaldor{} are
%prefixed by a unique identifier (in this case {\tt BL}).
%To avoid these
%names and make the program more readable a macro file is automatically
%included to allow this prefix to be ignored.
\item The symbol \ttin{==} indicates a definition --- in this case
a definition of a function named \ttin{double}.
\keywordIndex{==}
\item
The function has two declarations using the syntax
\ttin{:~Integer}.  
Names indicating values (variables, parameters, \etc{})
may each contain values of only a specific {\em type}\index{type}.  
The first declaration in this program
states that the parameter \verb"n" must be an \verb"Integer". 
The second asserts that the result of the function will
also be an \verb"Integer".  (The type of the function itself is
represented as \ttin{Integer -> Integer}; a name and type together
are called a {\em signature}\index{signature}, as in
\ttin{double: Integer -> Integer}.)
\item
The declarations cause the exports from the type \verb"Integer"
to be visible.  Typically, a type exports special
values (such as 0 and 1) and functions on its members.
In this example, the name \ttin{+} has a meaning as an exported
function from \verb"Integer".
\item
The body of the function \verb"double" is the expression
\ttin{n + n}.  
The value of this expression is returned as the result
of the function.  
It is not necessary to use an explicit \ttin{return}
statement, although it is permitted.
This turns out to be very convenient when many functions have very
short definitions, as is normal with abstract data types or
object-oriented programs.
\end{enumerate}

%%%%%%%%%%%%%%%%%%%%%%%%%%%%%%%%%%%%%%%%%%%%%%%%%%%%%%%%%%%%%%%%%%%%%%%%%%%%%%
%\newpage
\head{section}{Square roots}{asugLangSimpleII}
\thegeneralfile{h}{SimpleProgII}
{Simple program 2}{examples/simple2.as}

Our second program 
%is given in Figure~\ref{SimpleProgII} and
illustrates several more aspects of the language:

\begin{enumerate}
\item Comments begin with two hyphens \ttin{--} and continue to
      the end of the line.
\item Abbreviations (``macros'') may be defined using \ttin{==>}.
\keywordIndex{==>}
    The line

\begin{verbatim}
    DF ==> DoubleFloat;
\end{verbatim}

    causes \ttin{DF} to be replaced by \ttin{DoubleFloat} wherever it is used.
%\item Functions can accept multiple arguments and return multiple results.
%      In this example, the function \verb"sums" accepts three parameters
%      and produces three results, all of which are \verb"Integer"s.

\item A function's body may be a compound expression.  In this example
      the body of the function is a sequence consisting of eight
      expressions separated by semicolons and grouped together by braces.
      These expressions are evaluated in the order given.
      The  value of the last expression is the value of the sequence,
      and hence is the value of the function.

\item The semicolons separate expressions.  It is permitted, but not
      necessary, to have one after the last expression in a sequence.

\item Variables may be assigned values using \ttin{:=}.
\keywordIndex{:=}

\item The variable \ttin{r} is local to the function
      \verb"miniSqrt": it will not be seen from outside it.
      Variables may be made local to a function by a \ttin{local}
      declaration or, as in this case, implicitly, by assignment.

\item In this function the variable \ttin{r}
      contains double precision floating point values.
      Since this may be inferred from the program,
      it is not necessary to provide a type declaration.
\end{enumerate}

%%%%%%%%%%%%%%%%%%%%%%%%%%%%%%%%%%%%%%%%%%%%%%%%%%%%%%%%%%%%%%%%%%%%%%%%%%%%%%
%\newpage
\head{section}{A loop and output}{asugLangSimpleIII}
\thegeneralfile{h}{SimpleProgIII}
{Simple program 3}{examples/simple3.as}

The third program
%, shown in Figure~\ref{SimpleProgIII},
has a loop and produces some output.  Things to notice about
this program are:
\begin{enumerate}

\item This example has expressions which occur at the top-level,
outside any function definition.  This illustrates how the entire source
program is treated as an expression sequence, which may (or may not) contain
definitions among other things.  
This entire source program is treated in the same way as a compound
expression forming the body of a function:  it is evaluated from 
top to bottom, performing definitions, assignments and function calls
along the way.

\item As we saw in a previous example,
the declarations \ttin{:~Integer} suffice
to make the exports from \verb"Integer" visible within the
\verb"factorial" function.
This gives meaning to \ttin{1}, \ttin{*} and \ttin{..}.

These declarations do not, however, cause 
the exports from \verb"Integer" to be visible at the top-level
of the file, outside the function \verb"factorial".
%Word order changed to avoid overfull box.
This conservative behaviour turns out to be quite desirable when
writing large programs, since adding a new function to a working
program will not pollute the name space of the program into which it
is inserted.

In order to be able to use the exports of \verb"Integer" at the 
top-level of the file, we have used an \ttin{import} statement.
Importing \verb+TextWriter+ allows the use of \verb+stdout+,
\verb+String+ is needed for \verb+"factorial 10 = "+ and
\verb+Character+ allows the use of \verb+newline+.

\item The last line of the example produces some output.
The general idea is to use the infixed name \ttin{<<} to
output the right-hand expression via the left-hand text writer.

Syntactically, \ttin{<<} groups from left to right so a minimum number
of parentheses are needed:  the line in the example is equivalent to

\begin{small}
\begin{verbatim}
((stdout << "factorial 10 = ") << factorial 10) << newline;
\end{verbatim}
\end{small}

\item The last line is simple but refers to many things.
We shall say exactly where each of them comes from:
\begin{itemize}
\item The name \ttin{stdout} is a \verb"TextWriter".
\item The expression \ttin{"factorial 10 ="} is a \verb"String" constant.
\item \ttin{newline} is a \verb"Character".
\end{itemize}
All the above are visible by virtue of the \ttin{\#include "aldorio"} statement
at the beginning of the example.
\begin{itemize}
\item We have already seen where \ttin{factorial} and \ttin{10} come from.
\item This leaves \ttin{<<}.  
There are three uses, and each use refers to a different function.  
The first one takes a string as its right (second) argument and comes
from the \verb+#include "aldorio"+ statement. 
The second one takes an \verb"Integer" and is imported along with
the other operations in with the \ttin{:~Integer} declaration.
The last one takes a \verb"Character" and also comes from
the \verb+#include "aldorio"+ statement.
\end{itemize}

%%FIXME: Describe formatted output
%It is also possible to produce formatted output, and this is described
%in \chapref{asugBaselibDoms}.

\item Let us look more closely at the use of the \verb"factorial" function
in the last line: \ttin{factorial 10}.
No parentheses are needed here because the function has only a single argument.
If the function took two arguments, \eg{} \ttin{5, 5},
or if the argument were a more complicated expression, \eg{} \ttin{5 + 5},
then parentheses would be needed to force the desired grouping.

\item A word of caution is necessary here: the manner of output is defined
by the particular library, not the language.  
The form of output in this example is appropriate when using
\ttin{\#include "aldorio"} but may not work in other contexts.
%but {\em will not work\/} when using \ttin{\#include "axiom"}.
%(In the \axiom{} library, output is done by coercion to type
%{\tt OutputForm}).
\end{enumerate}

The difference between \verb+#include "aldor"+ and \verb+#include "aldorio"+
might not be clear yet, here is an
explanation. \verb+#include "aldor"+ is the general include statement
that will allow the types and functions from \verb+libaldor+ to be
visible. It is necessary to use it (or an equivalent library) in order
to write programs. \verb+#include "aldorio"+ is merely a convenience
which will automatically import types used for input/output (and, in
particular, printing values to the standard output).

\clearpage
%%%%%%%%%%%%%%%%%%%%%%%%%%%%%%%%%%%%%%%%%%%%%%%%%%%%%%%%%%%%%%%%%%%%%%%%%%%%%%
\head{section}{Forming a type}{asugLangSimpleIV}\index{type}
\thegeneralfile{h}{SimpleProgIVa}
{Simple program 4 --- Skeleton}{examples/simple4a.as}

The fourth example shows {\tt MiniList}, a type-constructing function.

%% I wonder if this example isn't a bit heavyweight for this stage of
%% the development.  I have added a forward reference for "Generator S"
%% and for the function "generate", which I've also mentioned explicitly
%% (otherwise its use in defining "generator" looks pretty circular).
%% A lot of other things in the example are still unexplained at this
%% point - for example:
%%
%%      Why does the definition of "cons" need a "per"?
%%
%%      What does the "union" in this definition do?  How is the user
%%      to find the answers to such questions?
%%
%%      How do the definitions of first and rest work?  (The "dot" is
%%      undefined, as yet - it's meaning can, perhaps, be guessed - but
%%      what happens if "l" is null?)
%%
%% It seems a bit pious to say "This program should be easy to
%% understand..."
%%
%% MGR

We will defer showing the body of the function for a moment,
until we have had a first look at the definition.

The {\tt MiniList} function will provide a simple list data type
constructor, allowing lists to be formed with brackets, for example
\verb"[a, b, c]".  All the elements of the list must belong to the same
type, \verb"S", the parameter to the function.  

This is a simple example of how one might use \verb"MiniList":

\begin{small}
\begin{verbatim}
MI ==> MachineInteger;
square(n: MI): MI == {
        sql: MiniList MI := [1, 4, 9, 16, 25];
        if n < 1 or n > 5 then error "Value out of range";
        sql.n  -- the (n)th component of sql
}
\end{verbatim}
\end{small}

The definition of \verb"MiniList" is just like the definitions
we have seen in the previous examples:
\begin{itemize}
\item
  {\tt MiniList} accepts a parameter, \verb"S", which is declared to belong
  to a particular type --- in this case \ttin{OutputType}.
\item
  {\tt MiniList} returns a result which is declared to belong to
  a particular type --- in this case to the type \ttin{MiniListType(S)}.
\item
  The body of the function {\tt MiniList} is an expression
  to compute the return value, given a value for the parameter. 
\end{itemize}

The name \verb"OutputType" is meaningful by virtue of the
``\verb+#include+'' line.

\verb"OutputType" is a type whose members are themselves types.
The same is true for \verb"MiniListType(S)".
A type such as this, whose members are types, will be called a
{\em type category}, or simply a {\em category} where no confusion
can arise.
\index{Category}

What we have seen implies that \verb"MiniList" is a function
which takes a type parameter and computes a new type as its value.
%The next step is to ask why the arguments
%are declared as being {\em different} kinds of types.
% I don't understand the above sentence - Minilist only has one argument
% so what are the "arguments" referred to?  The exposition is clearer
% without it.  MGR
%% PI - Last sentence should be:
%% Why the parameter of {\tt MiniList} is declared to be of type 
%% \ttin{BasicType}? The parameter type determines the set of operations 
%% available for the parameter. In other words, it provides
%% informations about the nature of parameter. We'll clearify this
%% concept later.

Now that we have had the bird's eye view, 
it is time to take a second look at the function.   
The complete definition is given in Figure~\ref{SimpleProgIVb}.
\thegeneralfile{t}{SimpleProgIVb}
{Simple program 4 --- Details}{examples/simple4b.as}
\clearpage
A few points will help in understanding this program:
\begin{enumerate}
\item 

The first thing to note is that the new type constructor is defined as
a function \verb"MiniList" whose body is an \ttin{add} expression,
itself containing several function definitions.  It is these internal
functions of an \ttin{add} function which provide the means to
manipulate values belonging to the resulting types (such as
\verb"MiniList(Integer)" in this case).

\item
This program uses various names we have not seen before,
for example \ttin{Record}, \ttin{Pointer}, \ttin{element}, \etc{}
Some of these, such as \ttin{Pointer},
are made visible by the \verb"#include" lines,
while others, such as \ttin{element}, are made visible by
declaring values to belong to particular types.

The names which have meanings in \libaldor{} are detailed in
the documentation for that library.

\item
While \verb"OutputType" and \verb"MiniListType(S)" are both
type categories, the types in each category have different characteristics.  
The difference lies in what exports their types must provide:

\begin{itemize}
\item
Every type which is a \verb"OutputType" must supply an
output function (\ttin{<<}).

Since \verb"S" is declared to be a \verb"OutputType",
the implementation of \verb"MiniList" can use
the \ttin{<<} from \verb"S"
in the definitions of \verb"MiniList"' own operations.
%% PI, last sentence: Since \verb"S" is declared to be a
%% \verb"BasicType", the implementation of \verb"MiniList" can use
%% equality and output, \ie{} the \ttin{=} and \ttin{<<} operators, 
%% on elements of \verb"S" in the definitions of its own operations.

\item
Every type which is a \verb"MiniListType(S)" must supply 
several other operations, such as a constructor function called \ttin{bracket}
to form new values, a test function called \ttin{empty?}, and so on.
\verb"MiniList(S)" provides a \verb"MiniListType(S)", so
% Avoiding overfull box again!
it must supply all these operations.  
Users of \verb"MiniList(S)"
will be able to rely on these operations being available.
\end{itemize}

\item 
The first line of  the \ttin{add}
expression defines a type \ttin{Rep} (specific to \ttin{MiniList}).
This is how values of the type being defined are really represented.
The fact that they are represented this way is not visible outside
the \ttin{add} expression.
%% PI, last two sentences: When a new type is created, its internal 
%% representation must be supplied. The elements belonging to the new
%% domain will be represented using previously existing domains. This
%% representation is not visible outside the \ttin{add} expression,
%% \ie{} is {\em encapsulated}, according to the {\em Abstract
%% Data Type} paradigm. \ttin{Rep} is a special type that must be
%% defined inside the \ttin{add} body to specify the representation
%% that must be adopted for the new domain.

\item
Several of the functions defined in the body have parameters or
results declared to be of type \ttin{\%}.
In any \ttin{add} expression, the name \ttin{\%} refers to the type
under construction.  
For now, \ttin{\%} can be thought of as a shorthand for \ttin{MiniList(S)}.

\item
There are several uses of the operations \ttin{per} and \ttin{rep}
in this program.  
These are conversions which allow a data value to be regarded
in is public guise, as a member of \ttin{\%}, or by its private
representation as a member of \ttin{Rep}.

\Export{rep}{\% -> Rep}{}
\ExportToo{per}{Rep -> \%}{}

These can be remembered by the types they produce:  \ttin{rep} produces
a value in \verb"Rep", the {\em rep\/}resentation,
and \ttin{per} produces a value in \verb"%", {\em per\/}cent.

\item
Some of the function definitions are preceded by the word \ttin{local}.
This means they will be private to the \ttin{add}, and consequently
will not be available to users of \verb"MiniList".

\item
Some of the definitions have left-hand sides with an unusual syntax:

\begin{verbatim}
    [t: Tuple S]: % == ...
    [g: Generator S]: % == ...
    (out: TextWriter) << (l: %): TextWriter == ...
\end{verbatim}

In general, the left-hand sides of function definitions in \asharp{}
look like function calls with added type declarations.
%% PI, last sentence: In general, function {\em definitions} in \asharp{}
%% look like function calls with added type declarations\footnote{Note that
%% this is not true when functions are {\em declared}, as we'll see
%% later.}.
Some names have infix syntax, for instance \ttin{<<}
above.  These are nevertheless just names and, aside from the grouping,
behave in exactly the same way as other names.
The special syntactic properties of names may be avoided by
enclosing them in parentheses.
Other special syntactic forms
are really just a nice way to write certain function calls.
The form \ttin{[a,b,c]} is competely equivalent to the call
\ttin{bracket(a,b,c)}.

With this explanation, we see that the defining forms above are
equivalent to the following, more orthodox forms:
%% PI doesn't like the word `familiar'; MGR is inclined to agree.

\begin{verbatim}
    bracket(t: Tuple S): % == ...
    bracket(g: Generator S): % == ...
    (<<)(out: TextWriter, l: %): TextWriter == ...
\end{verbatim}

\item
The use of the type \ttin{Generator S} is explained fully in
\chapref{asugLangGener}, so for now we will only provide a brief
overview.

The function \ttin{generator} illustrates how a type can
define its own {\em traversal method}, which allows the new type to 
decide how its component values should be obtained, say for use in a
loop.  Such a definition utilises the function \ttin{generate}, in 
conjuction with \ttin{yield}: each time a \ttin{yield} is encountered,
\ttin{generate} makes the given value available to the caller and then
suspends itself.  This technique is described more fully in 
\secref{asugLangGenerCreating}.
When the next value is needed, the generator resumes where it left off.

Since 
\verb"MiniList(S)" implements a \ttin{generator} function for
objects of type \verb"MiniList",
it is possible to iterate over them in a \ttin{for} loop.  For
example, in the output function \ttin{<<}, we see the line

\begin{verbatim}
    for s in rest l repeat out << ", " << s;
\end{verbatim}

Here \ttin{rest l} is traversed by the \verb"generator" to obtain
values for \ttin{s}.

%% PI, change example with a simpler one: For example, if \ttin{l} is
%% of type \verb"MiniList(S)", in your program you can write:
%%
%% 	for elem in l repeat
%%    		stdout << elem;
%%
%% \asharp{} is smart enough to understand that, since \ttin{l}
%% appears in a \ttin{for} loop, we want to traverse it, so implicit
%% applications of the \ttin{generator} function happen. As we'll see
%% later, there are other situations in which implicit call to
%% \ttin{generator} happen. In general, you'll never see explicit call
%% to \ttin{generator} in your programs.

\end{enumerate}

\clearpage
%%%%%%%%%%%%%%%%%%%%%%%%%%%%%%%%%%%%%%%%%%%%%%%%%%%%%%%%%%%%%%%%%%%%%%%%%%%%%%
\head{section}{Continuing ...}{asugLangSimpleContinuing}

These first examples already give a fairly good start at reading
\asharp{} programs.  The reader is now equipped to understand
most \asharp{} programs, at least in broad terms.
For those who wish to understand the language in more detail,
Part II presents further aspects of the language and
revisits what we have already seen in more depth.

At this point, readers may decide to continue by studying the examples
given in \chapref{asugSample}, or by trying examples of their own.
%Some may wish to skip to \chapref{asugNovice} to see how someone
%else went about learning to use \asharp{}.  
%Others may wish to read \chapref{asugFNotes},
%in which a medium-sized example is developed step-by-step.
