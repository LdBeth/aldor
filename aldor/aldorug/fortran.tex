%%%%%%%%%%%%%%%%%%%%%%%%%%%%%%%%%%%%%%%%%%%%%%%%%%%%%%%%%%%%%%%%%%%%%%%%%%%%%%
%%%                                                                        %%%
%%% :: Fortran Interoperability                                            
%%%                                                                        %%%
%%%%%%%%%%%%%%%%%%%%%%%%%%%%%%%%%%%%%%%%%%%%%%%%%%%%%%%%%%%%%%%%%%%%%%%%%%%%%%
%%%%%%%%%%%%%%%%%%%%%%%%%%%%%%%%%%%%%%%%%%%%%%%%%%%%%%%%%%%%%%%%%%%%%%%%%%%%%%
\head{chapter}{Using \asharp{} with Fortran-77}{asugUsingWithFortran}
\newcommand{\Cpp}{C{\tt ++}}
\index{Fortran!{\em see \chapref{asugUsingWithFortran}}}

\head{section}{Basics}{asugFortBasic}

This section describes how to call subprograms written in Fortran-77
from \asharp, and how to call routines written in \asharp{} from
Fortran-77.   Since there is no standard foreign-language interface to
Fortran-77 it may be necessary to customise your implementation of
\asharp{} to work with your local Fortran compiler.  As is the case
with \asharp{} programs which import other foreign code, programs
which use Fortran cannot be run in the interpreter environment.

The current interface supports all of the data types in Fortran-77,
and allows Fortran and \asharp{} functions to be used interchangeably in
many contexts.  For example a Fortran function can be treated as an
\asharp{} function transparently, and an \asharp{} routine may behave
like a Fortran Function or Subroutine.  The only major restriction is
that an \asharp{} program cannot see a Fortran Common block, and a
Fortran program has no way of accessing \asharp{} global variables.

Note that this mechanism uses the C generation facilities described 
in \chapref{asugUsingWithC} and does not provide a mechanism for
generating native Fortran code.  It is principally intended as a means
by which an \asharp{} programmer can make use of the vast body of Fortran
code available, and by which a Fortran programmer may embed \asharp{}
code in his or her application.

\head{section}{Simple Example}{asugFortSimpleEx}

This simple example demonstrates the main concepts you need to know to
call Fortran from \asharp{}.  It shows the use of a Fortran routine
for sorting all or part of a vector of floating-point numbers.  

\program{f77sort}

\begin{description}
\lineNote{6--8} Here we import the function from
\verb"Foreign(Fortran)".  We will describe the exact correspondence
between Aldor and Fortran types later, but for the moment remark that
objects whose value will be changed on exit are passed using the
\verb`Ref` constructor.
\lineNote{20} The Fortran routine is called as if it were an Aldor
one.  The operator \verb`ref` is used to pass a reference to the error
flag (this is described in more detail below).  Notice that it is
possible to pass numerical data directly (the second argument, which
in this case represents the start point of the segment of the vector
to be sorted) as well as via a variable.
\lineNote{24} Note that the contents of the array have been changed by
the Fortran routine.
\end{description}

The program could be compiled as follows, assuming that the
\verb`fsort` routine is contained in a library called \verb`sort`.  

\begin{small}
\osprompt\ \asharpcmd{ -Cfortran -Fx -laldor -Clib=sort f77sort.as}
\end{small}

Note the use of the \option{-Cfortran} flag.  This has the effect of
causing the linker to link to the appropriate Fortran runtime
routines, and may also cause compiler-specific initialisation code to
be generated.  It is not necessary to use this flag except at the link
stage. 

\head{section}{Data Correspondence}{asugFortData}

Fortran-77 has a fixed and relatively small set of data types, and
passes all subprogram parameters by reference (i.e.~it passes a
pointer to the data rather than a copy of the data).  \asharp, on the
other hand, has a rich and extensible type system, and in general will
pass copies of subprogram data (at least in simple cases).   The aim
of the interface is to ensure that Foreign functions behave naturally
in their host environment.

In practice it would be very restrictive only to be allowed to pass this
fixed list of types to Fortran, so \asharp{} uses a set of categories
to indicate that a domain can be used to pass particular types of
values.  This usually means that the domain's representation is the
appropriate basic machine type or a pointer to it.  For example the
{\tt DoubleFloat} type belongs to the {\tt FortranDouble} category since
its representation is a boxed double precision floating point number.
Note that none of these categories have any exports, the complete list,
and an example of an \asharp{} type which belongs to each one, is:

\begin{center}
{\small
\begin{tabular}{|l|l|l|}\hline
& & \\
{\bf Aldor Category}      & {\bf Fortran Type} & {\bf Example Domain}\\
& & \\
\hline
& & \\
\verb`FortranInteger`       & INTEGER          & \verb`MachineInteger`  \\ 
\verb`FortranReal`          & REAL             & \verb`SingleFloat`\\
\verb`FortranDouble`        & DOUBLE PRECISION & \verb`DoubleFloat`\\
\verb`FortranLogical`       & LOGICAL          & \verb`Boolean`\\
\verb`FortranCharacter`     & CHARACTER        & \verb`Character`\\
\verb`FortranString`        & CHARACTER(*)     & \verb`String`\\
\verb`FortranFString`       & CHARACTER(*)     & \verb`FixedString`\\
\verb`FortranComplexReal`   & COMPLEX REAL     & \verb`Complex(SingleFloat)`\\
\verb`FortranComplexDouble` & COMPLEX DOUBLE   & \verb`Complex(DoubleFloat)`\\
& & \\
\hline
\end{tabular}
}
\end{center}

Fortran strings (or rather Character arrays) are not null-terminated,
so to manipulate them it is necessary to know their length.  Data from
the \asharp{} domain \verb'String' is automatically converted to the
equivalent Fortran object, which is in principle a (length, data)
pair.  An alternative is to use the {\tt FixedString} type which is
parameterised by its length. 

By default \asharp{} passes a copy of a scalar parameter to Fortran,
in line with its usual semantics.  Since many Fortran routines return
results by modifying their arguments there is also a \verb'Ref'
constructor which will in effect copy the value of the parameter at
the end of the call to the Fortran routine back to the \asharp{}
object (this is often referred to as ``copy-in/copy-out'' semantics).
The choice of whether to declare an argument to a Fortran routine to
be e.g.~a \verb'DoubleFloat' or a \verb'Ref(DoubleFloat)' is left up
to the user, and will depend on whether he or she wishes to inspect
its value after the call to Fortran has been completed.  Note that it
is perfectly safe to declare an argument as e.g.~\verb'DoubleFloat'
even if it will be modified by Fortran, although of course the
modified value will not be visible in \asharp.  The \verb'Ref' domain
exports two operations to dereference and update instances of itself.

\head{subsection}{Array Arguments}{asugFortDataArr}

The normal layout of data in an \asharp{} array is not suitable for
direct use by a Fortran routine because it may contain pointers to 
its elements, or indeed be a vector of pointers to its rows etc.  There
is also the important fact that Fortran lays out its arrays in column
order whereas languages like \asharp{} and C tend to use row order.

Supposing that we want to pass an array of objects of domain {\tt T}
to Fortran.  It simplifies matters greatly if {\tt T} satisfies
\verb'DenseStorageCategory'.  This means that each element of the
array can be stored in a fixed amount of memory and so the array can
be constructed as a sequence of objects rather than a sequence of
pointers to objects.  By default the compiler will try and determine 
this automatically, but it can be done by hand to improve efficiency.

For an array-like object to be passed to Fortran, it must satisfy either
{\tt FortranArray} or {\tt FortranMultiArray} category, depending on
whether it is a single-dimensional or multi-dimensional object.  These
provide exports for converting between the \asharp{} and Fortran 
representations, and are {\em automatically\/} applied by the compiler.
Domains in the standard library which can be used in this way include
{\tt PrimitiveArray}, {\tt Array} and {\tt TwoDimensionalArray}.

There are two special cases to these rules.  The first concerns arrays
of strings, in \asharp{} we need to ensure that we use a fixed sized
string representation, and that the array type satisfies the category
{\tt Fortran\-F\-String\-Array}.  An example of a domain which can be used
in this way is {\tt Array Fixed\-String(10)}.

The second special case arises when we are defining an Aldor function
to be passed to (and called from) Fortran.  Here we are forced to use
{\tt PrimitiveArray} for all array types since it is the Fortran runtime
environment rather than \asharp{}'s which will set-up the call.  In
practice this is not too inconvenient.


\head{subsection}{Passing Subprograms as Arguments}{}

The use of Fortran {\tt Function} and {\tt Subroutine} arguments is
supported.  For example:

\begin{verbatim}
DF ==> DoubleFloat;
import {
        integrate: (DF -> DF, hi: DF, lo: DF) -> DF;
} from Foreign Fortran;

-- integrate the sin function between 0 and 1
integrate(sin, 0.0, 1.0);

-- integrate the cos(fn) function between 0 and 1
foo(fn: DF -> DF): DF == {
        integrand(x: DF): DF == cos fn x;
        integrate(integrand, 0.0, 1.0);
}
\end{verbatim}

In the examples, `{\tt integrate}' is being used with functions which
are either exported from a domain or package, or locally defined.

There is one restriction on the way dummy procedures can be used.  An
\asharp{} function which calls a Fortran routine cannot be invoked
recursively by the call to the Fortran routine.

\head{section}{Calling \asharp{} Routines from Fortran}{asugFortCallAldor}

This is very similar to the way \asharp{} routines can be passed to C
functions.  There is one restriction --- exported functions must be
defined in the top level of a file, not within an add-body.  Of course
the exported function itself may use other functions defined in add
bodies so this is not really a problem.

For example, consider the \asharp{} program:
\begin{small}
\begin{verbatim}
#include "aldor"

import from IntegerPrimesPackage, Integer;

export {ISPRIM:MachineInteger -> Boolean} to Foreign Fortran;

ISPRIM(n:MachineInteger):Boolean == prime?(n::Integer)
\end{verbatim}
\end{small}
This creates a function \verb`isprim` which can be called from
Fortran.  Note that the name of the function must be legal in the
Fortran world (strictly speaking this means no more than six
characters and uppercase, although most modern compilers take a more
relaxed view!).  A Fortran routine to call this function might look
like:
\begin{verbatim}
       INTEGER I
       LOGICAL ISPRIM, B
       EXTERNAL ISPRIM
C
       DO 100 I=1,100
         B=ISPRIM(I)
         IF (B) THEN
            PRINT*,I," is prime"
         ENDIF
 100  CONTINUE
C
      END
\end{verbatim}
Notice that in Fortran only the return type is given, and that the
data type correspondence is the same as that provided earlier.

To compile and link the routines together we might do the following:

\begin{small}
\osprompt \asharpcmd{ -Fo isprime.as}\\
\osprompt f77 testprime.f isprime.o -L\$ALDORROOT/lib -laldor
-lfoam
\end{small}

Note that as well as linking to the appropriate \asharp{} libraries it
is important to link to the \verb`foam` library.

\head{section}{Platform-dependent details}{asugFortPlatform}

The foreign language interface of Fortran is not standardised, and so
there are some details which vary from platform to platform.  It is
possible to tailor an installation of \asharp{} to a particular
Fortran compiler by editing the file
\verb`$ALDORROOT/include/aldor.conf` and setting the values of the
following keys:  
\begin{itemize}
\item {\tt fortran-name-scheme} (indicates how Fortran identifiers are
decorated in C: common cases are {\tt underscore} which means an
underscore character is added to the end of the name and {\tt
no-underscore} which actually means that the identifiers are undecorated);
\item {\tt fortran-cmplx-fns} (indicates the protocol for returning
Complex values via the name of a function);
\item {\tt fortran-libraries} (list of linker options needed to link Fortran
programs to \asharp{});
\item {\tt fortran-io-init-fun} (a function which should be called
before any Fortran routines are invoked, typically to initialise streams
for standard input and output).
\end{itemize}

The configuration file is discussed in more detail in \chapref{asugUnicl}.

\head{section}{Larger Examples}{asugFortExample}

This first example shows the use of a Fortran routine to find a root
of a function (in this case $e^{-x}-x$) in an interval.  Note the use
of {\tt Ref} types to return results.

\begin{small}
\begin{verbatim}
#include "aldor"
#include "aldorio"

DF ==> DoubleFloat;
SI ==> MachineInteger;
import
  { root:(DF, DF, DF, DF -> DF, Ref DF, Ref SI} -> () ; }
from Foreign Fortran;

-- Interval containing root
lo : DF := 0.0;
hi : DF := 1.0;

-- Tolerance
eps : DF := 0.00001;

-- Result
x : DF := 0;

errorFlag : SI := -1;

f(x:DF):DF == exp(-x) -x;

root(lo, hi, eps, f, ref x, ref errorFlag);

if not zero? ifail then stdout << x << newline;
\end{verbatim}
\end{small}


The following code calls the NAG library function {\tt e04jyf}
to minimise the expression 
$(x_1+10\,x_2)^2+5\,(x_3-x_4)^2+(x_2-2\,x_3)^4+10\,(x_1-x_4)^4$
subject to the constraints:
$$
\begin{array}{rcccr}
1       &<& x_1 &<& 3 \\
-2      &<& x_2 &<& 0 \\
-\infty &<& x_3 &<& \infty \\
1       &<& x_4 &<& 3 \\
\end{array}
$$
It uses the {\em BasicMath} library to provide the mathematical
domains.
For more details of what
the various arguments do please see the NAG Fortran Library Manual,
which is available in both printed and electronic versions. 

There are several points to note about this example:
\begin{enumerate}
\item When importing the Fortran routine we have named each parameter.
This is not strictly necessary but makes the program easier to read.
\item Note that in the import statement we have chosen a particular
\asharp{} type for each parameter.  In some cases we are using {\tt Vector}
and in others {\tt Array} since this is natural for our application, even
although the traget Fortran type is identical in both cases.
\item Note the cunning definition of {\tt objfun}, which makes use of
{\tt p} which is defined in an outer scope.  Also note the use of
{\tt Primitive\-Array}s as parameters (see the note earlier) and the
modifiable argument {\tt fc}.
\item The workspace arrays have been constructed using {\tt new}.  The
category {\tt Linear\-Aggregate} defines an export {\tt empty} which
creates an aggregate with no values, but its semantics do not require
that any memory be allocated.  Thus the use of {\tt empty} should be
avoided in cases like this.

\newpage
\begin{small}
\begin{verbatim}
#include "basicmath"
DF  ==> DoubleFloat;
SI  ==> MachineInteger;
VDF ==> Vector DF;
VSI ==> Vector SI;
ASI ==> Array SI;
ADF ==> Array DF;
POL ==> Polynomial DF;
S   ==> OrderedSymbol;
PDF ==> PrimitiveArray DF;
PSI ==> PrimitiveArray SI;

import {e04jyf : ( n : SI, ibound : SI,
                   funct1 : (SI, PDF, Ref DF, PSI, PDF) -> (),
                   bl : VDF, bu : VDF, x : VDF, f : Ref DF,
                   iw : VSI, liw : SI, w :  VDF, lw : SI,
                   iuser : ASI, user : ADF, ifail : Ref SI) -> () ;
} from Foreign Fortran;

import from DF, SI, VDF, ADF, POL, S, List S, NonNegativeInteger;

minimise(lower:VDF, upper:VDF, start:VDF, p:POL, vars:List S):()=={

  n  : SI  := #lower;

  objfun(nn:SI, xc:PDF, fc:Ref DF, iusr:PSI , usr:PDF):() == {
    vals : List DF := [ xc.i for i in 1..nn ];
    update!(fc, ground eval(p,vars,vals));
  }

  -- Workspace:
  user  : ADF := new(0,0);
  iuser : ASI := new(0,0);
  liw   : SI  := n+2;
  iw    : VSI := new(liw,0);
  lw    : SI  := n*(n-1) quo 2+12@SI*n;
  w     : VDF := new(lw,0);

  -- Output Parameters
  f : DF := 1.0; 
  ifail : SI := -1;

  e04jyf(n, 0, objfun, lower, upper, start, ref f, iw, liw, w, lw, 
         iuser, user, ref ifail);

  print << "fail = " << ifail << " minimum = " << f << newline; 
  print << "minimum point = ";
  for i in 1..n repeat 
    outputAsFixed(print, start.i, 8, 4)$FormattedNumericalOutput;

}

lower : VDF := [1.0, -2.0, -1.0e6, 1.0];
upper : VDF := [3.0, 0.0, 1.0e6, 3.0];
start : VDF := [3.0, -1.0, 0.0, 1.0];

-- Set up symbols and build polynomial
vlist : List S := [ +"x1",  +"x2",  +"x3",  +"x4"];
x1 : POL := coerce(+"x1"); x2 : POL := coerce(+"x2");
x3 : POL := coerce(+"x3"); x4 : POL := coerce(+"x4");
p  : POL := (x1+10.0*x2)^(+2)+5.0*(x3-x4)^(+2)+(x2-2.0*x3)^(+4)+
            10.0*(x1-x4)^(+4); 
minimise(lower,upper,start,p,vlist);
\end{verbatim}
\end{small}

  
\end{enumerate}
