%%%%%%%%%%%%%%%%%%%%%%%%%%%%%%%%%%%%%%%%%%%%%%%%%%%%%%%%%%%%%%%%%%%%%%%%%%%%%%
%%%%%%%%%%%%%%%%%%%%%%%%%%%%%%%%%%%%%%%%%%%%%%%%%%%%%%%%%%%%%%%%%%%%%%%%%%%%%%
%%%                                                                        %%%
%%% :: Generic Tie-ins
%%%                                                                        %%%
%%%%%%%%%%%%%%%%%%%%%%%%%%%%%%%%%%%%%%%%%%%%%%%%%%%%%%%%%%%%%%%%%%%%%%%%%%%%%%
%%%%%%%%%%%%%%%%%%%%%%%%%%%%%%%%%%%%%%%%%%%%%%%%%%%%%%%%%%%%%%%%%%%%%%%%%%%%%%
%\newpage
\head{chapter}{Generic tie-ins}{asugLangTies}

Several syntactic constructs are defined in terms of function
application in \asharp{}:
the treatment of literal values,
the application of one object to another,
the updating of objects,
the interpretation of tests in conditional statements and
the mechanism for generating a set of values for iteration.

This allows user programs to define how these syntactic constructs
behave in a particular lexical context. The function calls themselves
are treated as standard function calls, so the normal rules apply.

%\asharp{} has provisions which allow newly defined types to mimic
%nlanguage-defined types in every aspect.  Some of these are syntactic
%equivalences and others are semantic operations.

%%%%%%%%%%%%%%%%%%%%%%%%%%%%%%%%%%%%%%%%%%%%%%%%%%%%%%%%%%%%%%%%%%%%%%%%%%%%%%%
\head{section}{Literals}{asugLangTiesLiterals}
%%%%%%%%%%%%%%%%%%%%%%%%%%%%%%%%%%%%%%%%%%%%%%%%%%%%%%%%%%%%%%%%%%%%%%%%%%%%%%%

Types provide meanings for literal constants by defining
functions named \ttin{integer}, \ttin{float} or \ttin{string}
taking values of type \verb+Literal+. The type {\tt Literal}
represents the source text of the particular literal. Valid literal
values are described in \secref{asugLangExprLiterals}.

The meaning of string-style literals is determined by what operations

\begin{tt}
\verb@    @string: Literal -> X
\end{tt}

are visible, where \ttin{X} may be any type.
In \libaldor{}
the type \verb"String" provides string-style literals.

%Both integer and floating literals are passed in the same way, so it
%is legal to call the function {\tt string} on a floating point or
%integer literal. This can be used to allow numbers to be parsed as
% changed to avoid very bad page break:
Both integer and floating point literals are passed in the same way,
so it is legal to call the function {\tt string} on them.
This can be used to allow numbers to be parsed as
strings. For example, the function \verb"scan", from {\tt
IntegerTypeTools} in the \asharp{} base library, scans a {\tt
TextReader} stream to find a possible {\tt Integer} value.
The following example is a possible implementation to
create an {\tt Integer} from a {\tt Literal}.

\begin{small}
\begin{verbatim}
        integer(l: Literal): Integer == {
                 import from IntegerTypeTools Integer, String;
                 scan (string l)::TextReader
        }
\end{verbatim}
\end{small}

Note that the literal \ttin{l} is converted to a string, and then 
to a {\tt TextReader} input stream before {\tt scan} is called on
the result to form a new value.

%%%%%%%%%%%%%%%%%%%%%%%%%%%%%%%%%%%%%%%%%%%%%%%%%%%%%%%%%%%%%%%%%%%%%%%%%%%%%%
\head{section}{Program-defined tests}{asugLangTiesTests}
%%%%%%%%%%%%%%%%%%%%%%%%%%%%%%%%%%%%%%%%%%%%%%%%%%%%%%%%%%%%%%%%%%%%%%%%%%%%%%

There are several types of expression in which a condition controls
the evaluation of an \asharp{} program:  

\begin{small}
\begin{tt}
\verb@    @if {\it condition} then ... \\
\verb@    @{\it condition} => ... \\
\verb@    @while {\em condition} repeat ... \\
\verb@    @for ... in ... \verb+|+ {\it condition} repeat ... \\
\verb@    @not {\it condition} \\
\verb@    @${\it condition}_1$ and ${\it condition}_2$ \\
\verb@    @${\it condition}_1$ or ${\it condition}_2$
\end{tt}
\end{small}

%% So far we have only seen cases where the condition produced a \io{Boolean}
%% value, but we are not limited to this case.

In many situations, a value can be treated as a condition, even though
it may not be a value from the \ttin{Boolean} type. \asharp{} allows
these to be treated as logical values in the above constructs. If a
{\em condition} above  produces a value of type $T$, different from
\io{Boolean}, and there is a single function \ttin{test: $T$ ->
Boolean} in scope, then that
\ttin{test} function is implicitly applied to the condition value to determine
the outcome of the test. For example:
\keywordIndex{test}

\begin{small}%
\begin{verbatim}
#include "aldor"

-- In "aldor" the type List(S) has a function "test" which
-- returns true on non-empty lists.

-- This function determines which of two lists is longer.
LI ==> List Integer;

longer(a: LI, b: LI): LI == {
        (a0, b0) := (a, b);
        while a and b repeat (a, b) := (rest a, rest b);
        if a then a0 else b0
}
\end{verbatim}
\end{small}

%%%%%%%%%%%%%%%%%%%%%%%%%%%%%%%%%%%%%%%%%%%%%%%%%%%%%%%%%%%%%%%%%%%%%%%%%%%%%%
\head{section}{Generator}{asugLangTiesGenerators}
%%%%%%%%%%%%%%%%%%%%%%%%%%%%%%%%%%%%%%%%%%%%%%%%%%%%%%%%%%%%%%%%%%%%%%%%%%%%%%

If an expression traversed by a \ttin{for} iterator does not evaluate
to a {\tt Generator} value, then the operator \ttin{generator} is
implicitly applied to the expression. This makes loops more readable
if the {\tt for} is traversing, for example, a list or an array:

\begin{small}%
\begin{verbatim}
    #include "aldor"
    #include "aldorio"

    import from List Integer, Integer;

    ls := [1,2,3,4];

    for elem in ls repeat stdout << elem << newline;
\end{verbatim}
\end{small}

The \ttin{for elem in ls repeat ...} of the previous example is
equivalent to:

\begin{small}
\begin{verbatim}
    for elem in generator ls repeat ...
\end{verbatim}
\end{small}

When {\tt List Integer} is imported, then the application:

{\small \verb^    generator: List(Integer) -> Generator(Integer)^}

comes into the current scope.


%%%%%%%%%%%%%%%%%%%%%%%%%%%%%%%%%%%%%%%%%%%%%%%%%%%%%%%%%%%%%%%%%%%%%%%%%%%%%%
\head{section}{Apply}{asugLangTiesApply}
%%%%%%%%%%%%%%%%%%%%%%%%%%%%%%%%%%%%%%%%%%%%%%%%%%%%%%%%%%%%%%%%%%%%%%%%%%%%%%

The application \ttin{a(b)} may be treated as a call to the function
\ttin{apply} with the first argument being taken to be \ttin{a}, and
the remaining arguments being taken from the arguments to the original
application.  Example:

\verb^    ^{\tt f(a,b,c)} becomes {\tt apply(f,a,b,c)}

For example, consider a matrix domain over a ring, $R$. A desirable
syntax for retrieving elements of a particular matrix might be:

\begin{small}
\begin{verbatim}
    mat(a, b)
\end{verbatim}
\end{small}

where {\tt a} and {\tt b} are integer indices for the matrix. To
achieve this, a matrix domain would export a function with signature:

\begin{small}\begin{verbatim}
    apply: (%, Integer, Integer) -> R;
\end{verbatim}
\end{small}

The function is defined in the normal way.  This use of apply is
treated as a normal function call, so for example if an export 'mat'
is in scope, then the one matching the function call context will be selected.

%%%%%%%%%%%%%%%%%%%%%%%%%%%%%%%%%%%%%%%%%%%%%%%%%%%%%%%%%%%%%%%%%%%%%%%%%%%%%%
\head{section}{Set!}{asugLangTiesSetBang}
%%%%%%%%%%%%%%%%%%%%%%%%%%%%%%%%%%%%%%%%%%%%%%%%%%%%%%%%%%%%%%%%%%%%%%%%%%%%%%

If the left hand side of an assignment is an application, the
assignment is treated as an application of the operator {\tt set!}  to
the operator of the left hand side, the operands of the left hand side
and the right hand side.

\verb^    f(a,b) := E ^ becomes \verb^ set!(f,a,b,E)^ \\
\verb^    f a    := E ^ becomes \verb^ set!(f,a,E)^   \\
\verb^    f.a    := E ^ becomes \verb^ set!(f,a,E)^   \\
\verb^    f.a.b  := E ^ becomes \verb^ set!(f.a,b,E)^

As an example, consider the matrix domain above. We would like to assign
into the matrix using a syntax like:

\begin{small}\begin{verbatim}
	mat(a, b) := 1;
\end{verbatim}
\end{small}

To achieve this, the domain should export a function with signature:

\begin{small}\begin{verbatim}
    set!: (%, Integer, Integer, R) -> R
    ++ update and return the previous value of the element
\end{verbatim}
\end{small}

As this is just a normal function, there are no restrictions on the
return type or value. In this case a value from $R$ is returned, and
the description indicates which. Note that the description is purely
for documentation purposes, and not used to interpret the program.

%%%%%%%%%%%%%%%%%%%%%%%%%%%%%%%%%%%%%%%%%%%%%%%%%%%%%%%%%%%%%%%%%%%%%%%%%%%%%%
\head{section}{Bracket}{asugLangTiesBracket}
%%%%%%%%%%%%%%%%%%%%%%%%%%%%%%%%%%%%%%%%%%%%%%%%%%%%%%%%%%%%%%%%%%%%%%%%%%%%%%

An expression of the form:

{\small \verb^[ ^ {\em Expr} \verb^ ]^}

is treated as an application of the operator \ttin{bracket} to {\em
Expr}.

Example:

\begin{small}%
\begin{verbatim}
   #include "aldor"
   #include "aldorio"

   import from List Integer, Integer;

   a := [1,2,3];
   b := bracket(1,2,3);

   stdout << a << newline; -- Produces: [1,2,3]
   stdout << b << newline; -- Produces: [1,2,3]

\end{verbatim}
\end{small}

As can be seen above, this is a useful syntax for creating aggregates
of various types. The value passed to the \ttin{bracket} function in
this case is a tuple of the three values 1, 2 and 3.

%%%%%%%%%%%%%%%%%%%%%%%%%%%%%%%%%%%%%%%%%%%%%%%%%%%%%%%%%%%%%%%%%%%%%%%%%%%%%%
\head{section}{Coerce}{asugLangTiesCoerce}
%%%%%%%%%%%%%%%%%%%%%%%%%%%%%%%%%%%%%%%%%%%%%%%%%%%%%%%%%%%%%%%%%%%%%%%%%%%%%%

An expression of the form:

{\small \verb^    ^{\em Expr}\verb^ :: ^{\em T}}

is treated as an application of the operator \verb"coerce" to {\em
Expr} and restricted to be of type {\em T}:

{\small \verb^    coerce(^ {\em Expr} \verb^)@^{\em T}}

This allows types to define their own mechanisms for converting
between types, and enables types to do appropriate error checking.
